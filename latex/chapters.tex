\section{为什么不为``民主化''投资}

理财这回事,是给焦虑中的中产阶级们服务的。中产阶级这个群体,往好里说是社会中坚,稳定基石,如果刻薄一点,那就是谨小慎微,前怕狼后怕虎,手里有点钱,憧憬过上好生活,但又禁不起风吹草动,尤其是当自己不小心有了更多欲望的时候。

Larry
Lohter在谈到巴西的中产阶级时说,``中产阶级倾向于模仿他们渴望加入的精英阶层。比如在有仆人的家庭,孩子们很少被要求自己收拾东西。''我觉得全世界的欣欣向荣的中产阶级差不多都是这样,就是向往需要使劲够一下才能够到的奢侈生活。现代人早就不把``奢侈''当回事了,但二百多年前,大卫·休谟还要郑重其事地为奢侈这事纠结:奢侈让人沉迷,让人耽于欲望,这就是恶------什么是启蒙呢?就是告诉人类,奢侈是没有办法的事。接受这种``不道德'',由此产生了现代商业文明,简单一点说吧,因为奢侈的存在,这世界就存在着大量的``稀缺'',我们知道,在这个基础之上从此诞生了一种叫经济学的东西。

好吧,我又要说沃伦·巴菲特了。我觉得能够产生伟大价值的公司,应该是那些解决稀缺性问题的公司,它有远大理想和抱负,跑出来说,王侯将相宁有种乎,想办法把稀缺的奢侈品做成大家都能得到的东西,于是它就成了最伟大的公司。但巴菲特不这么认为,虽然他是中产阶级财富梦想的化身,在这一点上我可不认为他做得有多好,他投资的那些公司,在解决稀缺性问题上,做得很不够。当然,这按巴菲特的理论也是可以理解的:因为有可能解决稀缺性,所以它一定会聚集很多的资金,这通常会让更多的逐利者、冒险家进入------在这种情况下,一个公司的真正价值就很难被准确定位,它就不是一个值得投资的好公司。

巴菲特的这些理论当然是不会害你的,但我敢打赌你一定会错过很多好公司。

比如一个叫乔治·伊士曼的富家子弟,他打算去海边旅游,当时这个世界上最时髦的娱乐设备是一架照相机。如果你是一个物理学家懂得一点光学成相的道理,还是一个化学家懂得显影原料的配比,最后你还有钱能置办起一套设备并雇得起为你扛机器的小工,你就可以享受它了。这是一个典型的奢侈品的故事,乔治·伊士曼懂得不是每个人都能把这些元素凑齐了去享受,于是先做了一门工业化生产感光胶片的生意;然后为了卖出更多的胶片,他又开始卖1美元的相机------要知道,它还是稀缺的奢侈品的时候,要卖56美元!

现在这个公司就要破产了,但如果你活在1870年代,你可以有一百多年的时间来投资这个叫柯达的公司,正儿八经地价值投资。

汽车刚诞生的时候,不仅仅是奢侈品,它甚至只是有钱人的一个玩具。但亨利·福特觉得应该让所有人都享受这样的奢侈生活。于是,他发明了廉价的T型车,从此距离不再奢侈。记住,当汽车不再稀缺的时候,是距离不再成为奢侈品了。每个人都可以自由到达远方,这世界于是就变化了,汽车是人类打开的诸多潘多拉盒子之一,因为在奢侈这个领域里,它具有无限延伸性。比如中产阶级的梦想不再是四个轮子两个沙发就能搞定的,他们要更好的车,于是又有了小阿尔弗雷德·斯隆,他发明了汽车信贷,让中产阶级们把旧车卖掉作首付款,然后贷款------你看,资金不再稀缺,信贷不再是奢侈品,好车不再是奢侈品。于是福特的T型车就此被打败了。

再接下来你还有机会。比如你觉得开大卖场的山姆·沃尔顿是发明家吗?汽车那个潘多拉的盒子打开之后,中产阶级们解决了距离问题,从此爱上了郊区的大房子,当居住空间不再是奢侈品,储存条件不再是奢侈品,距离不再是奢侈品,山姆·沃尔顿就成了最好的卖家。为什么他要坐飞机选店址?那可不是为了奢侈,我们知道他出行最爱坐经济舱买廉价票,但他有一个私人飞机队,他只是想找最适合的那个点。不过,我估计有了谷歌卫星地图之后,飞机钱又省下来了。

于是我们可以说到一个重点,信息原来也是奢侈品,比如教士垄断知识和圣经的解释权,但古登堡让《圣经》成为人手一本的读物,受教育权不再是奢侈品,于是新教革命出现了。彭博为什么会在1980年代异军突起,短短几年时间就可以挑战路透、道琼斯的地位?通常的说法是垃圾债券兴起导致业务大量增加,让电脑终端有了用武之地。但另外一个层面则是,以前的承销商靠信息不对称赚钱,把交易的权力放在自己手里,彭博的终端机让非机构也有了交易的可能,是信息传递技术从奢侈品向中产阶级的转移------那些本来垄断在机构投资者那里的专业信息现在每个人花费不多就可以得到。终端的显示屏和传递技术的改善,也包括美国联邦通信委员会的开放,于是电脑不再是奢侈品,传输不再是奢侈品,信息不再是奢侈品\ldots{}\ldots{}现在我们可以说另外一句说了无数次的话,在解决稀缺的道路上,每一步似乎都事关创新,布隆伯格是电视终端和传输技术,比尔·盖茨是计算与存储,创办了CNN的泰德·特纳依靠的卫星传播,这些都是投资价值。

你会发现在解决稀缺性这个问题的时候,还有另外一个词在跟着它们,这就是``民主化'',因为Google的缘故,``信息民主化''被更多的人知晓。我得承认,这一点也不过分。1美元的柯达相机,是娱乐和记录自己的民主化;T型车是交通距离的民主化;沃尔玛是海量日用品的民主化,这些年来那些跨时代的东西,都是由这些号称``民主化''的创业家们完成的。记住Larry
Lohter的话,奢侈生活,是中产阶级的大爱。如果你能提供更多的享受,更多有价值的服务,你就可以有机会获得投资。如果这还不足以让你印象深刻的话,中国有一句古诗,``旧时王谢堂前燕,飞入寻常百姓家''。这就是投资价值。

商业是个政治问题?投资是个意识形态问题?我还没办法完全把它们串连起来,但我记得彼得·德鲁克说斯隆在研究通用汽车的企业管理组织结构的时候,每天必看的书是美国宪法。有什么样的制度,就会有什么样的企业。我们这里的企业家,分为有理想和没有理想两款,后者设计制度的时候可能看得最多的是厚黑学,而有理想的那一坨可能更爱高调表示对毛泽东着作的热爱。当然,这是题外话。

回过头来说,中产阶级被认为是民主制度的一个重要推手和产生根源,而且正如休谟之后亚当·斯密所发现的,这群热爱奢侈生活的人最终成就了这个社会,和文明。所以如果你真的想价值投资这回事,记住,奢侈生活人人热爱,稀缺性永远是一个大事,而背后归根结底还是一个民主化问题------解决了这些问题的人或者公司,你值得为它投资。

\begin{center}\rule{3in}{0.4pt}\end{center}

\textbf{《我在通用汽车的岁月------斯隆自传》}

小艾尔弗雷德·斯隆

华夏出版社 2005年7月

定价:39元

\textbf{《萨姆·沃尔顿自传》}

萨姆·沃尔顿 约翰·休伊

中国社会科学出版社 2008年7月

定价:38元

\textbf{《我是布隆伯格》}

迈克尔·布隆伯格

中信出版社 2010年1月

定价:38元

\section{为什么对莫斯科维茨心生崇敬}

李翔再一次把他收藏的马克·扎克伯格的视频给大家看。那几天Facebook刚刚宣布要IPO,一群人把《社交网络》的盗版碟翻出来,试图重温一下这部电影,在帕洛阿图瞻仰过马克真身的李翔说扎克伯格对这部电影的评价是------只有T恤演得像。我很喜欢这个答案,虽然我觉得这些故事里也没有什么不真实的------这个电影是从爱德华多·萨维林那里出来的,他看到的可能就是这样,这个世界上大部分事情之所以缤纷,可能只是看问题的角度不一样而已。

说说标题中的达斯汀·莫斯科维茨吧。

他是扎克伯格的室友,有关他的故事是这样的:扎克伯格要创业做Facebook这个网站,莫斯科维茨跑出来说,我也要参加!扎克伯格说,兄弟,你不会编程啊!然后莫斯科维茨周末回家买了一本PERL编程入门,``现在我准备好了!''扎克伯格又说,兄弟,网站不是用PERL语言写的\ldots{}\ldots{}但是,如果你遇到一个这么热心的室友你也不会太拒绝吧,于是他分到了5\%的股票,另外两个股东是马克·扎克伯格和前面说到的爱德华多·萨维林。

我一直想把莫斯科维茨翻出来好好研究一下:他为什么把自己的未来捆绑在室友身上呢?光是对室友的了解不能说明所有问题,我怀疑他有另外一种天分------他能发现别人看不到的东西。

好多跃跃欲试要教普通投资者跟风险投资家们学投资的文章都是以差不多相同的让人心潮澎湃的故事开头的。有关Faceb
o ok的通常是这样,它的最早的投资者彼得·席尔拿出5
0万美元搞到6\%的股份,8年过去了,连套现再稀释剩下了3\%------如果Facebook真的达到1000亿的估值,那么他回报是30亿美元。这故事虽然美好,但我们可能真学不来,它貌似可以用一位我所尊敬的媒体人士的观点来解释:``为什么国有企业做不好新媒体业务?因为新媒体相当大的动力和资金来自于风险投资,风险投资讲的是投十个项目等上几年之后可能九个赔了一个赚到了,哪个国有企业敢做这种事呢?国有企业的目的与风险投资正好相反,国资委给的任务是国有资产的保值增值,它们连每年审计这一关都过不去,你让它如何安心投资一个市场和盈利在未来的公司?''

所以这里可以插播一条投资指南:如果你投资跟风险投资竞争的新媒体、新业务,一定要看它的出身,如果来自于传统产业或者过分强调它的国有企业的资源,那么你要当心------恕不举例说明。

广告完毕。即使国有企业没给你赚大钱的机会,你也慢着点嘲笑它们,虽然它们看着总是不合时宜,但你自己小本经营的理财事业在理论上与国有企业没有什么两样,你投资理财的目的也不过就是年终岁尾的保值增值。即使你像彼得·席尔一样有魄力,但可能你早就把它给卖了。很久很久以前,在股市还只有10
0多只股票的时候,我的一个朋友说他那一年赔了60多万元------这是一个天文数字,他的计算方式是,他把卖掉之后又涨了的股票,这中间的差价计为自己的损失。所以如果你即使运气好到有马克·扎克伯格这样的室友,最终你还是会算计着你少收了三五十亿美元。

我们之所以在纠结于到底是赔了60万元还是赔了30亿美元这些问题,其核心还是在于投资者或者合伙人对创业者或者老板的信心的问题。我一直觉得如果你要真是有志于投资界,与其费那么多的工夫去考察基金经理、理财师或者什么机构,还真不如去好好琢磨一下公司老板这个人。不管你是做投资,还是给人打工,跟对老板才是最重要的事。

马克·扎克伯格身上被贴了无数个类似于黑客、理想主义者、阴谋家、创业小青年等等的标签,但我觉得他最大的长处还是做老板的能力。我对他印象最为深刻的可能是一个小细节:如果他想找到好的员工,一定要去谷歌的大门口那里去堵。谷歌那个时候正是如日中天,股价直奔700美元,是全世界最有创新精神的也是最有前途的大公司,而那时Facebook还只是个给美国大学生、中学生提供泡妞平台的小网站。这种眼界可不是一般人所能有的。而其后挖来AdWords的开发者、谷歌的灵魂人物雪莉·桑德伯格,也就顺理成章了。一个老板知道自己的竞争对手是谁,这是外向格局。

再比如他面对错误的能力,大多数时候他坚持己见,可他一旦发现酿成了错误,他承认错误绝不拖泥带水。在我理解起来,面对批评或者错误的应变能力才是老板素质里最关键的一个。这是内向格局。

当然扎克伯格的格局还不止这些,比如掌控研发的速度,推出新服务的节奏,进入新市场的节奏,这些我们可以理解成是审时度势的能力------当初史蒂夫·乔布斯比起来可差了很远。

面对这样一个老板,彼得·席尔就以风险投资家的敏锐不断为自己找到可能性,你要知道在8年时间里还能保留一半股份,那不是件简单的事。而爱德华多·萨维林既不相信扎克伯格管理公司的能力,也不相信他赚钱的能力,还认为哈佛的文凭比去硅谷更重要------他在判断自己的老板的时候,就没有什么优势可言。结果呢,就是首鼠两端,觉得自己有能力的时候就滥用权力控制财务(大小股东最爱祭起的一件利器),出局之后又哭着喊着要把股份抢回来,这就是没有信心的标本。他几乎还口述了一本扎克伯格的发家史------这时候你连他的人品都要怀疑了。

这也是我对义无返顾的莫斯科维茨心生崇敬的原因。你想,你的一个大学室友,搞了个泡妞网站,然后你就要参与进去,是不是觉得有点过分?然后赶上个暑假,你就跟着一起去硅谷租房子去了,然后你还跟着他辍学了------虽然扎克伯格是哈佛的,可以增加一点保险系数,但莫斯科维茨也是哈佛的啊。

我还是想把莫斯科维茨的这个选择看成是一种超凡能力,这能力由扎克伯格的老板天分、泡妞的人类最基本需要、互联网的发展方向,当然还有莫斯科维茨对自己的判断\ldots{}\ldots{}这些东西来组成。你看,他判断准确,而跟他们一起的另一个室友因为听了他爸爸的话(他爸爸可能对以上所有都持否定判断),就错失了机会。

最后还要说一句题外话,虽然对莫斯科维茨心生敬意,但真正的大人物是乔布斯的创业搭当史蒂夫·沃兹尼亚克,他对乔布斯的所有判断都没有什么错误,但他就是不care这些东西,于是他就辞职走人了。并且,虽然他认为乔布斯背叛了极客的分享开放原则,但他一样会在苹果发布新品的时候,兴高采烈地连夜在苹果店前排队------对他来说,这可算不上什么难为情的事。沃兹告诉我们的是,嘿,有关钱和成功的事,你可能想得太多了。

\begin{center}\rule{3in}{0.4pt}\end{center}

\textbf{《重返小王国》}

迈克尔·莫里茨

中信出版社 2010年12月

定价:46元

\textbf{《Facebook:关于性、金钱、天才和背叛》}

本·麦兹里奇

中信出版社 2010年5月

定价:29元

面对高达一亿美元的估值,有人心存疑虑望而却步,有人却甘冒风险毅然出资。

\textbf{《Facebook效应》}

大卫·柯克帕特里克

华文出版社 2010年10月

定价:49.8元

\section{我们这个时代最智慧的头脑}

有这么一个公司,笃信自由市场理论并在这个自由市场里表现得游刃有余;它还有强大的创新能力,不断地在自己领域里创新商业模式,颠覆既有的传统规则;它们还有明星CEO,给自由市场、创新打气;并且它还是在《财富》500强中能进前十名的大家伙\ldots{}\ldots{}对了,它还是清洁能源的推广以及开发者,代表人类绿色的未来------当一个这么好公司摆在你的面前,你觉得怎么样?你投资吗?

如果我是一个投资者,面对这样一组Tag,面对这个华尔街的尤物,我也会义不容辞地进去捞一票。如果我还是它的员工,哈利路亚,各种顶级的福利和补贴,各种丰盛的养老金计划,这真是前世修来的福分啊。

当然,这些都是2001年前的事了。在那一年里本·拉登撞烂了世贸中心的大楼,一个月之后的10月16日,这家公司的高管们撞烂了对自由市场的信任和膜拜,撞烂了对创新的憧憬,这家公司就是安然。

这个世界总是有些事不那么笃定。台面上领风气之先,一派风光,但对于这些公司来说,永远要替它们捏着一把汗。如果你身为一个投资者,那可能就惶惶不可终日,细想一下成天为它们吹牛而提心吊胆,这种事还真是我等小户的悲剧命运啊。我最想说的是,如果有公司跟你说它们在改变世界,你可千万别以为它们是乔布斯的转世灵童。

但我们就是在这样一个年代,自从乔布斯的``活着就要改变世界''成为世界的座右铭,我们对大公司的胃口就完全被吊起来,所有做手机的如果不揣着一个改变世界的梦想,简直就没脸在同行中混,非把自己卖出去才行。

我觉得我们有必要审查一下过去这些年改变世界的这些事。先重温一下拉里·佩奇在当年谷歌上市之初说过的那段最着名的话:``谷歌不是一家因循守旧的公司,我们也不打算成为这样的公司\ldots{}\ldots{}''自从佩奇以来,这几乎已经成了有志于改变世界的那些公司的标配,相当于创新公司的``我有一个梦想\ldots{}\ldots{}''马克·扎克伯格则会说:``我从来没有想过要运营一家公司,对我来说,商业只是一种完成事情的方式。''他们这些人的共同前辈杰弗里·斯基林则说,``一家过于担心成本的公司会限制创新思维''。这位前辈是谁?如你所料,他是安然的CEO。如果在2001年的上半年,在这两个人中间选一个骗子的话,估计十有八九佩奇这个犹太人会当选。

好在大多数时候我们不用去想关于如何处理到底谁是骗子的问题。万一改成了呢?比如瞧不起卖糖水更愿意改变世界的乔布斯,在他回到苹果的十几年时间里,给苹果带来4000\%的增长,这一样可以当成我们的一个大Bonus,如果你不想跟着这个疯子改变世界,那你就跟着巴菲特去投资糖水,30年时间也一样可以成为首富。再题外话的是,自从乔布斯过世、苹果的市值又成了世界第一,就一直有人在拿苹果那个倒霉催的第三个创始人罗恩·韦恩打镲,按苹果市值算下来,这个拥有10\%股份的人,怎么说也损失了几百亿美元吧?我看韦恩在接受采访的时候倒还实诚:应该损失了几千万美元,如果不是隔几天就卖的话,谁知道他们用了不到四年时间就鼓捣上市了,如果知道这么短的时间,我就忍下来了!但他也表现出了另外一种不实诚,``史蒂夫有他的改变世界的方式,我有我的。''天知道他怎么改变世界,由此可见把``改变世界''这样的话挂在嘴边可能是这个世界最简单的几件事之一。

当然,如果说这些科技奇才们如何改变世界,有一点倒是货真价实的,就是他们建立起来的不给投资者分红的惯例。我一直没有弄清楚这对世界的改变究竟有多大,或者它是一个恶因还是恶果。你这么想,你有一个邻居,有一天他跟你说,哥,我想把门口的饭店盘下来,手头钱不够,你借我1万元,年底还,二分利息。你权衡一下,觉得这事还能做,你就变成了他的债权人------你手里的借条就是债券。年底的时候你拿钱就好了。但你的邻居可能还是一个善于创新的人,他跟你说,哥,我想把门口的饭店盘下来,手头钱不够,我们一起做吧,你给我拿1万元,年底怎么也能赚个三两千。你权衡一下,也值,这个店至少有1万元钱的股份从此就是你的了。这是证券,证明你的产权的。

我说的一个恶行是,你这个邻居跟你手里拿了1万元钱,然后年底了跟你说:哥,现在这生意不错,我们还得把隔壁盘下来,你这钱放你手里又没有什么用,我们今年不分红了吧!然后,再一年,他又跟你说,你看我这生意已经值几个billions了,我们继续投资吧。

久而久之,你作为一个投资哥,除了把手里的证券转给别人之外,没任何办法,因为借钱的兄弟认准了,赚了钱分红给你是资源浪费,因为你这个猪头搞不定资产增值这回事,只有他才可以为你的钱做好主。所以最后你就只剩下炒股票这一种办法了,我还有一个没弄清楚的是,就是这帮高技术公司自从不分红之后,就逼得投资者只剩下在二级市场炒股票这一条赚钱办法了,这是好事还是坏事?

谷歌的恶行则是,它再也不跟你客气了,明确地跟你说:猪头,把你的钱拿给我,我来改变人类获取信息的方式!但我没打算给你分红!把话说得这么明白?即使我们真的是猪头,也不要这么明说啊!于是连投行的那些聪明人都跟着信了,股价从100美元直接升到700美元。虽然这是一单很漂亮的生意,但我还是得说,猪头啊,从此人类中的那些杰出者就开始都用这种语气跟我们这么说话了。比如Facebook说,把你的钱给我,我来重新组织人类关系!Evernote说(或者打算说),把你的钱给我,我来做全人类的大脑外挂,让你的大脑先云起来。

但是你说这些人改变了什么,有的时候我还真是困惑。这些年里这几个最着名的大公司,如果分个高下的话,那么基本上的顺序,我觉得是微软,苹果,谷歌,Facebook。微软是新旧两个时代交接的时候最伟大最规矩最纯粹的公司,它真的是在卖东西,而且赚了钱!比尔·盖茨是最纯粹的企业家,我很喜欢。乔布斯也这么表达过,然后他又说,微软最大的问题就是产品太丑了,盖茨这个人早就应该去做慈善了,嗯,他慈善做得很帅。你要说为什么苹果被我排在第二位,我觉得单凭这句话和那4000\%就值得这么做。而谷歌和Facebook,我相信它们可能是改变了世界,可改变了世界又怎么样呢?

Jeff
Harmmerbacher,一个数学家,他在Facebook负责数据挖掘和分析,有一天他辞职了,他说,他容忍不了这个世界上最聪明最智慧的头脑做的工作,只是为了怎么让人更多地点击他们的广告。嗯,这也是我想表达的,佩奇啊或者扎克伯格啊,他们做的事的确看起来是改变世界的样子,但是,那也就是一个卖广告的生意啊。

\begin{center}\rule{3in}{0.4pt}\end{center}

\textbf{《被谷歌》}

肯·奥莱塔

中信出版社 2010年9月

定价:39元

\textbf{《乔布斯传》}

沃尔特·艾萨克森

中信出版社 2011年10月

定价:68元

\textbf{《房间里最精明的人---安然破产案始末》}

贝萨妮·麦克莱恩,彼得·埃尔金德

中国社会科学出版社 2007年2月

定价:42元 \# 乔布斯的One More Thing \#

如果史蒂夫·乔布斯还活着,每半年就搞点新意思出来,我觉得也是一件挺难的事。当然,他有``扭曲现实力场'',所以他可以让人跟着他的魔法棒一起激动一下,比如在最近一次的新一代iPad发布上,我就会想,当他最后说``and
one more thing\ldots{}\ldots{}''
的时候,会不会把那个iPhoto说得天花乱坠,说不定就会这样。所以老实巴交的宅男Tim
Cook老老实实地把视网膜显示技术、新的图像处理这些东西都一骨脑地说了出来,总是让人感觉缺了一点激动人心。

其实如果从正常经营的角度看,Tim
Cook我看可以说正领导着一个史上最牛的公司隆隆向前。如果从战略家们那些高屋建瓴的角度看,苹果公司有好的产品,在全球范围内建立起了庞大的供应链,有成熟的研发,好像还有一个不错的更有利于确定决策的团队(那个搞一言堂法西斯不民主不nice的乔布斯终于不再指手划脚了);
如果从苛刻的财务家或者分析师的角度看,现在也很完美,巨大的收入总额,好看得不能再好看的利润,还有源源不断的巨大的现金流,而库存采购这些成本完美得近乎数学公式\ldots{}\ldots{}作为一个冷静理智的商业的信仰者,似乎没有任何理由来怀疑无辜的Tim
Cook。

但是,做乔布斯的接班人,搁谁身上都是一件悲催的事,传说中谁都想要、但谁都会怕的那种活,就是这个接班人吧?

好像是杰克·韦尔奇说,每个CEO上任第一天就开始的工作应该就是寻找他的接班人,我很同意这句话,对于一个公司来说,它最重要的产品当然就是它的团队,而团队这个产品在公司中由哪个生产部门负责,当然也只能是CEO。如果我们再向前走一步,一个公司所具备的价值,表现在未来的收益可能之上,所有现在的财务报表证明的都是过去所取得的成绩,而投资者买的是一个公司的未来。也是从这个角度上看,史蒂夫·乔布斯临终之前,从口袋里掏出来的那个``one
more
thing'',应该就是他的接班人和苹果的未来,这样看起来,这个最后的thing就不是那么成功。

为什么我们现在会把对想象力的赞美投射在苹果这样的公司身上------有足够充分的资本,又没有倾家荡产的担忧,心无旁鹜,可以任意想象和谋划未来,这当然是公司创造力的最佳状态。但是对于投资者来说,也无形当中增加了风险。不过,人类么总是对高风险高回报有着非同寻常的热爱,于是风险投资们热爱的那些高技术股票,就成了普通投资家的新宠。炒得越多,
价格越高。虽然你有可能比别人聪明那么一点点,投资回报、风险系数之类、退出机制之类的都不用去管,把亏损留给别人,但你还是要记住一点-普通投资之所以不要跟风险投资们跑,是因为风险投资是别人的钱,而你的钱就是你的钱。

所以当你像一个VC一样跟着投资未来的时候,当你把希望寄托在不断出现的``one
more thing''的时候,你对那个接班人就更不能掉以轻心。我不是太看好Tim
Cook的原因还在于曾经有过一个叫约翰·扬的人,这个人继承了戴维·帕卡德的CEO,帕卡德就是HP的P,惠普的创始人。

而约翰·扬与Tim
Cook有许多共同点,深受信赖的老臣、对企业运营有深刻的理解和把握、像创始人一样爱自己的企业和企业文化、差不多是所有候选人里最好的那一个\ldots{}\ldots{}但结果怎么样呢?他最终还是一个过渡角色。江山易主,改朝换代,所有人都会敏感起来,不光内部要重新组织起来一个中心,外部也一定都是质疑之声,而以往乔布斯或者帕卡德他们犯的小错误,最后也会被无限放大。而接班人最容易犯的错误实际上跟投资者和旁观者是一样的,就是``锚定效应''的错误-接班人总是考虑做得像不像前任,而投资者更是把前任当做了标杆,你Tim
Cook已经两次没有从口袋里掏惊喜出来了,那还有创新吗?

惠普在约翰·扬之后,请来了卡莉女王,卡莉·菲奥莉娜可不是一个简单的人,
她是华尔街的最爱(你就当投资者跟华尔街是一回事好了),而她也知道这是她最重要的武器,她还没有约翰·扬的各种负担-于是她搞了一次惊天并购,惠普并购康柏,结果这最终惹怒了董事会和势力犹在的休伊特家族(HP的H)。虽然事后看起来,卡莉女王任期内的这个最重要的事,
决定了惠普由此进入消费电子领域,否则在2000年之后的这个消费电子主导的世界里,惠普可能还只是一个隔岸观望的半吊子军火商。你看,即使大家都意识到这一点,大家都是为自己的腰包考虑,并且大家还有能力做董事,但最后大家比的还是个人感情上的好恶。还有就是,大家都在比谁更代表``惠普之道''。一旦这东西变成一面旗,每个人都在想着当旗手,那也一定要记住一点,解释权比解释更重要。

未来我觉得一定也会出现一批试图解释乔布斯精神的专业人士,就像``惠普之道''的解释权一样热门。惠普虽然诞生在帕洛阿图的车库里,但一直被认为是最不硅谷的企业之一,一是因为它诞生得早,二是硅谷的传承与它没有什么关系。按照公司史研究者米克斯维特·伍尔德维奇的说法,硅谷熙熙攘攘你方唱罢我登场,但实际上可以理解成只有一个公司,这个公司不断分裂,不断延展,肖克利的实验室从东海岸过来,于是有了仙童半导体,然后有了英特尔,英特尔又培育了苹果、微软(虽然它不在地理意义上的硅谷),然后又有了网景、雅虎、谷歌、Facebook,至少在现在看来,
它还会一直分裂下去,所以你经常会看到风险投资商们突然间就疯狂起来,把4100
万美元砸向一个莫明其妙的Color的应用。不是他们疯狂了,因为谁也不知道下一次细胞分裂里哪一个会长成大咖。

所以你可以说,硅谷的One More Thing,就是吉姆·克拉克说的New New T h i
ng。你要说它们的产品,可能是罗伯特·诺伊斯,也可能是马克·安德森,还有可能是杨致远或者拉里·佩奇。他们没有一个人愿意等半辈子做一个接班人,但他们恰好就是接班人。这是硅谷最重要的产品。

最后我要说的是,未来最不硅谷的企业可能就是现在最代表硅谷精神的苹果。不是因为它太大,而是因为它太自成系统。我们总是担心苹果的封闭会让苹果裹足不前,但封闭从来是企业最重要的特征,苹果越来越像一个大企业,而硅谷的开放带来的意外之喜则会逐渐减弱。回到我们最初的那个问题,为什么Tim
Cook看起来不灵?
就是因为他继承的是一个封闭的公司。而在乔布斯那里,封闭之所以显得不那么重要或者不会让人产生与硅谷格格不入的想法,因为他每次从口袋里掏出的那个One
More Thing并不简单是一个产品,而是New New Thing,而是创新。

\begin{center}\rule{3in}{0.4pt}\end{center}

\textbf{《硅谷巨人》}

汤姆·珀金斯

中国青年出版社 2008年8月

定价:36元

卡莉·菲奥莉娜是一位典型的超级CEO, 她认为``惠普之道''
不再是扭转公司局面的关键因素, 相反会成为公司发展的一种巨大障碍。

\textbf{《两个人的帝国》}

迈克尔·马龙

中信出版社 2008年1月

定价:39元

\section{不要跟自己的肺生气}

``主张在投资者的证券组合中采用某种机构的方法,调整债券与股票之间的比重。

它的主要好处在于,让投资者有事可做。随着市场上升,他将不断地出售所持有的股票,并将所获收入投入到债券中;当市场下降时,他会采取相反的做法。这些交易活动将提供某种通道,以释放投资者有可能不断累积的能量。作为一名投资者,他能从下列想法中获得满足感:自己的业务操作与普通大众的正好相反。''

``不景气时,没有人会看一季甚至一个月以后的东西,没有人相信一季或一个月以后的预测值,这时候,股票值多少钱就得看公司有多少现金,而不是公司未来可能赚多少钱,那根本没什么价值。反过来说,有时候,市场的存续期也会比两年更长。''

``我常常说,在正处于上升状态的市场中买进,是最舒服的股票买入方法。请注意,关键在于不要一门心思想着买得尽可能便宜,或者卖得尽可能最高,而是一定要买在或卖在正确的时机。''

按照某种达人总结出来的理论,价值投资那伙人个个都是沉闷而且乏味的家伙,而大投机家们的一生则过得精彩纷呈,这表现在他们分头留给世人的着作里,前者也总是无聊透顶的财务分析,而后者恨不得写成充满悬念的惊悚小说。但是,如果仅仅是把上面三段话摆在那里,然后让你判断这三句话的主人,你还真不是那么容易就得出结论。

我所知道的,第一句话来自于价值投资的祖师爷式人物本杰明·格雷厄姆,第二句是摩根士丹利的前分析师安迪·凯斯勒,我们通常把他定位成是给机构投资者提供服务的人,第三段话则来自于杰西·利弗莫尔,号称是美国历史上最牛的投机大师。利弗莫尔15岁就开始就着股票价格猜走势,22岁的时候已经是大名鼎鼎人见人怕的投机界知名坏蛋,当然也已经破产了好几回了,但你看他说的话,``正确时机'',这有多中肯。当然,如果再多引用一句他就露出马脚了,``当我看空并卖出某只股票时,每次卖出的价格都必须比前一次卖出的价格更低。当我买进的时候,情况正好相反。我必须按照步步上涨的方式买进。我不按照步步下降的方式买入做空,而是按照步步上涨的方式买入做多。``关于投资与投机,有各种各样的区分办法。我觉得有一种可以考虑进去,就是如果你的优势是长于分析企业和财务的人,那么你做价值投资可能靠谱一些,如果你是一个更善解人意更乐于揣摩别人心事的人,那你没准可以做个投机者。股市这东西,如果参与者都是格雷厄姆的信徒,那真是一个挺悲惨的景象,所以一定要有一批人成天想着的事情是''我知道你怎么想,所以我有机会赢``。但是,慢着,为什么格雷厄姆也要说''自己的业务操作与普通大众正好相反``呢?

我能感觉到不管是利弗莫尔还是格雷厄姆,在他们心中都有着一大坨叫''大众``或者''普通``的人,利弗莫尔把他们叫做''羔羊``,格雷厄姆只说他们那一小撮''聪明的投资者``应该如何做,其他的人,嗯,格雷厄姆可没说人傻。

利弗莫尔给出的解释是:''当帝国钢铁每股100美元的时候,每个人都要买进它。

为什么不呢?人人都知道它是一只好股票,它曾经是,即便现在买也是折扣价。证据便是它的上涨行情。一个股票既然能够从70美元上涨30个点,就能从现在再涨30点。

许多人按照这种方式思考问题。``许多人,说的就是你,不管是价值投资者还是投机大师,发现了你,就发现了赚钱机会。利弗莫尔还说,许多人(说的也是你)以为既然某只股票曾经达到150美元的高点,那么当它回落到130美元的时候必定是便宜的,到120美元的时候甚至是了不得的折让了。想一想2008年的平安保险吧,那些一路从160元抄底一直抄到二十几元的人,格雷厄姆过的是生活,他们过的那也叫生活吗?

好吧,光控诉生活不公是没有出息的表现。其实利弗莫尔、格雷厄姆或者是安迪·凯斯勒都是非常nice的人,他们好像从来都不吝于分享他们的原则。你要知道,这个股市之所以没有因为他们的经验分享而死气沉沉,这还真是要感谢我们这些羔羊的贡献呢。

比如,利弗莫尔这个机会主义者告诉我们最大的问题是不要跟自己作对,``投机的行当肯定不属于自然的商业行为,原因在于我发现普通投机者的天性都是同他自己作对的。所有人概莫能外地倾向于出现的一些弱点对投机成功构成了致命的威胁------通常正是这些弱点才使他能够讨得他的同伴们的青睐。''有点复杂是吧?利弗莫尔把自己做的蠢事都跟我们分享了,他做期货,棉花赔钱,小麦有利润,这时他面临的选择是两个出清一个,``我竟然做出一个完全错误的抉择。棉花带来账面亏损,我留着它;小麦给我带来账面利润,我卖掉它。

这真是愚蠢透顶的做法。在投机者铸成的所有大错中,几乎没有什么再比企业为已经亏损的交易摊低成本更要命的了。永远要卖掉账面亏损的头寸,保留账面赢利的头寸。显然这才是明智之举。''你是不是也经常会好容易有了一个盈利的股票,然后你想着落袋为安,把那只还跌个不停的股票留在手里?你确信真是为了未来翻盘的可能性吗?这在人类行为中,一般叫做赌博。

赌徒利弗莫尔告诉我们的真理是:职业赌徒并不指望长线挣大钱,而是指望挣钱有把握。这个时候,我就会想到一个困惑自己好久的问题,赌徒到底是理性的还是非理性的,或者说,为什么我们在提到``投机''的时候,会天然地认为它是非理性的?

实际情况当然不是这个样子的,在他们这群人之后,在信息技术发达起来之后,有一个词叫``市场定价错误'',他们都是靠这个东西赚钱的。能发现定价错误当然不是靠一腔热血猜骰子点数赢的。

我前面说过了,如果你更善于揣摩别人心思,你就具备了投机者的天分,但杰西·利弗莫尔可不仅仅是这样,事实上他告诉我们,作为一个伟大的投机者,你要做的功课应该更多。他提到一个叫罗塞尔·塞奇的家伙,是个``百折不挠的怀疑主义者'',坚信必须提出自己的问题,通过自己的双眼来观察,他听说一家铁路公司的总裁是个铺张浪费之徒,于是闯上门去,他看到总裁先生如何在雕印着双色精美公司title的亚麻纸上随手写上一个盈利数字或者写上它们是如何压缩运营成本的------然后就丢在纸篓里\ldots{}\ldots{}光听说是没用的,你还要闯上门去;光说什么是不重要的,你还得看他在做什么。简而言之,你需要搜集和消化与该公司相关的所有东西,了解那个公司的基本面,预期它的获利是否会增加,这些事一向是安迪·凯斯勒那一卦爱强调的。

作为一个价值投资的信奉者,我得说,在这个世界上虽然投机者的名声不那么好,做空人公司这事也挺耽误事的,但他们也确实在客观上让这个世界变得好了不少。并且虽然他们不能让这个世界变得更聪明,但至少还能让我们知道自己傻。

你知道这有多重要!

杰西·利弗莫尔说:``你看看,很多人徒有精明的名声,他们之所以看多,是因为他们已经持有股票。我不允许手上已有的头寸------或者先入为主的成见------为我做任何思考。我再三强调我从不和行情滞怠争论,这就是缘故。因为市场意外地、甚或毫无道理地对你不利,你便对市场生气,那就像得肺炎的时候对肺生气一样荒唐。''

\begin{center}\rule{3in}{0.4pt}\end{center}

\textbf{《股票大作手回忆录》}

埃德温·勒菲弗

万卷出版社 2010年11月

定价:48元

\section{聪明的投资者}

托马斯·伍德罗·威尔逊,一个美国总统看到了得克萨斯的石油富豪们是如何呼风唤雨的,心里多少有点酸溜溜,他说:``你知道,上帝把全部石油放在地下,然后来了一个人,他做任何事情都没有成功过,结果他把石油开采了出来。从那一刻起,他就觉得自己在哪一方面都是专家,不管是政治,还是裙子。''

我现在还挺担心这种情况再度发生的,换算成我们的语言可能是,``你知道,党和国家把煤埋地下,然后来了一个人,没给国家做过任何贡献,把煤挖了出来,从那一刻起,他就觉得自己在哪一方面都是专家,不但当了人大代表,而且还懂了裙子。''他很有可能还懂了电影、电视,还懂了媒体\ldots{}\ldots{}这种事可不算什么新鲜事,能源大亨们热爱娱乐传媒业,可不光是在美国,这些年我听到看到的也不少。他们通常都来势汹汹,张嘴便是这样的---``我们投资一个亿,我们不用想钱的事,我们打算大干一场。''噢?羡慕嫉妒恨的我总忍不住大惊小怪,难道已经有人进化到不心疼钱了吗?一般情况下,对面的人会搓搓手,半羞涩半得意地说:煤老板啊,试试看吧,替他们花点钱吧。

如果较起真来,凡是有人说``钱不是问题'',那基本上可以断定这里的问题大了去了。找工作这事,就是拿自己做投资,我们在拿万八千元钱做投资的时候脑子里都要想一下,长线还是短线,投资还是投机,纠结半天,但轮到自己的这个最大的本钱的时候,糊涂人就多了。狠狠地赚上一年半年煤老板的钱,别的管它呢!我得说,这可不是聪明的价值投资者应该想的事。找工作跟买股票差不了太多,必须问自己一些基本的问题。对于媒体来说,有个相当重要的问题就是这个资金的背景究竟怎么样。

这里面至少包括四个问题:投资者背景是不是周期性行业,是不是高风险行业,是不是政策变化会比较剧烈的行业,对于投资者来说是不是主业。如果这些问下来,各个都是肯定答案,那这笔投资要画一个大问号。煤老板不管多有钱,多有新闻理想,表了多少不干涉内容的决心,那也不要太相信。

好吧,这听起来太像吐槽同行了。我们跳开来说点普适的道理。不管是投资还是找工作,如果出资方号称不缺钱,祭出福布斯排行多少位,或者神秘兮兮地说``那些首富,切\ldots{}\ldots{}'',你首先要判断的是,``钱''这个概念:它可能是一捆扎好的钞票,也可能是资产,也可能是投资出去的股份,还有可能是可探明的储量折算出的未来的收益,当然很多时候它还是负债。我得说,很多时候我们会很善良地把它想象成是一个装满了钱的ATM机。膜拜一个非主业的煤老板,就像膜拜在银行下棋的奥特曼---有房、有车、有钱、还有帅,还能保护我---视频网站也好,媒体也好,电影也好,这里凝聚的善良的文艺青年们,通常就直接把``钱''当成现金流供起来了。

不过话说回来,也不怪文艺青年们善良,我怀疑有一点是真的,就是热爱娱乐传媒业的煤老板们真的是拿现金流来投资副业,并且通常是多余的现金流。所以我们开始问的那四个问题就显得尤其重要了,前三个的变化都有可能导致现金流大为减少,然后第四个问题就出现了,老板们如果遇到困难,可想而知,首先要砍掉的就是那些美轮美奂的度假村酒店项目、那些许诺了投资拍了一半的电影、还有可能就是那些可有可无的媒体\ldots{}\ldots{}你知道,它的实际功能也就相当于一个耀武扬威的悍马啊。

我还是克制一下自己刻薄同行的那点小心眼吧。我想说的是,太阳底下无新鲜事,看得克萨斯四大石油家族的故事,感觉真的是进入到了鄂尔多斯或者山西煤老板们的世界,暴富、保守、喜欢明星、投资媒体、买牧场(在我们这里主要表现是买潘石屹的房子)、富二代、奢侈\ldots{}\ldots{}``媒体对格伦·麦卡锡的报道催生了一个新的文化符号:孤星州的花花公子,乘飞机考察喷油井的间隙和小明星们谈恋爱的石油商。在二战之后的岁月里,许多为这种形象添枝加叶的故事源自好莱坞。像每一代美国人中的暴发户一样,麦卡锡和其他富裕的得克萨斯人被那里的浮华和魅力所吸引,而且他们也受到渴望资金的电影制片人和一群群愿意结交有钱老男人的年轻女演员们的欢迎。正如1954年时好莱坞一位专栏作家所说的,`电影对得州百万富翁的吸引力就像蜡烛对飞蛾的吸引力一样。'\,''

这些标签不重要,我觉得太阳底下无新鲜事更应该包括以下这些:当油价降到2美分一桶的时候(别算,伤不起这个心),周期性行业导致现金流断了;当政府规定每口井每天只能出100桶油的时候(哪里都有一些匪夷所思的政策),政策变化导致现金流断了;富二代中最聪明的那个家伙跑到利比亚去找油,并且探明储量足以让他成为新一代首富,卡扎菲上校给国有化了(这事发生的概率不要太大吧),高风险行业的风险可不是说着玩的;最倒霉的卡扎菲受害者也是最败家的,他不知道从哪里看到了一本类似于《货币战争》这样的书,打心眼里相信共产主义者、犹太人和洛克菲勒家族正在搞一个世界级的阴谋,于是他去投资白银,不但投资而且把那些银子真的买回来存家里,于是他一半的财富就都被自己搞出来的阴谋论搞没了---这是说,找矿这个主业他很灵,但副业这事可不那么容易搞定。

当然,好的东西也有,当市场景气的时候,他们这些金主还真是为人类留下了一些好东西,小一点的比如小克林特·默奇森的美式足球,达拉斯牛仔队和他创办的超级碗---这是一个标准的副业,他做得很成功,虽然,最后他也因此破产了。

唯一一个成功的例子倒是符合专业性的,理查森家族的希德·巴斯,他继承了微不足道的5000万美元,他跟他的合作伙伴理查德·雷恩沃特说:``我想在可靠的公司投资,有值得信赖的管理者,而且是长线投资。''``你说话的口气听起来像沃伦·巴菲特。''``谁是沃伦·巴菲特?''于是两人开始研究奥马哈这位投资圣人的文章,然后是本杰明·格雷厄姆。``你一生中一共就那么多好主意,为什么不把资金投在你真正相信有利可图的地方去。''在这个故事中,是他们看准了迪士尼的价值,用5亿美元投资让自己成为迪士尼的大股东,然后请来了迈克尔·艾斯纳做总裁,最后变成了28亿美元。

对了,这对聪明人还参与了风险投资,1970年他们请来了风险投资家大卫·邓恩。邓恩在考察了一圈之后得出的结论是:``我一生中从来没有见过这么多的钱掌握在天赋这么低的人手中。''十年之后,邓恩把巴斯家的钱从800万美元变成了2亿美元。

你一定看出来了,前面一个讲的是我们老生常谈的投资与投机的故事,后面一个讲的是专业人士的风险投资故事。

最后讲一个故事。理查森家族的一位世交证实,从``二战''期间开始,艾森豪威尔和理查森就开始共同投资的传言是真实的。这位世交说,``我知道他是这么做的,希德告诉我的。石油圈里有一个古老的游戏,你知道吗?你的一些朋友,他们只投资到你的好井,而不是坏井?你明白吗?艾森豪威尔就是用这种方式。你永远无法证明这一点。但他确实是这么做的。''

选择靠谱的钱和老板很重要,这还是一个职场的故事。你听说上周大连实德刚刚以0:5输掉一场比赛吗?我打赌它还会再输上一阵的。

\begin{center}\rule{3in}{0.4pt}\end{center}

\textbf{《大富时代》}

那时候大富豪们似乎是在金钱和财富玩具中游泳:任性的石油商们为了实现每个费伯式的神话,像收集糖果一样收集飞机、牧场和艺术品。

布赖恩·伯勒

中信出版社 2010年12月

定价:59元

\section{如果回到1997年}

如果把1997年的苹果公司摆在你面前,你对它会有什么样的见解?或者更直白一点说,你会投资它吗?

在那一年里,苹果发生的最大的事就是创始人史蒂夫·乔布斯又回来了。对于投资者来说,这是一个需要认真考虑的大事。

自从这个人被苹果董事会赶出去,在个人电脑业差不多已经一事无成了十年时间,他自己的皮克斯工作室虽然看着不错,但这成功是给文艺青年们的,不是华尔街的,离改变世界还远着呢。

就这么说吧,如果回到1997年,当时关于苹果的基本面是这样的:在内部管理上,人事有重大异动,如果把它视为利好消息,那就是他对产品有极致的完美追求,如果视为利空,就是``那个以生产华而不实的产品而着称''的CEO又回来了;在战略上,苹果还在坚持``旧式的纵向型计算机生产模式'',除了底层芯片之外,从架构开始到操作系统、应用软件、供应链销售分销都由它自己来完成,而这种方式已经被微软和英特尔横向分工的Wintel联盟打败;从产品角度,苹果有包括打印机在内的上百种产品,都不怎么样,有一款叫牛顿的掌上电脑,很有想象力,但做得也很吃力;从竞争对手看,那些Wintel联盟中的硬件制造商正抓住大家涌入互联网的机会让PC摆到每个人的桌子上,最强劲的竞争对手微软Windows95如日中天\ldots{}\ldots{}我对照了一下克莱顿·克里斯坦森的创新理论,他面对这个公司,应该不会给出一个太好的投资建议。学者出身的商业理论书,总是不招人待见,什么东西被冠以``学院派'',基本上要被当成脱离实际的代名词。我揣测对于那些习惯并接受``实践里面出真知''的成功者来说,很有可能还会在心里呵斥一下克里斯坦森:说得简单,你给我创新个看看!但我想说的是,如果你能用好它,比如乔布斯,他是克里斯坦森的信众中最成功最忠实的一个\ldots{}\ldots{}这还真是一件利器。

两年以前,我在看克里斯坦森最重要关系到我们未来收益的新业务,如果你拿当年利润来衡量,那就是鼠目寸光了。

文|伊险峰如果回到1997年的着作《创新者的窘境》的时候,最大的感慨有两个:一、他说得可真好;二、为什么这么点事会说个没完没了,不就是大公司受制于利润、资源布局、战略而很难让破坏式创新成为工作重心吗?

他絮絮叨叨的理论,举个例子来说,就好比可口可乐在中国推了无数种纯净水、矿泉水还有茶饮料之类的非碳酸型饮料,但总是做得虎头蛇尾,为什么卖不好呢?因为碳酸饮料给它们带来的收益太好了,它在这里花1元钱可能会带来5角钱的回报。

但是任何一款新的饮料,虽然它们可能代表了未来的饮料趋势,还意味着未来收益,但可能1元钱只会回来1角钱,可能还赔了。

那么不管是决策者还是财务部门都不会给鼓捣茶叶水的那帮家伙们什么好脸色,于是他们就会一直抱着可乐不放,直到某一天碳酸饮料市场崩溃了\ldots{}\ldots{}克里斯坦森的理论还可以很轻松地解释,为什么微软做不好互联网,雅虎做不好搜索,谷歌做不好社交网络。当然,它教会我的最重要的一件事就是,在此之前,在提起我们的``新媒体业务''时,我总是``呵呵''那么一下,连自己都觉得不怀好意。在此之后,我就克制住了这种冲动。是赔钱啊不假,但这可是关系到我们未来收益的新业务,如果你拿当年利润来衡量,那就是鼠目寸光了。

你看,看书使人进步啊。两年以后,我再看克里斯坦森的另一本《创新者的解答》,我就觉得心平气和了好多,并且,如果你假装自己是一个``价值投资者''的话,那么,他的说法可能是真的有帮助的。

比如最近惠普和诺基亚日子过得不够好,说起来跟1997年的苹果还有点像。惠普的动作是:它们不断地换CEO,说明战略游移,前一任买了Palm公司和它的WebOS系统,后一任隔不到半年就要砍掉这块业务,再后一任觉得这东西也还行啊,还是别卖了;而诺基亚的动作相对来说要坚决准确得多,它们只换了一次CEO,只做了一件事,就是傍上微软,虽然两只火鸡绑一起也不能变成凤凰,但至少它把塞班这样的鸡肋给扔一边了,并且新推的产品有型有款,用了都说好\ldots{}\ldots{}最近它们都出了年报,简单点说都是收入增长乏力,盈利能力下降,市场份额丢失,但华尔街的分析师们对惠普很友好,对诺基亚则恨不得一棒子打死。如果用克里斯坦森的理论来衡量,诺基亚对于自己来说有足够的破坏力,但对于市场来说还处在``延续性创新''这个范畴之中,而惠普基本上是毫无作为,它的市场完全暴露在所有``破坏性创新''的火力范围之内。华尔街的分析师们可从来不把你在做什么这事太放在心上,创新是什么?你财务表现好吗?在对公司未来的把握上,他们从来都是随波逐流,惠普下降也没有关系,只要比我预期的好就是进步。

我多少有点反华尔街倾向,尤其是它们拿着利润说个没完的时候。资本这东西,有的时候过于势利,有的时候又过于主观性。这也会影响我们对公司的判断。比如我和同事总是对中国的几家电子商务公司前景争论不休,我以前的看法是,苏宁这种公司简直是没办法跟京东去竞争的,苏宁花的是差不多都是自己的钱,那可是门店里一分一分赚出来的,京东花的是投资者的钱,谁能烧得起呢?克里斯坦森捎带着也替我们分析了这些资本的区别,他把它们分为风险资金和企业资金,然后他告诉我们,不要去想这些钱是企业自己的钱还是风险投资的,而是要想这些钱的目的是什么。企业投资,也就是那些已经功成名就的公司通常犯的错误并不是烧不起钱,而是相反,它们过于``追求成长,对利润有反常的耐心''。有个朋友是好几家电商的客户,在他眼中,苏宁的电商最爱说的一句话是``钱不是问题,我们有钱''。

京东花风险投资的钱。克里斯坦森又会告诉我们,风险投资经常会在它们竞争的压力下不断扩大自己的投资额度,比如只要投20
0万美元就可以保证一个公司发展,但它出于竞争需要,投下了500万美元,这通常导致的后果是让一个企业从应急性转向谋划性策略,而通常谋划性策略对延续性创新更有利,对破坏性创新没有什么益处。简单点说吧,就是钱投得过多,导致的恶果就是让一个破坏性创新公司变成了一个大公司\ldots{}\ldots{}京东已经拿到了15亿美元了。这还能叫风险投资吗?当所有人的出路都只有华尔街一条路的时候,那么华尔街的分析师们就可以说更多话了。

我们再回到1997年看看苹果的价值。

在克里斯坦森总结的规律当中,破坏性创新不是小公司的专利,通常来说创始人更容易成为大公司破坏性创新的引擎;那款牛顿掌上电脑停产了,在十年以后出了一款叫iPad的产品,改变世界了;对美好产品的无止境龟毛追求成为苹果品质的最重要的一部分;纵向型的策略变本加厉,已经下沉到零售环节,是世界上每平方米产生利润最多销售额的专卖店\ldots{}\ldots{}如果你等上十年,这些都是你的价值投资的回报。

但估计没有人有这份耐心为乔布斯下这么大的工夫。你也不要指望华尔街在1997年就有这样的想象力。因为它所关注的那些财务指标,那些盈利预期之类的,也是你所关心的。你等不来创新,他们也一样,因为总的来说,他们最终体现的是投资者的意志,也就是你自己的需求。

要怪只能怪自己。

\begin{center}\rule{3in}{0.4pt}\end{center}

\textbf{《创新者的窘境》}

克莱顿·克里斯坦森

中信出版社 2010年6月

定价:38元

和好钱一样,

坏钱既可以来自风险投资,

也可以来自企业内部投资。

\textbf{《创新者的解答》}

克莱顿·克里斯坦森

中信出版社 2010年6月

定价:38元

\section{创新也能是护城河?}

在我们错过了1997年的苹果,诺基亚和惠普也承载不起我们的希望之后,我们到哪里去判断和寻找那些能让我们赚钱的创新公司呢?啊,我们还是先从一个悲观的结论开始吧。

你逮不到那个让你后半生老有所依的公司。作为一个悲观主义的创新的爱好者,
想到的最悲剧的事情莫过于创新很难成为一个``概念''或者一个``题材'',你因为你的远见卓识,看出这个公司有可能飞黄腾达,
你暗中投资它,偷偷地等着它涨起来,涨起来\ldots{}\ldots{}你守着你的那个创新公司-寂寞无助,直到你开始算起了机会成本,然后对自己产生怀疑,然后在苹果的市值达到6000亿美元的时候,你恨恨地说:其实,我很早以前就看好它。其实,你早就输了。

当然,一个专注于创新的公司,大概也不会太在意华尔街的人会如何想,如果它有足够的自信,可能还要调戏一下那些投资者,就像谷歌当年做的那样。至少在2011年以前,我想像不出拉里·佩奇和谢尔盖·布林等人会在董事会上说起它们的股价问题,这是一件很可笑的事吧?但这种事还真是一般董事会里最重要的一件事,不管是独立的还是非独立的,执行或者非执行,董事们都会关心怎么让自己的期权更值钱一些。

只要不作假,这看起来也是一件正确的事。我们作为屌丝投资者最容易犯的错误就是总想着管理层那帮尸位素餐的家伙们赚钱太多了,``他们在抢我们的钱哦''。按照托马斯·索维尔的观点,我们有这种错误想法的原因是典型的``零和型思维方式'',
我们总以为高管的薪水是从我们的收益中拿走的,索维尔教导我们,如果CEO赚了一个亿,他拿走5
0 0
0万,有什么不能接受的呢?只要他为股东创造的财富超过他拿走的钱,这事没有什么不好的。

但是,如果这个公司只是为了一时的利益而忽略了市场变化,今朝有酒今朝醉,
满脑子都是落袋为安的想法,那投资者瞪起眼睛来说``你们拿钱拿得未免太多了'',

这质疑得还真是入情入理。这就说到我爱的``创新''这个话题了,一个好的CEO,当然要关注财务表现,因为你做的是当下的事,赚的是过去大笔收益抬起来的期权,而投资者关心的是未来,如果不想着创新带来未来收益,根据克里斯坦森的``高利润守恒定律'',你现在的利润早晚要转移到其他公司的其他环节上,你现在的高收益产品``模块化''和``货品化''之后,你要是没有新的高利润产品-你还真是在抢股东的钱。

不过鉴于那些创新能力特强的公司也不爱分红,投资者想的也都是怎么让股价翻番这样的事,我还真不知道创新的能力到底对于投资有什么作用。我安慰自己的办法就是,哪怕你炒短线,你盯着专业机构的各种指导来``价值投资'',但你没准可以从一个公司的创新能力那里得到:你判断下一个好工作的依据;好的产品或服务;让你很high的创新精神\ldots{}\ldots{}如果它还能带来投资收益,那就完美了。

但你要知道,即使那些明察团队价值的风险投资家们,其实也不是那么一下子就为投资者指出明路的。比如去年首轮融资就得到了4100万美元的Color,既没有一个好产品,貌似也没有一个让他们爱的好团队,
怎么听起来都像是一个骗局,至于那几个久经风霜的投资大佬是如何跟他们的LP们解释的,没有人来做这个事。安迪·格鲁夫说,``要了解公司的实际策略,就要注意公司管理者的行动,而不是他们的言论。''自从乔布斯之后,每个人都在说创新,但你看看,这世界创新了好多东西吗?

安迪肯定要算一个。我忍不住要再一次讲他的故事,他和戈登·摩尔对英特尔的未来犹豫不决的时候,想到如果董事会不能容忍他们了,那么就会派新的管理者跑进来,然后马上会改变战略,进入新的市场-按克里斯坦森的说法,是新的零消费市场-``那么,为什么不能是我们自己来?''我觉得这对于英特尔来说,是标准的创新时刻,对于投资者来说,这是一个值得投资的完美团队。

克里斯坦森对于大企业的创新,提了三个元素,资源、流程和价值观。安迪·格鲁夫,就是资源中最重要的人力资源。而他跟摩尔说的``何不走出门去,我们重新来过'',就是对流程的界定。英特尔没有堕落到跟日本公司一起去做低利润产品、参与低水平竞争,就在于这个被神化了的瞬间所产生的``破坏性创新''的价值。

如果我是一个投资者,看到安迪·格鲁夫把卖微处理器的钱大把地揣进自己的腰包,我觉得我是能接受的。克里斯坦森提醒我们的另一个点是,那些能够实行``破坏式创新''而得以保持持续增长的企业-
他把持续定义成``50年以内''-魄力大多来源于创始人的决策。你不能等一个公司起死回生,股价已经涨到天上去了再去做投资,你一定要事先就了解这些背景。那么去找一下那些还符合这个条件,并且号称正在创新的公司吧。

投资团队,是典型的从结果推导出原因。虽然它相当重要,但像巴菲特就不会太把这事放在心上。我看他不管投哪个公司,
他更关注的都是那个着名的护城河,而不喜欢苦逼兮兮的创新。在他的世界里,如果这个公司不管谁做都保证稳赚钱,那才是好公司,并且所谓护城河就是,即使把1000
亿美元摔你脸上,让你再做一个可口可乐,
你也做不出来。你可以理解为这是一种无趣的投资,但你也可以给它上升到``这就是人类基本需求''这个层面。

克里斯坦森也强调需求,就是我们一直说的``零消费市场'',它或者是因为利润不高大公司不屑于进入,而另一方面也有可能是它是一个消费者还不知道有这样需求的产品。保罗·格雷厄姆-不是我们常说的那个本杰明,保罗是风险投资界的大佬-他认为所谓创新需求,可不是消费者哭着喊着要的那些,而是他们可能根本就还不知道有这种需求的东西,你做出来了,他们发现,对啊,这东西真好。

那些有消费需求的,就是克里斯坦森说的大公司内部最容易产生的``延续式创新'',诺基亚都有了人类学家来策划产品,
这些都是为延续式创新服务的;那些没有需求但硬造出来的,我们说之所以苹果很伟大,iPhone、iTunes和iPad都算得上。我觉得在这一点上,巴菲特也好,克里斯坦森也好,保罗·格雷厄姆也好,这三个迥异的人可能有什么地方是相通的,就是创造一种无中生有的需求。2007年的时候,我第一次看到Twitter,我当时的震惊就是,这是什么人想出来的,我们做媒体的经常苦思冥想各种互联网应用,但这真不是我们想得出来的。

如果我是风险投资家,那我就投它们了,至于杰克·多尔西是否朝三暮四,可能还真不是最重要的事。

\begin{center}\rule{3in}{0.4pt}\end{center}

\textbf{《创新者的窘境》}

克莱顿·克里斯坦森

中信出版社 2010年6月

定价:38元

\textbf{《只有偏执狂才能生存》}

安迪·格鲁夫

中信出版社 2002年8月

定价:22元

\textbf{《创新者的解答》}

克莱顿·克里斯坦森

中信出版社 2010年6月

定价:38元 \# 投资者从垄断公司那里得到什么 \#

关于垄断型公司,有一个经典的误解是,
它会因为它的垄断地位和稳定收益,给投资者带来很好的回报。也就是说,在通常理解里,虽然垄断公司对消费者来说就是一场没完没了的噩梦,但对于投资者来说那是上天赐给你的贴心礼物。有关这个经典理论的阐释及推广者当然是中国移动的前总裁王建宙。他那句着名的话就不再重复了,
翻译过来就是,如果你们还是我的股东,那你们这些聒噪的消费者就该闭嘴了!

这个事也可以换个角度来理解,就是有一个人,他很黑心地赚了你的钱,然后再跟你说,``来,你也入一股,我们一起去赚别人的钱!''天,你觉得这像是好人做的事吗?
很多时候,人心里会有各种斗争,比如说,
这可是投资行为,不要拿道德律令来左右自己,傍上大公司赚上一笔好钱,这又有什么错呢?但你要知道,第一,任何人如果跟你说你有机会稳赚不赔,那么你一定要把他当成一个骗子。第二,如果每个人都认为投资某个公司是会赚钱的,那么最终结果可能是难以形成交易,还是没钱可赚。

\textbf{《沃尔玛效应》}

查尔斯·费什曼

中信出版社 2007年7月

定价:36元

比如沃尔玛,在我看来它是一个好的垄断公司,因为它的出身是市场,可不是那些靠批文啊牌照啊政策啊扶持起来的垄断,还因为它不但把那些试图横向竞争的小角色给一一干掉了,在纵向上也把上下游的产业链给整治得服服帖帖,举最简单的例子来说,中国制造中,出口美国的总量中有10\%是给沃尔玛的;再一个例子来说,有一天沃尔玛宣布含镍的产品是不环保的,
它拒绝含镍的家具了,于是全球大宗生产资料市场就掉头往下走了。就这么一个公司,
号称每年为美国人节省300亿美元,相当于给每个家庭节省270美元,你如果投资它,
那么在2000年到2005年间,你的确是不会赔钱的,但你也一分钱赚不到,如果算上机会成本的话,你基本上就赔了。并且你要知道,那个时候亚马逊也才刚刚开始盈利,真正产生威胁还早着呢。

这看起来还不是一个小概率事件。前一段时间史蒂夫·鲍尔默到中国来,微软最不喜欢媒体问的问题就是股价问题。如果你看它的垄断能力,它现在一样有自己的价值,如果你再只看它们的收入水平,制造利润能力,还有增长率,按理说应该还是华尔街那些唯利是图的家伙们眼中最好的公司。但实际上华尔街也没有人爱它们,在这个时候,那些投资分析师们突然关心起持续创新能力来了,我很怀疑这是华尔街在找借口,关于微软你投资的价值在哪里呢?如果它真是一个稳定的赚钱机器的话,你就跟着它分红好了,哪里会有惊喜呢?这东西就跟人生一样,你要是一马平川地幸福着、幸福着、幸福着\ldots{}\ldots{}那你就幸福着吧,除了坐等巴菲特求真爱,貌似也没有什么太好的办法。对于普通投资者来说,最爱听的还是各种故事,并且最好是不幸的家庭各有各的不幸那种,这样才有投资价值啊亲。

\textbf{《谁说大象不能跳舞》}

中信出版社 2010年1月

定价:38元

这里要说到的就是投资者理论上分为两种,一种叫普通投资者,另一种叫中国普通投资者。普通投资者呢,可能遵循着股权收益率的那个公式,严格算着每年股东权益回报这回事,就是想着投资分红带来的收益。中国投资者呢,基本上没听说过分红可以致富这回事,你在市场上弱弱地问上一句:大佬,分红吗?然后就会听到一大坨吵吵嚷嚷的关于资本市场发展走势的声音,
你马上就羞愧地湮没在各种未来预期的分析中。还好,最近来了个美国人,以一贯很《沃尔玛效应》

强横的语气说,你们得分红啊!如果你以为他们大发慈悲之心,想代言我们这些小股东,那你就又错了,其实他们跟我们一样也是炒股炒成了股东:相对于那些庞大的垄断国企的家业来说,美国这些小股东本来是想借着``新兴市场+市场监管缺失+垄断地位''大赚一笔的,最后也不得不沦为中国的普通投资者,幸好,他们还能想起来分红这回事,否则我们就连这个也都错过了。

当然,一定会有人提出相反的观点,
垄断,不正是巴菲特说的护城河吗?有什么不好呢,一条大河波浪宽,这是保护我们投资利益的最好方式啊。这么说当然也不是没有道理的,西奥多·韦尔就表现出一种垄断大师的胸怀,他说,在垄断的保障之下,
人性中恶的一面将得以抑制,而善良的天性则会得到发抒。没有达尔文主义优胜劣汰的你争我夺-他为资本主义所描绘的未来是温良尽职的人们与政府紧密协作,
以科学管理的企业为公众谋求最大福利的画面。但众所周知,韦尔描绘的这个画卷,
最后变成了AT\&T对新技术的恐惧,它们雪藏的东西包括电话答录机、光纤、移动电话、数字用户环路(DSL)、传真机、扬声电话\ldots{}\ldots{}作为一个通讯运营商,你还能想出它们有什么东西是没打算雪藏起来的吗?当然,你可以说,垄断公司不就靠着这个来保护自己的利润吗?但同样要知道的是,我们常说的那句是``垄断是市场的敌人'',它不但表现在消费市场,也会表现在投资市场上。如果你真心打算投资,长期持有,那么最终结果很可能是个悲剧。

\textbf{《总开关》}

中信出版社 2011年10月

定价:49元

这就要说到垄断公司的两种结局。一是它会变成恐龙,像我们前面提到的不再提供投资价值的那些,都是恐龙特征的一方面,而恐龙的另一个特征就是不光它自己面临着灭绝的风险,它还会耽误别的事。比如IBM有一段成了主机市场的绝对垄断者,
但是后来主机过时了,分布式运算让PC市场成为主流。被逼入个人电脑领域的IBM利用市场余威,联合初生的微软做了一件很流氓的事,把加里·基尔代尔的操作系统赶出市场,于是本来更先进的多任务操作系统就这样被耽误了整整10年。在IT业,10年意味着什么?比尔·盖茨在1995年的时候预测10年之后的事,每件事都很认真,都符合逻辑,
但基本上没有一个靠谱的,这就是10年的概念。垄断公司会利用垄断地位作恶。而如果它已经进化到恐龙阶段,那么作恶的级别会更高。作为投资者你最关心的IBM公司的结局-如历史告诉我们的,在1990年代,它们已经行将崩溃,传统业务已经彻底失败,
而新的业务那是微软和英特尔的天下。

而垄断公司的另一个结局就是分拆, 远的有洛克菲勒的标准石油,近的有AT\&T
对投资者来说,总不是好结局。

最后一个问题是,垄断难免最终会带来思想和行为的禁锢,套用那句着名的话,
绝对的权力导致绝对的腐败,韦尔想到的那种其乐融融的人类画卷不但不会发生,
而且更有可能的是借助垄断的地位产生更多的企图。设想一下,如果每个垄断公司的老板都是王建宙的话,那么它的极致状态可能就是,每个人都是中国移动的股东,你真的会觉得作为消费者的你会把冤大头的损失从投资者身份那里补回来吗?每个人都是股东的事我们又不是没有经历过,那就是全民所有,就是国家至上,就是一大坨人人有份的东西成为``无主资产'',实际上每个China.Inc主义者也都是这么想的,也这么干的。

\section{荒谬历史中的一点奇葩}

新媒体兴起之初,总会有领风气之先者脱颖而出,运用得当,表现不俗。比如广播时代有富兰克林·罗斯福,围炉夜话,让全美上下一条心好像由此就战胜了萧条与纳粹;电视时代有肯尼迪,年轻的世家子弟,又是普利策奖获得者,美国一下子就进入到新纪元了;如今的互联网时代,差一点成就了偶像奥巴马------之所以说是差一点,是我还不知道他能否对付得了这个金融危机以及随之而来的欧债危机,现在但凡有一个法西斯或者铁幕国家横空出世,奥巴马可能就不光是得一个诺贝尔和平奖了。在这里插一句,我能想象得出欧洲人穷志短媚于世,但拿诺贝尔和平奖这事过了这么久还是远超于我的想象。

回过头来说,新的媒体形式一出,总是代表着新锐、革新甚至就是革命本身,捎带着使用者也似乎成了革新者的同谋,你要说狐假虎威也行,虚张声势也行。就说前面两个赫赫有名的已经盖棺定论的美国前总统,到底做得怎么样,还真不好下结论,我们看到的都是被媒体和彼时新媒体放大了无数倍的一个结果,实际情况如果我按常理推断,应该没有我们以为的那么花哨。这么想来,有的时候会让人喟叹:历史原来还真是荒谬的。或者说,人类才是更荒谬的。

当然,他们还不能算是最荒谬的,跟罗斯福同时代的希特勒或者墨索里尼才是集大成者。他们不仅一样是得风气之先的新媒体受益者,而且他们各自带着文艺青年和报社记者的身份粉墨登场,更为人类的荒谬添了两三个星,当然,要说不靠谱及破坏力那就更多了。

放心,我们这一次不是讲帝王术。我想说的是,帝王们借着新媒体横扫千军的势头,传播的力量往往被放大,个人的能力、魅力和战斗力也会被无限放大,就跟我们这些加了V的普通人一样,通常会被粉丝所鼓舞------看起来你在率领群氓出埃及,但实际上可能更是为粉丝们娱乐着消费者裹胁着往前走。一旦碰到什么事,比如金融危机,比如纳粹兴起,比如占领华尔街,这无数的粉丝还会有无尽的促进作用,是不是坏作用,难说。

从几个人生基本定律来看,有一些东西是肯定的。定律一,千穿万穿马屁不穿定律,所以高高在上者总是更容易笑纳粉丝膜拜,即使心生鄙视谓之群氓,但依旧会享用教主乐趣;定律二,控制力总是多多益善定律,有此影响力,宣传、鼓吹、动员、洗脑、指引、摩西、伟大领袖都于其中了;定律三听着要正派一点,选民永远是财富定律,有此三大定律在,自然如鱼得水,自然对这些大佬们是好事。

不过另外两个规律也会导致坏结果,我们可以把第一个规律叫``总是会得到最差结果''理论,就是说当一群人出现意见不统一的时候,聪明人总是更容易向傻瓜妥协,而傻瓜很少向聪明人妥协,所以一群人产生的判断总是无限接近于最弱智的选择;另外一个规律我们姑且称之为``4分钟人''理论,帝王们在得到粉丝的鼓舞之后,总是倾向于和粉丝们一道去挖掘更多的粉丝,并且粉丝通常表现得更有力量。在罗斯福之前,美国宣布参加第一次世界大战之后,美国政府组织了7.5万名演讲者,因为他们的职责主要就是发布4分钟的演讲,所以又被称为4分钟人,这一群人一共发表了75.
5万次演讲------这其中一定有一部分人是真心粉丝,另外那些我们可以称之为僵尸粉,或者称为五X党,虽然对于正直的政客们来说这听起来有点丢人,但如果它有效呢?

我们就拿罗斯福举例子来讲好了。在我们的语言体系里,他救美国人民于大萧条之水深火热当中,并且作为世界反法西斯统一战线的一个中坚力量,领导美国人民取得了二战的胜利------这差不多可以定位成大救星那个级别了。抛开二战不说,如果只说大萧条之后挽救美国经济,我看可说的还真是不多。以田纳西河流域管理局为代表的新政,号称让大家有工作做有碗饭吃------用政府增加支出的方式当然算不上什么错误,但如果我们知道大多数的大萧条悲剧故事实际上并非发生在1929到1933年的胡佛总统期间,而是发生在1936年,也就是新政开始了4年之后,你会作何感想?

大家还多半会以为罗斯福是师从于凯恩斯,这话可能说对了一半,另一半他更相信的可能是斯大林,于是才有白宫智囊们乘着``罗斯福总统号''的东游取经。1930年代全世界的资本主义都奄奄一息,全世界的自由派都对自己失去了信心,眼看着还金光闪闪的只有苏联了,只有它们还在按计划生产,并且创造着奇迹。当然在荒谬的历史中,大家纷纷表示作为一个被蒙蔽者真的没有看到那些个被遮蔽的集体农庄。

你要说这些都算是罗斯福总统的成就,我也不反对。反正结果就是,如果没有珍珠港,还真不知道如何收场。

这一次罗斯福总统做得很果断,利用他的围炉夜话,一夜之间就把所有日本人给弄到集中营里去了。在整个二战期间,没有人对此说一个不字,连正义的李普曼也没觉得这事有什么不妥的。你看这就是全民粉丝的结果。虽然在这几年当中,包括之后,没有一个日本人被指犯有通敌之罪。

在经济领域,通常来说,倒霉鬼的次序是这么排的:排在最前面的一定是犹太人,这个不用多解释了,希特勒能够深得民心,就是靠栽赃犹太人而来。再接下来的是垄断者。最后,赚了钱的所有人都会成为倒霉蛋,比如塞缪尔·英萨尔。

罗斯福为什么要把一个可能是美国历史上最伟大的企业家说成是``违背众人意愿的以实玛利'',这真是一个有趣的问题。我们再把目光放回到那些拥有众多粉丝的大人物身上,去看他们的传播、煽动、鼓动、宣传之恶。在罗斯福的同时代,还有一个库格林神父,这个人也有迷人的声音,也得了风气之先,也掌握了电台,也拥有一众粉丝,他做的所有事就是告诉这个世界,有J·P·摩根在,有那些血液里没有流淌着基督教血脉的毫无价值的人在,就一定有罪恶,世界就将陷于灾难。

如果我没说错,库格林神父应该就是那种有一点知识的人,但这些知识在他那里只是对他有用无用之区别,然后虚设几个前提,然后就着这些似是而非的东西,投愤怒之所好,展开单口相声似的批判。并且好像他是在反权威、反压迫,一门心思在为劳苦大众代言,但实际上是帮着政府在转移目标。再者,我觉得罗斯福和库格林之间的默契,细究起来,也是荒谬人类或者荒谬历史上的一个奇葩了吧。

如果说库格林神父之后谁是他的衣钵传人,那么就我的视力所及,我们这里有一位郎咸平,一定应该算一个。

\section{我们如何来到现在}

一个叫查尔斯·汤斯的科学家,他接到一个研究飞机上使用雷达的任务。在他做这个研究之前,除了原子弹这种闭着眼睛一脚把炸弹踢下飞机也能杀死无数人之外,飞机扔炸弹基本上是靠瞎蒙。研究这种雷达的目的是可以更有效率地杀人,但是当时的雷达波长很长,所以要求雷达直径也很长。

汤斯的任务是怎么搞出一个短波长的雷达。但是他发现一个问题,如果波长短到比如1.
5厘米,那么它很容易被水吸收---这样雨天雾天就没办法准确杀人了。我不知道雷达们后来怎么样了,反正汤斯发现了这个特点。于是,查尔斯·汤斯,这位来自贝尔实验室的科学家,发明了人类最重要的热剩饭的工具:微波炉。

如果单看这个故事,如我们标题所提出来的问题,人类能够走到现在除了庆幸躲过了汤斯扔下的炸弹之外,进步来自于战争和科学技术。

还有一个法国女人叫约瑟夫·玛丽·杰卡德,我觉得她与黄道婆可以并称为世界上最着名的两个纺织女工,她发明了一种织布机,可以用打卡的方式决定先编入经线还是纬线,这样可以产生想要的图案,并且它还是可以通过设计一定的程序来产生有规律的图案。美国人赫尔曼·豪瑞斯勒在织布机的基础上发明了制表机,它们都是二进制的机器。最后这东西落在了托马斯·沃森的手里,他搞出来了一个IBM。

如果从这个故事看``我们是怎么过来的'',那么可能要提到的是商业、需求、企业家,市场和搞定市场的人才。

关于如何来到现在,类似的说法我们可以列举无数个,但同样每个都可以否定掉:战争当然是一个重要的催生技术发展的关键,但得益于战争而发明出来的``图灵计算机''在英国有一段时间里都可以称为民用机了,到了冷战时期却被统统收缴了事。

政府不让大家用了:这么高科技的东西让民间来玩,如果被俄国人窃了密可怎么办?

沃森也不全知全能,虽然他看到了商业在社会中的中心地位,并把公司愿景定为``处理所有商业信息'',但他预测这世界上有五台计算机可能就够了,于是他把DOS系统让给了比尔·盖茨,盖茨给微软定的愿景才是``让每个人的桌子上都摆上一台电脑''。一个卖软件的有这样的雄才大略,基本上可以担当如何来到现在的推进者了,但最后在更强大的互联网面前,他还是一个卖软件的家伙。

``我们如何来到现在''这个问题的提问者安迪·凯斯勒有自己的一套答案,他是一个喜欢做减法的家伙。他在同名的书里说到了伊丽莎白一世,她的减法基本上奠定了现代社会:国家和政府是管不好投资这类事的,不如把它交给银行吧。

由此,现代社会开始了。但银行靠谱吗?全世界所有银行都爱大公司,它们最爱锦上添花,却很少雪中送炭,为什么?因为它们最大的敌人是``欠钱不还'',所以银行家们最讨厌的是风险。

当然,你也可以说国外有那么多小银行,它们可是做得不错的!没错,但你可以看加雷·加勒特怎么说。

他是从金匠开始说起的。金匠在替人做首饰的同时,也会替人保管,而保管的凭条就成了一种可交易的凭据,时间久了之后金匠发现金主们同时来提走金子的可能性并不大,这样他可以多发一点凭条并没有太大风险---银行就是这么来的。

加勒特让我们想象某个镇上的某一个小银行,它服务的对象包括面包师、铁匠、养马的、种地的等等,他们从银行里借了钱买原料,等到赚了钱再还给银行。但问题是银行要把钱贷出去的,面包、师铁匠们需要借的钱并不能满足一个银行需要的贷款比例,它有大量的信贷放不出去,怎么办?在美国,通常他们会把钱借给认识更多客户的大银行,终于要说到纽约了---那些面包师、铁匠们的钱最后就跑到了纽约的大银行手里。

接下来的事情就是我们知道的另外一些故事,大银行们从小银行那里借来钱之后,再借给更多需要钱的人。它们不愿意把钱借给没有信用的人,那就借给AAA吧。后来的事你知道的,它们买了一堆债券回来。

小银行呢?小银行的资产不见了,面包师的钱当然也不见了。

如果从另外一个角度来看``我们是怎么来到这个操蛋的世界的'',我看这些故事很能说明问题。但安迪·凯斯勒显然是个乐观主义者,他在否定了银行之后,讲了瓦特的故事,瓦特是个Geek,琢磨提升蒸汽机的效率,有一个叫道尔顿的家伙把钱投给他,让他获得了成功。

这时候要说一个艰难的结论:钱,可能是一个相当重要的东西。让没钱的人把自己想出来的东西做好,钱太重要了。阿基米德说,给我一个杠杆我要把地球撬起来。找银行家是不会借给他这个杠杆的,风险太大了。但总会有人愿意试一下,承受风险能力强的那个银行家就有可能做到这一点。于是,VC诞生了。

你知道,如果是你愿意借钱给别人,比``可能借钱不还''还要重要一点的事是什么?一定是``我能得到什么好处''(雷锋们请转过身去)。因此这些愿意拿风险来换取回报的人,他们爱上了IPO。这是一个好东西,从此他们就可以不用放高利贷也可以有更高收益了。证券市场出现的时间其实并不算太晚,18世纪英国就已经有南海泡沫了(这也是英国人受了刺激从此不爱证券的原因)。但它始终没有成为主流,摩根他们家的人只要提到股票一律斥之为美国人的粗俗玩意,绅士们只搞债券。但是在技术成规模之后,IPO的价值更突出了。证券市场可以让没有钱,但是有技术的瓦特们变得有钱起来,变得有信用起来。

在历史上,商业社会之前最好的致富手段一是靠抢劫,二是靠继承。但是商业社会之后,这世界才可以让普通人有钱的。如果你意识到这一点,不但可以理解为什么你痛恨社会不公,而且还会明白如果你均了别人的财富一样也是史前行为。最重要的是你会了解为什么我们会到了现在---文明社会该有的东西我们不都是假模假式的都有了吗?为什么看着有很多地方不对呢?

当然,你也会明白为什么安迪·凯斯勒会把这本书的副标题叫作``商业、技术与金融趣史''。他本人就是个贝尔实验室的程序员,后来觉得上班做个张江男来钱太慢,改行去华尔街做对冲基金,在2000年泡沫的时候,他把他的10
0
0万的基金做到了10个亿,赚了钱之后他决定做点有益人类的事,就去写畅销书去了。

成功的人会用他的成功来解释世界。

加雷·加勒特的人生是成功的,只是他生活在一个不怎么样的年代,他在1930年代的时候写了《压垮世界的泡沫》,所以他会那么悲观地看待银行。

相比之下,我更希望乐观一点的人生。

有一天,某位同学跑过来跟我说:侬知道伐?太阳黑子可以穿越我们的身体!黑子,它就,从我们身体里穿过去了!我在听到那些匪夷所思的东西的时候一向都是很淡定的:这有什么奇怪的,这不是很正常吗?但是我心里说的是,啊,居然可以这样!但如果想上那么十分钟,就会想起来,怀孕MM的防辐射服应该是担心什么辐射会透过皮肤伤到胎儿,能穿越一层皮肤自然就会穿越人体。哦,我们拍的X光就是这样的,我还听说过人体吸收紫外线这样的话,黑子也好、紫外线也好、X光也好,都是某种光波,听说过波粒二相性吗?那些粒子---它就那么穿越过去了。

这么想想,真的是挺好一件事。``居然可以是这样!''即使我们不掌握那么多的钱,我们也没有什么可以赚钱的技术,但我们可以有很多的好奇心,这个东西同样可以让我们走到今天。

\begin{center}\rule{3in}{0.4pt}\end{center}

\textbf{《沃森父子与IBM王朝》}

理查德·泰德洛

中国人民大学出版社 2005年10月

定价:29元

\textbf{《压垮世界的泡沫》}

加雷·加勒特

法律出版社 2010年12月

定价:32元

\textbf{《我们如何来到现在》}

安迪·凯斯勒

机械工业出版社 2011年1月

定价:36元

\section{从长远看,我们都是要死的}

安迪·凯斯勒在香港机场遇到几个衣着光鲜的家伙,用他的话来说,一看就是华尔街的货色。那个时候,他已经不再是摩根士丹利的分析师了,不用再当那些拿着巨额佣金的业务员们的``肉''了---在他那本《华尔街的肉》中,他总结他的前半生命运,所谓分析师,不过是替业务员揽活的帮手,他做的所有事不过是在替销售部门背书。那个时候安迪·凯斯勒是正准备收手的对冲基金管理者,赚了很多钱,他上去招惹了一番这几个华尔街的光鲜小弟,然后道出他心目中的人生浩叹:

在华尔街,人人都会碰上``这到底是怎么回事''或``我到底在干吗''的困惑时刻,有时候,困惑更深时的念头是,这一切到底代表什么呢?

就某方面来说,它代表一切。华尔街是资本主义强盗上路行劫的用武之地,如果没有华尔街,资本主义之轮和经济全都动不了,我们全都会变成奴隶、愤青,或该死的烙牛肉饼,或以上皆是。

就另一方面来说,不论你做的是哪一种工作,我可以向你保证,它什么意义也没有!你只不过是个小齿轮、小卒、小兵。

我是华尔街里的一块肉,华尔街里的每个人都是一块肉,他们从我们身上割掉的脂肪被拿来当成保持市场润滑、有效率的机油。每个人的位置都有利可图,但是,每个人都可以被取代,一想到这,什么``个人福祉''啦、``个人价值''啦,全都不过是空话。

但那又怎样?在华尔街,人人都是犬儒主义者。嘿,你的待遇优厚得很。有什么问题吗?那就像其他人一样,去找个心理治疗师吧!

犬儒这个词,或者说犬儒主义者,现在是个大帽子。当你左看右看,发现这个世界有那么多不如意的事情的时候,你就尽情地把这个事用五个字概括---``犬儒主义啊'',如果这事关系到某个具体的人,那就用六个字,``犬儒主义者啊''。当然,一旦这五六个字跟``这个世界''联动的时候,发出评价的你也基本上要划归到``犬儒主义者''这个行列中来。比如,当被问到:这事儿为什么做不好?你可以找无数个理由,然后,直到你被问烦了,你发出震聋发聩的反问:这事儿做好有什么价值呢?这事曾经在你身上发生过吗?---那么,你是一个犬儒主义者。证明完毕。

有的时候,我甚至觉得这个词就是平民百姓的代名词。这么说当然不是有与平民百姓对立的优越感,而是平民百姓很习惯``有什么法子呢''这样的感叹了,还不如攒点钱像有钱人那样去找个心理治疗师。

实际情况是,犬儒主义当然不是``哀其不幸怒其不争''的平民百姓的专利。拉吉夫·甘地在主政印度的时候,有一次判了一个主动要求离婚的穆斯林女人可以得到赡养费---这就激怒了毛拉们,他就把这笔钱又收回了。反对党人民党得出结论说,这家伙所做的一切无非就是为了选票罢了,犬儒主义者!你看,就是帝王将相,首鼠两端的时候,也跟我们市井之辈一样,祭出``这东西算什么''这一法宝。

虽说犬儒主义几乎已经成了老少咸宜朝野上下都可以贴上的一个标签,但我试着跟很多人探讨过它的定义。在不看百度百科的情况下,基本上没有什么人能把它说得明明白白,虽然例子大家都会举得很贴切,甚至还有人拿出了第欧根尼和他着名的``别挡住我的阳光''来比喻,但结果又会把自己绕进去---古希腊的哲学家们如果听说他们的教派已经成了没原则的代名词,估计会拒绝把自己带入21世纪吧?

我之所以提到百度百科,关键在于这是我看到的为数不多的说得热闹又准确的一个条目。有个例子比较高级---大意是说:一个叫德勒斯的犬儒派收了一笔钱,他觉得这事如果拒绝就把钱看得太重了,所以就收下吧。他的逻辑是金钱虽然重要,但在生活当中有比这个更重要的东西,所以在坚持更高原则的情况下看轻金钱是高尚的---虽然这事透着矫情做作,但至少还有一个原则。这事发展到后来就变成了,既然无所谓高尚,也就无所谓下贱。既然没有什么东西是了不得的,也就没有什么东西是要不得的。你要知道,这基本上就成了上至拉吉夫·甘地,下至平民百姓的人生法则,我忍住了没说这世界就是这么败坏掉的,但实际可能确实如此。

在百度百科中还有若干条箴言,其中有一条说的是极端主义者看上去是犬儒主义的死对头,但实际上每个极端主义者内心深处都藏着深刻的犬儒主义。莱茵霍尔德·尼布尔是个神父,他曾经写过一本着名的书《光明之子与黑暗之子》,这本书写在1944年,那个时候纳粹快不行了,但所有人都在反思为什么这种怪胎会出现在信奉民主自由的西方世界,尼布尔祖父认为是犬儒主义的结果,他的逻辑是:独裁或者极权是民主自由的对立物,民主自由除了是政治制度,它还是一种公众的选择,公众的态度。但公众显然不会直接选举独裁和极权的,一定是有什么样的原因导致了极权和独裁的产生。尼布尔的结论就是,犬儒主义是民主和自由的根本对立物,是现代社会的根本对立物---是你们犬儒的做法最终选择出来的国家社会主义党\ldots{}\ldots{}

这世界就是这么奇怪,最摩棱两可的一种人生观,在什么都无所谓、什么都算个毛的过程当中,就成就了希特勒的第三帝国。你可以说前世不忘后世之师,我们不会踏进同一条河流的。但尼布尔就觉得不是简单这样---我得交代一下背景,这是一本写普世价值的书---他相信悲观以及人类的局限,他叫着嚷着``光明之子'',但他还是相信黑暗之子会笼罩大地。比如他说在现代工业共同体中形成的阶级结构,是民主国家能够持续健康的一个相当重要的资源,健全的现代民主体制,一定与运用由穷人手中掌握的选举权演变而来的政治权力相关,这个权力限制有产阶级的经济权力。犬儒主义者盛行的后果就是最终造成了各种力量进入到混乱状态,通常情况下,没有一个群体会有足够的力量朝着积极的方向迈进,并且所有群体都有足够力量阻止各级的行动。

乐观的人可能不这么看。凯恩斯主义还没有成为放之四海而皆准的真理的时候,乐观的经济学家会说,从长远看,经济是可以通过自我调节恢复正常的。凯恩斯那时还活着,他为此说了一句着名的话:从长远看,人都是要死的。

凯恩斯并不是一个犬儒主义者,作为最早炒汇的那一批人,他的投资回报率还很高。但他的这句话差不多成了犬儒主义者的座右铭,你看,为所欲为的洪水猛兽就这样来了,真的连心理治疗师都用不着了。

\section{一个理想主义者的价值投资}

我第一次看到``人类思维数据库''这个说法的时候,还是惊了一下的。在此之前,在我眼里,谷歌就是谷歌,是一个搜索工具,在此之后,我每次打开那个搜索框,心里就会默念着:它知道我在想什么,它知道所有人都在想什么。

这其中的逻辑很简单,你在网上搜索的东西,是你想要了解的东西,只要你一搜索,你的各种卑微的、阳光的、猥琐的、有趣的、热烈的想法就已经作为呈堂证供摆在那里了。当然,有一派人认为如果你觉得这有侵犯你的隐私之嫌,你可以选择不去搜索---但问题在于,自从有了这个方便的搜索,我当然不会放弃这种方便的应用;而另一方面,作为一个有独立思考能力的人还应该追问一下,我为什么不能搜索啊?

所以在最初的最初,你很容易被``不作恶''这种说法迷住。再过了一段时间之后,你会反思起来:诶,它凭什么可以不作恶?这种气吞山河的话是谁都可以说得出口的吗?它不过是一个公司而已,谁赋予了它这种气概?接下来你对这家公司的态度就会向对立的两极转变,一种是崇敬有加,这真是一个伟大的公司啊,它有作恶的能力而不作恶!另一种则是,它有作恶的能力\ldots{}\ldots{}谁知道它什么时候就作起恶来?!

我们知道,在这个世界上有一种叫价值投资者的人,在这些价值投资者当中还有一个小的流派,我们可以叫他们为``理想主义的价值投资者'',他们试图触及价值的真谛,问题就出现了---比如归真堂,从投资价值角度去分析,它实现了熊胆汁的可持续性开发,熊胆汁这玩意又是东亚人认为可以让自己生龙活虎起来的利器,需求可是没问题的,养个熊能花几个钱啊\ldots{}\ldots{}但是,你作为一个理想主义者,接下来应该想到,活取熊胆这生意给人带来的健康收益,真的确有其事吗?再接下来,对熊来说痛吗?我们就那么贪婪?所以一个理想主义的价值投资者,即使看到这里有无限的投资收益,也会收手不做。同理,紫金矿业,不能做,它污染的时候投资是不应该的;它被罚一笔小额巨款,利空出尽也不应该投;它宣称整改成功,貌似还是不可以投,你可以参照一下另外一些理想主义者对耐克的态度来判断:你每一笔投资,都会让更多的东南亚儿童在黑暗的空气污浊的血汗工厂里被盘剥;这事再往下面深入去想---通常来说,我们总是相信贫困导致童工现象的出现,但实际上童工现象可能也是贫困的原因。

做一个有理想的价值投资者难免要像上面那样陷于纠结。如果好奇心强,还可能会一头扎入到社会学经济学的意义探讨之中。醒醒吧,兄弟,现在都高频投资了,你想什么呢?

有关理想主义的价值投资的极致就是挪威的主权财富基金,有一年它们决定清除其投资组合中所有使用童工的企业,它卖掉了4亿美元的沃尔玛公司的股票,这一举动引发了一场外交冲突,美国大使指责挪威这样做``实质上是对(沃尔玛公司)伦理道德的国家审判''。又过了几年,同样来自挪威的退休基金同许多对冲基金一样带有预见性地卖空冰岛银行的股票,于是更巨大一场外交风波暴发了。我就是作为一个旁观者都会想,你不是道德领先的公司吗?你不代表着一个国家体面的价值观吗?原来你也靠对冲另一个国家的风险来捞钱啊?

当然了,理想主义者总是赚不到钱,也并非全怪到对冲基金头上。因为对他们来说,这世界上根本就没有可投资的公司。比如奥马哈的圣人沃伦·巴菲特,虽然他勤俭持家,虽然他把钱都投给了有志于人类健康事业的梅林达及比尔·盖茨基金会,但他最爱的可口可乐据称是全球肥胖症的元凶之一,虽然他自称要做长期投资,并且投资给那些有价值的伟大的公司,但在遇到纳贝斯克这样转手就可以捞上一大票的LBO机会的时候,他也绝不手软,好在他是个诚实的人,不会去说这仅仅是看中了``扭一扭舔一舔泡一泡''的奥利奥饼干。

如果回到我们开始的关于隐私的担忧。谷歌呢,巴菲特还没打算投,但如我们前面所说,它基本上是要把隐私一网打尽的。当然,谷歌会说,我们不会去关注个体的人做了什么,并且如果当你搜索糖尿病的时候,有一些治疗糖尿病的广告跑出来也未必是坏事。关于这事,一方面理想主义可能会稍微有那么一点不爽---噢,原来我只是你搜集数据的一个工具而已;甚至还有可能出现伯纳斯-李说的那种情况,他说``我想知道,如果我就某种不属于保险公司保险范围的癌症翻阅了一大堆书,然后将会发现我的保险费上涨了5\%,因为他们预料到我在看那些书。''

伯纳斯-李是万维网的发明者,拒绝为他的发明申请专利和变现,是一个品德高尚的人,他对于谷歌的批判和对信息数据的滥用说到了点子上。并且智慧的伯纳斯-李说得的确没错,有关谷歌之后的Facebook的一个故事说的是:一个人在社交网站里不断记录他喜得贵子的愉悦,他交的朋友还看起来各个体面,生活也应该还幸福,但是他始终没有任何照片发出来,那么万能的Facebook啊,就应该推给他一个数码相机或者智能手机的广告。我们要指责它对隐私的侵犯吗?这些大公司哪里有一个省心的。

当然,对于大多数人来说,隐私权的扞卫基本上可以放到叶公好龙这个模板里。前面说的运通、维萨卡之类,有线电视网之类,谷歌Facebook之类,它们实在是让我们的生活变得好了很多,方便了很多。也因为这个道理,你也接受了谷歌地图,也默许了每天飞过几次的卫星拍下了你的屋顶(拍不拍我们真管不着,但谷歌居然用它做起了生意),我们也容忍了谷歌的街景车顺道记下了你的Wi-Fi密码,我们还知道苹果会记录你的Wi-Fi足迹,然后可能还分析你,但你还是放心地把本地都不敢存着的隐私照片放在iCloud上。因为它们确实太方便了。

这么说吧,最后你会发现那些对你虎视眈眈的大公司,虽然在一定程度上蚕食着你的自由,但它反倒可能会给你带来财富,有巴菲特可以作证。这也验证了对冲风险的某种理论:你应该买你供职公司竞争对手的股票,反正你的公司赢了你可以涨工资,对手赢了,你还可以多分一点红。所以,如果你不是一个理想主义者,那么你的生活就是这样的:你用你的隐私权投资了你的美好生活,文艺一点讲,你把你的灵魂典当给了这些大公司,我们可以把它们一律视为墨菲斯特公司,投资一点说,你跟墨菲斯特投资公司搞了一个对赌协议,并且可以想见的是,你最后总是会沉迷在这个美好生活里。

\begin{center}\rule{3in}{0.4pt}\end{center}

\textbf{《搜》} 约翰·巴特尔 中信出版社 2010年3月 定价:29元

\textbf{《巴菲特传》} 罗杰·洛文斯坦 中信出版社 2008年9月 定价:49元

\textbf{《梦断硅谷》} 乔希·勒纳 中信出版社 2012年5月 定价:38元

\section{大公司的边界}

有这么一个公司,它控制了德国100\%的合成橡胶、甲醇、润滑油,90\%的塑料,80\%的炸药的生产,它生产一种叫阿司匹林的药片让人类不大会因为一场感冒就轻易死掉,是全世界的救命药。它创办于1925年,是几家最优秀的德国公司强强联合的产物,在1930年代是德国最大的公司,也是全球最大的化学工业集团。这个公司的投资价值怎么样?

好吧,你决定投资了,但是\ldots{}\ldots{}这个公司赞助了奥斯维辛集中营的修建,还在里面搞起了科学试验,就是各种人类活体实验,以考察药物功效或者人种差异。因为它的这些行为,这个化学工业康采恩在第二次世界大战后被盟国解散了,你还觉得它的投资价值很可观吗?

这个叫法本的公司,是由阿克发公司、巴斯夫公司和拜耳公司联合发起成立的。在奥斯维辛做实验的就包括拜耳的医学家,并且这些医学家中的一个在作为战犯出狱之后回去做了总裁\ldots{}\ldots{}你是不是要重新考虑一下它的投资价值?

这个故事可以最直观地告诉我们,企业应该做什么。但每个人对此的解读却又多种多样。最核心的可能有两种,第一种是企业就应该做企业该做的事,不要有什么太多的想法;第二种是企业要做正义之事,不能胡来。

但如果放在19 3
0年代的那个背景之下,德国作为欧洲大战的战败国,经历了一次债务危机和通货膨胀,正委屈的时候,有这么一家表现卓越、提振信心的公司,民族主义的气概当仁不让,差不多承载着大国重新崛起的信念,它的投资者或者德国人会觉得这个公司在行非正义之事吗?某种意义上,还要为这个投资加分才对。

企业这种东西,跟人一样,当发展顺利势如破竹的时候,难免会多想一点,如理查德·布兰森时刻不忘教诲世人,他最热衷的社会责任这东西,近来又进阶到``社会企业'',动辄就会看到那些号称不以盈利为目的,或者说想着给世界做贡献而不求回报的企业---看到它们我总是在想,谁给你投资?为你投资的目的就是你替大家做善事吗?你免税吗?否则这交税的钱难道不应该从赚的钱里出吗?这是一个解决不了的难题。其实从投资者角度去判断可能会简化一点,比如一个公司做了很多善事,你是它的股东,你对它的印象可能是:这公司靠谱啊!这公司资本雄厚啊!这个公司有责任心,想必一定是个好公司!我很少见到抱着慈善目的来做人家股东的:我钱太多了,投资一下这个公司吧,它会替我献爱心\ldots{}\ldots{}

爱心这东西,总是自己捧出去才显现出价值,即便巴菲特要借道梅林达和比尔·盖茨基金会,那也是大张旗鼓地公之于世的。公司么,如果说它本质的社会责任,其实比布兰森说的要宏大得多,比如这个差不多与工业革命同时发达起来的制度,其实承担了工业革命重新组织社会的职能。工业革命让更多的人进入工厂成为工人,工厂的业务越来越多,继而又让做计划的、财务的、市场的、销售的、采购的这些人成了专业人士,然后才有新的洗脚上田的家伙们成为城里人的一部分---它们对社会的重组,比起搞个千里送爱心这类的事来说,可要隆重得多。我一直对``单位''这个词既爱又恨,在我们这里单位意味着一种劳动合同,意味着一种人才所有制,有的时候还意味着一种特殊的身份,这都是让它看起来很讨厌的原因,但另一方面,我觉得没有什么能比它更好地传达这社会的组成方式了。

公司既然承担了这么伟大的功能,你让它再增加新功能,未必意味着是好事。比如,它要是各个都像法本公司那样热血沸腾,那还真是要警惕起来。

比如美国邮政,它觉得淫秽物品不应该在它的投递范围之内---这个本来是能够理解的,但谁来界定什么是淫秽物品呢?应该是这家企业吗?美国邮政很不客气地把这个活大包大揽下来,于是这就成了经典案例,在一个据说有``宪法第一修正案''保护自由表达权的国家,界定淫秽物品也属于言论自由的一部分,企业越界了。与它类似的还有苹果公司正在决定我们可以看什么杂志,或者玩什么游戏---它的审查部门,一定是某个具体的人,Ta对宗教、淫秽等的态度在决定我们能看到什么。你可以信任苹果公司的产品,但你会信任那个加州土生土长的美国青年对一个中文杂志的判断吗?

在吴修铭看来,一个生产蔗糖的公司不管它有多垄断,它也就是生产糖的公司,对这个社会没有什么影响,但如果这个公司是一个平台,那事情就完全不一样了。吴修铭把服务行业分为三类,一类是像贝尔公司那样,承诺成为那种必须为公众担负起责任的公共承运行业,一类是富兰克林一手创立的邮政局那样由政府运作的国营行业,还有一类是可以放手交给自由市场的类似于制糖业的绝大多数的``一般服务行业''。谷歌和苹果这样的伟大公司都有志于成为第一类,而第二类我们见得太多了,就先放在一边好了,第三类虽然吴修铭不担心,但如果它像法本一样有理想,那还是一个危险的事。

所以像谷歌,它发明了一套算法,在这套算法里实际上暗藏着一套规则,这个规则决定你得到什么样的信息,有一天它说要优化搜索以对抗那些坏公司做的手脚,你就要警惕一下了,这个标准是如何制定以及标准的授权合法性来自哪里;如果你再深想一步,我们总是会搜到``根据当地法律\ldots{}\ldots{}'',这个同样决定了我们能看到什么,虽然决定权不在谷歌,但它与根据公序良俗建立的避讳在本质上又有什么不同呢?

好吧,我们可以再举一个例子,在美国电影史上有一个着名的``海斯法典'',它规定了若干种场面不能出现在电影里,它来自于一个号称很正义的叫威尔·海斯的政客,也来自于一个叫丹尼尔·劳德的天主教教士,这两个人制定的极端保守的原则最后成为了好莱坞的拍片准则,所有的大制片厂们的趣味就这样控制在这两个人手中,然后间接地去影响整个社会。要提一下的是,威尔·海斯还出任过美国邮政总局的局长。虽然这些道德洁癖者通常还都是正直的人,但这个海斯可要除外,因为他据说也是哈定总统的重要竞选资助人。

这就说到这一大串的家伙,他们各个都是理想主义者,关于理想主义者我们总应该提防的一点就是他们不要与强大的权力联姻,当强大的一方是类似于军方的时候,我们通常会保持足够的警惕,但如果这是一些大公司的时候,我们就莫名地赞扬起它们更多的责任心---这是一个我还没想太明白的事。

\begin{center}\rule{3in}{0.4pt}\end{center}

\textbf{《总开关---信息帝国的兴衰变迁》} 吴修铭 中信出版社 2011年8月
定价:49元

\textbf{《阿司匹林传奇》} 杰弗里斯 生活·读书·新知三联书店 2010年7月
定价:38元

\section{所以,它就安静地站着}

一个人会问一头猪:``为什么你不对我说你的快乐而只是站在那里盯着我呢?''这头猪会很高兴来回答,它可能会说:``这个原因就是我总是忘掉了我准备要说的。''但是就在这一刻,它也忘掉了这个回答,所以,它就安静地站着。这段话是弗里德里希·尼采说的。关于这个故事的中国智慧版本是惠子和庄子共同完成的,惠子说,``子非鱼,安知鱼之乐'',庄子说的是,``子非我,安知我不知鱼之乐''。

如果给上面两句话找一个共同点就是,哲学家们认为,人和动物之间实在没有什么可交流的,我们不知道它们在想什么。我还能隐约感觉到他们代表自己或人类表达了谦虚之情,古生物学家斯蒂芬·杰伊·古尔德差不多也从科学的角度浪漫地说:``一条聪明的章鱼很可能认为,八条手臂要比两只手臂更加完美。''但他们难以回避的一个共同的问题就是,谦虚虽然是美德,但一旦表现出我是解释权拥有者,那基本上开始营造的亲切友好谦虚谨慎的氛围就再也不见了。林恩·马古利斯也说,动物是历史进化的叙述者,没准这个叙述者就主动去偏袒了自己的种类。我想说的是,解释权很重要,人类的优越感大部分来自于他们解释世界的能力和解释世界的机会。

在这个栏目里写了一年书评,它多半与投资有关,有的时候还会涉及公司到底是怎么回事,虽然鼓励大家赚钱或者教大家赚钱的思维方法不是一件特别扬起理想主义风帆的事,但我也暗自严格要求自己,这也是一个要传递正确的人生观、世界观和价值观的事啊,可得要严肃起来。按照达尔文的理论,你做什么事,十有八九就会对你的进化路径产生影响(当然,也可能是退化)。我的感觉可能就是,做书评这种事,就是挟天子以令诸侯,我得反思一下自己是不是有那么一点解释权尽在曹丞相了。

虽然笛卡尔说,人啊,你就是一根会思考的芦苇。但如果我是一根芦苇,可能还真会傲娇一下:你们人会思考又怎样,看看你们想的都是啥?卡尔·齐默是个博物学家,他发现黑猩猩在太阳底下互相理毛这种事并不是因为痒,而是它们在交流。人类后来出现了,并且进化成文明的样子,但是在他看来,人类大部分时候的语言跟理毛没有什么区别。他说有一名科学家在火车和餐厅里偷听别人说话,发现一般人谈话内容的三分之二都是和别人有关的事。

好吧,我还是回到我的反思现场来。在有关财富和公司的解释权这个问题上,在说了一些``三观''和方法论的东西之后,我觉得可以换个角度去思考一些问题。比如我总是相信勤奋这事很重要,但你可能要反过来想一下,可能正是懒惰才促成了进步;还有我一直在讲商业带来的民主化是一件很让人兴奋的事,但如果把它翻译成另外一种语言,可能就是:你有,我也得有,你要,我更要---是贪婪在主导着人类的进步;此外就是赚了钱会更容易让一个人产生伟岸的感觉,所以人觉得自己渺小是件很有价值的事。

嗯,不管是因为钱,还是因为别的什么事,人很容易会夸大自己。比如我认识一些环保主义者,他们一方面想象自己的能力强大和优越感(也是从对一些事的解释权那里换来的),另一方面还无限夸大人类对环境的破坏能力。林恩·马古利斯就不这么看,他说,``人类的工业生产使大气中与臭氧层相对立的含氯氟烃的含量增长大约100倍,使之达到约十亿分之一个百分点。这种变化与藻青菌引起的地球环境变化相比,根本算不了什么。藻青菌的生长使得大气中氧气含量从不足一千亿分之一增长到五分之一''。所以要说起造物主,我觉得藻青菌才算得上。

当然了,如果只是让一个臭鸡蛋味儿的地球从此氧气充盈,那还不能完全叫做造物主,藻青菌后来爬到了植物细胞内,就成了叶绿素,一如既往地贡献氧气。这里要说句题外话,人类本能地认为植物不如动物高级,但实际上也不能过分高估自己这个门或界。虽然动物会跑会跳,有些自称高等的还会思考会投资能说会道,无非就是更多的细胞参与工作---就像在硅片上印了更多的电路一样,硅片本身还是那么回事,并且以密集程度或者量大取胜也不怎么高级。我就不觉得果蝇的复眼就比人类更能发现世界的奥秘,一株植物可能看到动物趋利避害地背井离乡,或许还会觉得这些玩意太不够淡定了。在生物学家眼中,如果从细胞的复杂和精密程度上看,动物还真不如植物,原因也在于藻青菌的后代叶绿素在继续发挥着造物的作用。

不过,人类或者其他动物界也不用妄自菲薄,这东西在人类细胞中也有。卡尔·齐默为此还特意提醒我们,``在人体细胞中和在人体中是两个完全不同的概念''。估计你能想象得到的是你身体里有无数微生物,你每天拖着一个一百来斤的小生态环境走来走去,但你可能很难想象得出来,你洁身自好到一尘不染,哪怕活在真空里,你也跟史前的一坨坨藻青菌活在一起---这么说可能不够严谨,但你就理解成,你不是你,你是由这些玩意组成的。

从这里我们可以得出四个结论:第一是林黛玉这样的自由主义洁癖症患者还是一头撞死算了;第二个是我觉得那些环保主义者如果能发挥更多藻青菌的功能会更好,但还要注意不要在人体内发作,因为有研究发现,线粒体这东西在人体内一旦有所作为起来,通常会导致伤寒病;第三个结论是,你可以觉得人类或万物很神奇,但你要说多高端多伟大,那可谈不上;最后一个关键的结论是,这些东西很容易导致你的``三观''碎了一地,但我想说,碎几次也不是坏事。

回到林恩·马古利斯的那个关于进化论是动物界的人想出来的、优越感来自于解释权这个命题,我们还可以说一个经常被误解的事。我们通常认为哺乳动物更高级,但从卡尔·齐默的研究来看,哺乳动物和恐龙一样古老,早期有一种长得有点像狗的单弓类动物,有胸廓保护横膈膜,也进化出毛发,体力也不错,但在刚开始的那一亿来年,``它们微不足道,可以说是活在恐龙的阴影里,而且多半是小型动物,可能还是夜行动物,看起来没啥希望的样子。''听起来很悲催,不过如我们所知,那些大只的恐龙后来的运气不是那么好,遭遇了陨石天谴,所以最后它们就灭绝掉了。活下来混得好的现在在天上飞,偶尔被人类关在笼子里让我们看羽毛,混得不好的就进了肯德基。它告诉我们的道理是,进化不是一个线性发展的结果,你出来得晚不等于你就是最出类拔萃的,你能坚持到最后还硕果仅存,也未必就体面到哪里去。

\begin{center}\rule{3in}{0.4pt}\end{center}

\textbf{《我是谁:闻所未闻的生命故事》} 林恩·马古利斯/多里昂·萨根
江西教育出版社 2001年8月 定价:39.8元

\textbf{《演化,跨越40亿年的生命记录》} 卡尔·齐默 上海世纪出版集团
2011年8月 定价:88元

\textbf{《植物的欲望》} 迈克尔·波伦 上海人民出版社,2005年5月 定价:28元

\section{富有的邻居和聪明的小贼们}

苹果打赢了跟三星的官司,按照通常的理解,这对于苹果来说算是一件好事。你可以很轻松地推导出,苹果依靠知识产权的专利为自己建立了一条很宽的护城河,这可不是谁都可以游过去的;还能推导出,借助它领先的技术优势,它的领先产品还会给它带来更多的收入;如果你已经熟谙这些大佬们的专利战,你还会知道苹果除了罚来的那些钱,接下来还会拿到大量的特许使用费用。你可以换个角度去理解,对于苹果来说,``偷是不对的''不是讨来的说法和结果,更不会哭着喊着让三星道歉,主要是你得买。

从这个意义上说,你最应该做的事情就是趁着利好买进苹果的股票,与世界上最好的公司一同发财,这是一件很体面的事。但是我们每个在资本市场上有着丰富的失败经验和血泪教训的人都知道,如果事情有这么简单,那一定有哪里出了问题。

我们从坏处去想一件好事------虽然在现实生活当中这么做的人个顶个是讨厌鬼,但有一位记者界的Mentor说过一句话:``你妈说爱你,你也要质疑一下!''进入到投资领域,到了把钱拿出来的这个环节,你忍着恶心自己的心情去质疑总是没错的------如果一个技术公司成为被偷的对象,那很有可能是一件坏事,它可能正在从技术创新公司,变成一个扞卫技术荣誉的公司。当然,你也可以说,技术创新和保护技术不被侵犯是可以同时存在的,没错,我相信苹果也是这么做的,但你回过头来想想,这一次让你产生投资冲动和行为的原因是哪一个\ldots{}\ldots{}你买的是苹果既有的那些创新能力,而不是它的未来。

诸位一定还记得史蒂夫·乔布斯和比尔·盖茨那段经典对话,当乔布斯指责盖茨是偷的时候,盖茨说:``我们都有一个富有的邻居叫施乐,我进去他家准备偷东西的时候,发现你已经先我一步偷走了。''被偷的那个施乐,当时拥有鼠标,还有图形处理页面,还有道格·恩格尔巴特这样的思想家和发明家在为其指引方向,但它们也仅仅是一个被偷的家伙,它们可没打算让这些小玩意抢了复印机的风头。二十多年过去之后,这个拥有帕洛阿图研发中心的公司已经更名为富士施乐,你能想象有一个叫``三星苹果''的公司吗?从坏处去想就是苹果会不会有一天变成施乐。

我可不是在主张像苹果和微软在1980年代去偷,我也不觉得三星偷来的那些东西有多牛,成王败寇是人类最庸俗的价值取向之一,但它多少有那么一点意思,我见过的最有力的辩解就是:谁让你暴殄天物的?我要插一句,我对拥有鼠标专利的道格·恩格尔巴特,对于本来可以拥有万维网专利但放弃申请的蒂姆·伯纳斯-李更充满敬意一些,他们看到了发明的伟大意义,于是,他们觉得这是人类的事。而那两个小贼,说老实话,我也原谅了他们,作为一个伪技术控,我确实对暴殄天物这事有那么一点零容忍。如果你是一个老谋深算的投资者,又不像沃伦·巴菲特那样对新东西那么冥顽,你早就跟着这两个小贼发了大财了。

来,我们继续往坏处去想一件好事。施乐只是不知道什么东西代表着未来,也可能是知道但无力去处理,还有另外一种情况是,它为了保持自己的垄断而把技术雪藏起来。这方面的代表是AT\&T,把包括移动电话在内的各种新玩意都藏着掖着。在1960年代之前它有一个保护天使叫美国联邦通信委员会,把调频广播整整藏了四十年,还是为了保护那些企业的利益不受侵犯。没有证据表明苹果也会这么做,也没看出来谁会做它的保护神,当然论市场占有的垄断水平,更比不上通信业电视业那些大佬,但你要知道,这还是一个有阶段性的问题,之前你为它过去的创新而选择投资是混淆了利润的阶段性,现在可能就是迷惑于技术创新本身的价值。

我还是得重复一下,这是从坏的方面去看一件好事。我们的初衷是寻找投资价值。那么,现在可以做一道应用题。我们知道有这样的信息,一是有业内人士批评日本电视制造企业在做一些无用功,比如在人的肉眼根本无力分辨的显示功能上做了很多的技术突破;二是日本的电视制造公司现在日子都不那么好过,基本上亏得一塌糊涂,恨不得把这些业务都卖掉;三是现在全世界都在研究各种屏的价值,没有人能想明白这个曾经最为人类所热衷的屏、也创建了自己成功的商业模式的屏,到底有没有可能再度表现出来它的价值。

好了,问题是,现在如果给你一个领先的电视机制造公司,你要不要买它呢?

一种答案是,既然人类这么依赖屏,为什么大屏没有前途呢?既然PC屏、手机屏都可以催生出技术领域里最好的公司,总会轮到我的吧?我只要慢慢等着有人做出来一个杀手级应用,日本人的前期研发没准就是一种图形处理系统呢。

另外一种答案是\ldots{}\ldots{}我们可以讲一个故事:在汽车刚被人类发明的时候,最开心的人里有一坨是铁匠---这东西是铁做的,我们的大生意来了。众所周知,很快铁匠就失业了,而亨利·福特可没打算在流水线上给铁匠师傅留一道工序。所以,另一个答案就是,可能大屏是有前途的,也可能会带来一个新世界,但这个新世界跟你没有关系。手边现成的一个例子是摩托罗拉,它也一直欢呼着智能手机时代的到来,也欢呼着安卓系统会打破苹果iOS,没准谷歌会创造一个新世界,但是,最后是谷歌并购了摩托罗拉,看起来很有可能的另一个结果是,最后它还会被卖掉。

你看到了富士施乐,这事不奇怪,你会觉得``三星苹果''是一件匪夷所思的事,完全没有可能,但你对``华为摩托罗拉''这种古怪的名字应该不会觉得太意外,你可能还要替某个收购企业掂量一下,并且可能还会暗自问一下:这是图啥?

有时候会觉得作为投资者是一件很恐怖的事。作为一个正派的伪技术控,你要思考什么是偷,偷什么是可以原谅的,偷什么样的东西,那些不作为的超级大公司是不是应该被偷,技术创新到底应该在哪个阶段有价值,你扮演一个什么样的角色\ldots{}\ldots{}有一本书,叫《玩转金钱》,几乎通篇都是安迪·凯斯勒在为他的对冲基金找好的项目和投资人,他对技术公司充满热爱。

最后一个要说的是,最近有一种理论在苹果和三星的专利大战中总是被提及,那就是他们认为这个判决说明大公司的创新更有市场,未来创新很有可能主体会回到大公司身上来------就是那个富有的邻居会成为扛着创新大旗的人。我要说的则是,从贝尔实验室那个时候算起,总是会有人在各种节点或转折点的时候有人谈到这一点---总有人会这么说,但这句话从来没对过。

\begin{center}\rule{3in}{0.4pt}\end{center}

\textbf{《技术垄断》} 尼尔·波兹曼

北京大学出版社 2007年10月

定价:22元

\textbf{《玩转金钱》} 安迪·凯斯勒

世纪文景出版社 2011年1月

定价:38元

\textbf{3个推销自己公司的理由。}

理由1:一个大市场。

理由2:一个不公平的竞争优势。

理由3:平衡这种不公平优势的商业模式。

\section{通过欢乐获取力量}

很久以前,我有一个同事在接受了多年计划经济学的教育之后,成为布哈林的信徒。其基本观点就是拖拉机比汽车重要,拖拉机是生产资料,能种地能运输,比小汽车这种纯粹的奢侈消费要有价值得多,这东西能贡献更多的产值啊。

当时消费社会还没有兴起,我还没有各种伟大理论来把自己武装起来,所以只有听他讲的份儿。事隔多年以后,总结下来的第一条就是革命者们研究起经济来,总是像攻山头一样目的明确,如果还是坚定的革命者,那还有排除万难的气魄,不达目的誓不罢休,如果还有资源和权力,那么就到了集中精力办大事的境界,不把人搞成拖拉机肯定不会回头------当然,有的时候这还会成为一个伟大成果,大家都是生产资料,人也可以是传说中的螺丝钉。

相比之下,文艺青年研究经济要更靠谱一点。比如说比布哈林稍晚一点,有一个文艺青年就认为汽车这东西更重要。在他的理想中,这种汽车的成本不应该超过1000元钱,最高时速应该超过100公里,油耗不要超过百公里7升,空间应该能容纳两个成人和三个孩子\ldots{}\ldots{}我觉得这是一种很伟大的想法。在提到这种满足基本需求的汽车的时候,浮现在眼前的总是那个冰冷冷傲骄的亨利·福特和他的T型车,因为他成功了。但实际上有这种理想的人可不止他一个,也有几个不那么成功的,于是大家就都忽略了,这其中就有这位文艺青年。

关于这个故事完整版是这样的:文艺青年阿道夫·希特勒在搞定了德国人之后,找到保时捷家族,决定生产一种``人民汽车'',希特勒还亲自为样品设计提供帮助,成立了德国大众汽车发展会。并且在1939年,还建成一个工厂,把工厂所在地小镇更名为Kraft
Durch
Frende------通过欢乐获取力量。一个叫``通过欢乐获取力量''的小镇公司,还是在生产一种满足人民大众需求的产品,尤其值得一提的,还是一个政府工程,我想我会很乐于投资这样一个公司。想想吧,沃伦·巴菲特遇到一个``超越你的梦想''(BYD)都会投资,我们为什么不投这个又欢乐又正能量的KDF呢?

当然,这些车就像``超越你的梦想''的E6没生产几辆一样,KDF一共只生产了200辆就因为战争打断了计划。整个德国重工业从此就走上了布哈林之路,忙着造坦克去了。总之,文艺青年因为兴趣转移,而断送了他的人民汽车计划。顺带着也毁了法国人的``人民汽车'',众所周知,在法国文艺青年以及文艺中老年基本上覆盖了全部人口,所以他们也有自己的浪漫计划,他们对汽车的要求是可载四人,或者装载两人以及一大袋土豆。这种车型在1936年出台,1939年生产了250辆,然后就中止了。

时光荏苒,当这些人民汽车再度出山的时候,通过欢乐获取力量的人民汽车化身为大众甲壳虫,几经辗转,终于回到了文艺青年们的大众手中。法国人的浪漫也没有停,1949年恢复生产之后,直到1990年雪铁龙停产。与此同时,英国的人民汽车也修成了正果,奥斯汀-莫里斯生产的Mini汽车也成为文艺青年的最爱,只是它现在也成了德国汽车的宠物了。

在探讨消费社会的形成这个问题的时候,总是与中产阶级脱不开干系,中产阶级与布尔乔亚是一组同义词,但这回你一定注意到了我在不停地说文艺青年,我觉得在消费社会的形成过程当中,小布尔乔亚发挥的作用一点也不输于布尔乔亚,虽然小布尔乔亚和文艺青年不能划等号,但他们的相似度绝对可以超过90\%。这就好比说,在创造消费社会这个过程当中,亨利·福特搞定了布尔乔亚,那么Mini、甲壳虫这些就是搞定了小布尔乔亚。前者看着正襟危坐,但实际上心思缜密,算准了你的需求:卖给你一种产品,给你制造更多的空闲,然后再卖你一种产品,把你的空闲给填满。后者则貌似漫不经心:让你产生一种需求,然后让你厌倦它,然后再让你产生罪恶的空虚感,最后再通过新的需求填满这个空虚。前者取了个好听的名字,叫有用;后者取个更好听的名字,叫有趣。

至此,在这两位老兄的共同努力之下,消费社会终于有了那么一点样子。前者的致命武器是那些被称为``营销大师''的人物,比如一个叫布朗妮·怀斯的女人,其实她就是一个卖保鲜饭盒的,她精明的销售策略是对家庭主妇身份的重新界定------从家务工人和消费者变成``女主人''的这个过程。她通过提供给郊区妇女最需要的东西(一种社交生活而非食品容器),并构建这种欲望,将其``特百惠化''。怀斯带领着人们穿时髦衣服,摆放火烈鸟的粉色装饰软垫,开起粉色卡迪拉克。实际上怀斯为特百惠设计了一个形象,一种生活方式,而这正是妇女们购买这些简单的塑料制品时将会获得的。再比如,即使在1950年代,胡子也是男人诚实可靠的形象代表,但架不住有一个卖刀片的吉列,最恨胡子了,于是铺天盖地的营销广告,最终让男人们放弃了胡子------幸好还有文艺青年们保留着它,当然,这是作为另外一种生活的代表。

登峰造极的大师出现在马克西姆·格拉德威尔笔下。他提到一位大师在做一个维生素泡腾片的案子的时候,随手扔到玻璃杯里两片:为什么不是放进去两片,这样我们就可以赚双份的钱了。

我一直怀疑宜家就是这么想的。这是一个很有趣的公司,比如有的时候它是布尔乔亚的,便宜、实用、质朴,把自己打扮得跟沃尔玛一样,有着北欧农民的劲儿,但它又充分把握了质量的度------既不能太坏,丢了布尔乔亚的本分,又不能太好,太好哪里有机会赚双份儿的钱呢?所以它有的时候还得表现出小布尔乔亚的那种文艺劲儿,比如在几年前的中国。

当然,小布尔乔亚这一消费分支也一直没闲着,他们一直进化到形式与功能的分离---有那么一群文艺青年,在推翻了布尔乔亚的``优良设计''原则之后,主张``形式追随意义'',而不是追随功能,这伙人由一个叫埃托·索特萨斯的人统帅,他们开始管自己叫``新设计''。但后来觉得不够有特色,有一天在鲍勃·迪伦那首抒情歌曲《被困在莫比尔小镇,再听孟菲斯的蓝调》的音乐声中,他们给自己更名为孟菲斯。

我之所以提这个故事,是因为孟菲斯实在很符合形式追随意义的消费理念。索特萨斯说:``好的,那就叫孟菲斯吧。包括蓝调、田纳西州、摇滚、美国郊区,还有埃及,法老的首都,仆塔神的圣城在内的所有人,都觉得孟菲斯是个绝佳的名字。''我能从这样一堆名词中想象,小布尔乔亚、文艺青年们是如何得意于这样一个时刻的。

兰迪·科米萨,我们可以把他看成是一个准文艺青年,有一天他在硅谷的咖啡厅里跟服务生说:给我来一杯无咖啡因脱脂奶特咖啡。服务生白了他一眼:你现在也搞这一套,好像很爱惜自己身体的样子。在我们这里,这玩意叫``不如不要''。我觉得兰迪·科米萨的故事告诉我们一个消费社会的真谛:它起因于实用家们的理想主义,让一些东西成为你的必需品和``欲望之物'',然后是欢欣鼓舞的布尔乔亚们尽情地享用,然后是小布尔乔亚们对空虚和意义的指证,带来新的消费观念,让孟菲斯一样的亚文化成为新的主流,最后一击就是那些卖泡腾片的大师,他们心里想的是卖个双份儿的钱,但实际上给我们的让我们乐此不疲的那些东西是``不如不要''。

\begin{center}\rule{3in}{0.4pt}\end{center}

\textbf{《大设计---BBC写给大众的设计史》}

彭妮·斯帕克

广西师范大学出版社 2012年1月

定价:168元

\section{我们所热爱的青春期}

最近关于理性、爱、国家、热情等等之类的词很泛滥,我还在视频截图上看到一些脸,当然,你可以读出来一些邪恶、暴戾,你也可以不按文字的指示读出热情、冲动,更有一种可能那就是一些无所事事、迷茫而且年轻的脸。

几年前看过一本叫《青春无羁》的书,不那么好看,感觉就是一个才力不逮的乐评人把自己对这个世界中有关青春的历史和知识做了一个总结。事后想来,这本导致我很困惑的书并非是作者的脑子有多糊涂,而是青春和青春期的表现形式实在是过于多元了,这本来也是一个很难被定义的东西。比如,他写到一战结束之后的德国的一群年轻人,他们才刚刚说完``我们必须有勇气与祖国保持一定的距离,与强制灌输给我们却不让我们思考的爱国主义保持距离。''之后几年工夫就成就了``德国青年联盟'',然后服膺于纳粹的光环\ldots{}\ldots{}在书里提到的一种解释是,在他们这群``保持距离''的青春期家伙中,另一条发展脉络是彼得·潘、童子军,然后是候鸟运动,这听起来就又红又专,他们也是追求一种纯洁------纯洁导致排斥,导致独裁,这个应该不难理解,但如何就成为工具了,成年人又是如何``引导''和``利用''的,他们在这里究竟做了什么,我是没弄大清楚的。

相比之下,我还是审慎地对彼得·潘表示一点敬仰,龌龊的成人世界啊,请迟一些向我走来。但是,如果你知道《彼得·潘》的作者悲催的人生和不明不白不清不楚的性取向,你多少还是会有一点绝望的。好吧,我本来不想提这个事的。

关于青春期,我很喜欢安迪·沃霍尔的一句话。``对于青春期我记得的不多。由于体弱多病老跟查理·麦卡锡玩偶躺在床上,我大概错过了绝大部分的青春期,就像我错过了《白雪公主》一样。我一直到45岁才看了《白雪公主》\ldots{}\ldots{}我等了这么久或许是件好事,因为我想我不可能有比当时更兴奋的了。''我把这句话理解成是另外一种向彼得·潘致敬的方式。安迪·沃霍尔觉得既然如今我们活得长命得多,我们停留在婴儿期的时间真该拉长一点。

按照《青春无羁》所暗示的,社会节奏加快,让一代人与上一代人之间看世界的方式发生剧烈变化,所以代沟啊、反叛啊就产生了。这种说法当然也是对的,但显然保罗·格雷厄姆这个非青春期研究者的说法更靠谱:为什么青少年会有各种与成年人世界相冲突的地方?因为整个世界正处于越来越严重的``专业化''趋势当中,当工作专业化程度越来越高的时候,就必须接受更长时间的训练。在工业化之前,儿童14岁就要参加工作,阿尔·卡彭当芝加哥黑手党老大的时候才25岁,杰西·利弗莫尔15岁就已经开始炒股票了,22岁的时候已经作为着名股票作手破产一回了,亨利·卢斯虽然14岁还在念高中,但已经与他老爸平起平坐地探讨中国、美国以及人生的问题\ldots{}\ldots{}保罗·格雷厄姆觉得,``如果性和心理的成熟时间都发生在他与成年人的社会建立起关联之后,他更应该做的似乎是适应成年人。坏处是来不及思考,他就已经融入到这样的生活里去了,好处是不会为太多青春期的压力而发愁。''现在呢,你要接受一个最基本的教育也要到22岁,否则你是没办法跟上这个社会的节奏和脚步的,哪怕你是一个富二代或者官二代,这必备的流程也是不能少的。你觉得你生理上已经成熟了,但你却是一个既赚不了钱也不能给社会带来价值和财富的无业青年,你还要处处受制于父母、学校和社会的管制。这是一种发展速度不匹配的结果,不叛逆才怪。

现在,我们再也找不到25岁做黑社会老大、22岁大生意破产一回这样的机会了。但我觉得还是不能妨碍那些冲动的天才偶尔划过夜空出现在我们的世界当中,否则就难以理解为什么哈佛的比尔·盖茨和马克·扎克伯格会退学,史蒂夫·乔布斯也会觉得在里德学院无所事事无聊到爆,他们的青春期如此澎湃,根本等不得拿个什么寻常的文凭,直接就让青春期的荷尔蒙与改变世界的宏大理想搞在一起了。

我觉得这种顺应青春期的召唤是我们地球上最美好的事情之一。当我们发现有这样的迹象发生在我们的身边或者我们的时代时,我们无论如何不能错过。林恩·马古利斯这个进化论学者认为青春期的身体最富魅力,但是在进化过程当中,这个身体也是最具反叛性的时候,``对于动物的身体状况来说,青春常驻,长生不死这个愿望是无法实现的。但我们个体的失败却是细菌的胜利,因为它们会将我们体内的碳氢化合物回归于生生不息的大自然。细菌与生命的原始状态更为接近,它们不像我们一样必然死亡。除非发生意外事故、突变或与其他细菌发生基因交换,单个的细菌都可以其原始形态依靠细胞分裂而一代一接一代地复制自身,`永生'下去。''

看到了吧,这种劲头如果我们愿意赞美他们,那就是这帮家伙,他们对抗的是整个人类和进化的那种循规蹈矩:虽然我们每个人都经历过那种大逆不道的时候,虽然青春期面前人人平等,我们每个人都有足够的资本来炫耀挥霍我们的青春期,但只有他们让这种青春期绵延于他们的生命始终,始终显示出超凡的力量。

我得承认,虽然青春期这玩意难免与冲动和破坏性搞在一起,有的时候还显示出一种脑残的味道,但因为它乱糟糟的美感是我们的生活中所缺少的而让我迷恋。我能回忆起来的是1980年代,怎么说呢,那个时候又谦虚又冲动,又违和又有上进心,迷茫么总是有的,但更多的则是带着一点酒劲的,狄奥尼索斯式的混不吝往前走的感觉。我觉得那也是中国的青春期。

当然,戛然而止的青春期,狄奥尼索斯们巨大的失望会扭曲一个时代。在我看《青春无羁》的时候,最大的担忧如影随行:那些神采飞扬的青春一代,当他们于失望当中世故庸俗犬儒主义起来,青春期所积聚的幻灭和失落会凝结成多大的恶呢?

好在我们还有像刘易斯·托马斯这样的人,他说,``可是我们也不像有些人说的那样坏。一个世纪以来,时兴贬低人类,说我们是一次失败的尝试,是一场必输的游戏。我不同意。往顶坏里说,我们也许正在度过一个物种的青少年前期,这个时期什么感觉,想必人人还清楚记得。成长的烦恼,对于个人来说固然艰难,对整个物种就更是长期的磨难,对于一个像我们似的聪明又敏感的物种就更是如此。假如我们能平安度过这个阶段,甩掉这个世纪的记忆,静待一段时间,或许就会柳暗花明,再度起步上路了\ldots{}\ldots{}只有一个理由让我能原谅我们自己,就是,我们是游戏中的新来者,还没有怎么学会它。我们来这儿没多久,没资格用古生物学者和地质学者的语言去谈论自己的生活习惯。我们几乎是刚出炉的,说这个物种尚处于青春期,似乎还未免有僭越之嫌。''

这话说得真好。但乔治·卢卡斯或者道格拉斯·亚当斯,那么多人的想象还没有实现,我们就真的已经到了青春期?我也觉得还远着呢。不过,没关系,你我反正活不到人类尽头,在我们的生活当中,有我们自己的反叛的teenage的青春,有经历过中国的迷人的1980年代,有现在的纠结反思自己做得不够好的人类\ldots{}\ldots{}它总还是迷人的。与那个让人绝望的成人世界,我们还有距离,这就是我们的好世界。

\begin{center}\rule{3in}{0.4pt}\end{center}

\textbf{《青春无羁》}

乔恩·萨维奇

吉林出版集团有限公司 2010年4月

定价:54元

\section{为什么奥特曼不能在银行里下棋}

一到假期,关于公共服务就是热门话题,拜生活水平提高所赐,上一个十一黄金周中公共服务的焦点已经从铁路民航上升到高速公路大塞车。当然,免费是一个重要原因。一种服务从收费到免费,其服务体验会打一点折扣,这本来是可以理解的,但如果体验太差,那么就要被各种``理性''分析。我是蛮怕太理性的头头是道的分析的,因为我总觉得这世界上大多数事道理其实很简单,本来用不着说太多的------我看到的通常都是对某事的解释,以及对某个解释的解释,以及他为什么这么解释的解释,以及这些解释说明了什么的道理。

我所感觉到的,就是我们的高速公路公司提供的服务不怎么样,但我对它们本来也没有太高的期待,所以还可以接受。但有一个问题我一直弄不清楚,比如它们通常给自己提出的收费理由是``贷款修路,收费还贷'',这听起来像是以借贷方式为主来融资的行业,它们还大多是上市公司,对于主营业务就是这些个收费站和收费员的公司来说,它的业绩好坏前景多空,应该也是跟这些收费息息相关的。那它们上市时融的资主要是在干嘛,增发的钱主要花在什么地方------我也没做过什么调查,但我总觉得它们来来回回拿了很多钱。在拿着上市公司身份的时候,它们可以给股东谋些好利益;获得贷款的时候,它们是银行的优质客户;当然,它们还有一重身份是国有企业------通常的说法是,它还是公共服务的提供者,虽然你是要付出很多的钱才能享受这公共服务的。总之,这就是一个潜伏在收费一线的假装自己在市场里游弋的垄断公司,实际上就是一个政府的代理。

作为一个被自由主义经济学所毒害的人,以我对我们政府的理解,它们在看到``坐地收钱''的民营公司的时候,或者是同流合污分一杯羹,或者是举起``国计民生''的大旗,直接控股吃下。有一些不谙世事的人会憧憬一种理想的状态:如果那些高速公路的经营公司都是民营的,而政府只是一个监管者的角色,那么就在还完贷款之后------从此高速公路就免费了。我直觉这事不大可能,什么原因,我说不大清楚。迈克尔·刘易斯说到希腊政府的新总理---他号召希腊人重新建立起集体意识,不遗余力地劝希腊人不要再偷漏税------这个人不仅是正直而已,他简直不像希腊人。如果说原因,我觉得刘易斯这段话可以做个注解,有一些事跟你想的不一样了,是不是有点太奇怪了?

你知道当一个人在大一统、权力集中的环境里生活得久了,会变得犬儒起来,有的时候还会有点受虐倾向---有一段时间我就很替铁道部的人鸣不平:现在有了动车高铁,这难道不是好事吗?难道要我们都去挤绿皮车吗?现在一个春运要运多少人,原来才运多少人,这不是在能解决的情况下解决得已经很不错了吗?为什么要以春运甚至是除夕前3天的运量来设计中国铁路的运能呢?

跟我这种有点犬儒的想法接近的,是另一个绝不犬儒,甚至是号称绝不妥协的人,他的名字叫托尼·朱特。他的观点是,铁路这玩意可没办法搞自由市场、搞竞争,你不可能在纽约和波士顿之间摆上两列火车,你来挑选哪个服务更好,更不可能修两条铁路线,这东西就是自然垄断。你不能靠竞争来开火车,跟农业或邮政一样,它不但是一个经济活动,它还是公共利益,当然要垄断起来。并且因为是公共利益,所以一个有自尊的企业也不能接受书面指示:销售什么,销售的价格范围,以及特许它们经营的时间和期限(我们这里号称上市的而且在市场经济里的按照收费多少来计算还贷日期的高速公路公司们在这一点上可没有什么自尊。到底是不是免费,是免费发卡通过还是免费不发卡直接通行,都有人来指点的。)---所以,不但要垄断,而且只能由公共部门垄断。

我觉得不管是我想的,还是托尼·朱特所想的,都没有什么错的。如果我们不知道投资1850万元做一个拍得很烂的宣传片,这其中还有250万是给一个喝点茶水打个酱油的张艺谋;如果我们也不知道3.3亿元搭个卖票网站,那么国营的而且垄断的铁路确实是可以接受的。但问题就在于,你永远不能低估一个部门做蠢事的能力。当然,它们也未必是在做蠢事。你还可以说这不是国营、垄断的错,而是不透明,不公开或者说是制度本身存在重大缺陷。但希腊告诉我们,即使可以透明公开,即使制度本身也都建设好了,它还是难以避免一些匪夷所思的地方。

希腊的国营铁路一年收入是1亿欧元,而它们的职工------在它们那里这些人算是公务员------1年工资支出是4亿欧元,还要再加上3亿的运营支出,国营铁路的员工平均收入是6.5万欧元。你可以把这笔账算在欧洲一体化造成的各种笑话里的一个,但是在20年前,希腊就有人算出来,出钱让所有火车旅客搭出租车,也比它们的铁路公司要省钱得多。所以,当我听说欧盟得了诺贝尔和平奖的时候,我的牙都笑坏了一颗。

我在看迈克尔·刘易斯的《自食恶果》的时候---他讲了几个悲催的欧洲债务危机中的国家的故事,它的负作用就是又让我觉得相比之下中国铁路还对付过得去。很多人在微博上提出对铁道部的质疑时,核心的问题是``一票难求''!但实际上那些家伙在生活上大多数时候对铁路的要求是``要舒适,要快,要正点,要便捷'',我总觉得这个比起过去来说要好很多,如果再加上一个``要廉价'',这听上去就跟找男人找到``奥特曼在银行里下棋''这一步差不太多---很有意思的一点是,似乎价格始终不是舆论关注的重点------但实际上这个还是挺重点的,如果前面的四个都解决了,接下来一定是这个。

当然,还有更多的对铁路的质疑就是它们的欠债未免太多了。但我觉得这个时候倒是可以用上托尼·朱特的理论。在修铁路这一点上,一般来说都有一些非市场的力量在推动,表现为总有那么几个铁路狂人做了一些惊天动地的事,然后这些人也不会在意它的收益比,所以我们看到铁路开始的时候总是表现出一副烂摊子的样子。而铁路的建设资金,你要不是发债券,那就一定是上市融资,把你未来的收益卖给那些看好的股东------它的风险就是美国19世纪铁路的风险,因为资本结构复杂,铁路沦为超级概念,所有大佬都于其中风云际会,留下诸多丑闻---当然,铁路也留了下来,并且最终让美国成为一个整体市场。

这里还是可以提一句希腊,当希腊在财政预算赤字已经瞒不住的时候,高盛过来恶狠狠地``帮''了它们一次。它把大量的资产和未来收益都证券化,其中也包括铁路的未来收益,再卖出去------就跟高盛在次贷危机里对那些可怜的拉美裔和非洲裔美国人做的一样,哪里有弱者有loser,哪里就有高盛。

我提到这个的原因是想说,如果铁路未来的收益有那么一次证券化------其实对于中国来说未必是一件坏事。反正你也看到了高速公路上的大堵车,你也看到了公路运输费用的高昂,所以你真不用太担心铁路的竞争力。虽然它是垄断的、国营的,虽然它看起来总是磕磕绊绊的还不那么好看,虽然经常让你不爽,但可能这也是最后一个从垄断的、国营的提供的合适的机会了,或者说它可能是最后一个还在银行下棋的奥特曼。

\begin{center}\rule{3in}{0.4pt}\end{center}

\textbf{《沉疴遍地》}

托尼·朱特

新星出版社 2012年3月

定价:25元

我跟你各有一只猫与狗,我用10亿把猫卖给你,你再用10亿把狗卖给我,这样我们就各有非凡价值的宠物啦。

\textbf{《自食恶果------欧债风暴与新第三世界之旅》}

迈克尔·刘易斯

财信出版 2011年12月

定价:新台币280元(合人民币60元)

\section{它不像你想象的那么简单}

我看了一本叫《僵尸生存指南》的超级无厘头的书。它的无厘头主要表现在煞有介事,如果你把那本书里的僵尸替换掉,可以把它当做``当敌人入侵时,如何有效进行持久战'',或者当街头暴乱发生时,你如何自处,或者走夜路自卫指南,以及所有的枪支武器在什么场合下如何使用之类\ldots{}\ldots{}它一本正经地讲这些事并配上插图,我在厕所里看它的时候经常哈哈大笑,所以我真不建议大家看这样一本书。

通常来说,僵尸进攻人类,你的注意力全放在僵尸身上,这个时候来了一个或一坨人类,如果你又寂寞已久,你难免同仇敌忾将对方引为同道知己\ldots{}\ldots{}难说啊老兄!我觉得这是亮点,也是这本《僵尸生存指南》里有限提供的有价值的内容:连这么无厘头的书,都不忘传递厚黑的悲观人世哲学。

我虽然对作者心生一点敬意,但我也不是完全觉得他说的就很正确。我相对来说可能会更悲观一些:人会产生恐惧,而这个时候如果有同类出现,那么谁给谁提供安全感就很重要。看过美剧《LOST》的人,会想到在那个小岛上面对那个总也看不见的黑风怪物大家是如何形成一个领导核心的,我差不多就觉得这是人类的永恒命运。想一想吧,有那么几个人,弄了几架飞机撞了两个大楼,我们知道那是谁干的,我们也知道最坏结果会怎么样,我们差不多也知道如何去预防------这也不是太难的事。但接下来都发生了什么?接下来就是国土安全部的各种权力,就是公民电话窃听,就是关塔那摩,就是阿富汗战争,就是伊拉克战争\ldots{}\ldots{}只是因为美国的公民们感觉到少了一点安全感,你看为了这个安全感,要让渡多少自由给别人,最后我都分不清谁在制造恐慌------谁更能在恐慌中得利,谁对这个更起劲一些。

所以当僵尸入侵之时,最担忧的并非是那几个有可能起了歹意的小歹徒,而是突然来了指路人,他为你指引方向,他把你们这些群氓组织起来,然后如果指路人更有心机的话,他还会牢牢掌握住僵尸动向的解释权,让你始终保持警惕\ldots{}\ldots{}和恐惧,这还真不是一个让人乐观的世界。

刘易斯·托马斯曾经做过纽约大学和耶鲁大学医学院的院长,还是美国总统某个顾问小组的成员,用老话说依他的学识看起来就是圣贤之人,新的说法就是一个着名公知。作为一个反对核武器的代表人物,他也研究过类似的问题。他说:``这些年头我们甚至听到争论,争的是多少百万条性命才算可以接受的数目,这个可接受的水平算是有限度的核战争,这么多命双方都能损失得起,而不至于消灭各自的社会\ldots{}\ldots{}''其实,哪里有这么简单!在刘易斯·托马斯的分析中:大量尘埃和烟雾炸向大气中,可能在数月乃至一年内使整个北半球暗无天日,阳光可能被屏蔽掉99\%,内陆地表温度将会剧烈下降到摄氏零下40度\ldots{}\ldots{}

你看,虽然计划只是杀死几百万个敌人,但问题可不那么简单,最后很有可能把这个地球给毁掉了。不过,我要说明的是,一是圣贤托马斯是上一代或者上两代辈份的公知,他们还多少秉承了罗马俱乐部那些人灰暗悲观的态度;二是某种意义上此圣贤在无意当中也成了那个制造恐慌的家伙,虽然他本人肯定不会是这么想的。

散布恐慌情绪肯定是不好的,但你也要知道,多想一步总没错,一个事件发生了,表面看一切都在掌握当中,但实际上可能早就酝酿大变化了。

19世纪芝加哥有个叫古斯塔夫·斯威夫特的屠宰场老板,他觉得肉类加工是件极其费时费力的事。当年,牛仔们要把牛从草原上赶到火车上,一路运输到中西部的芝加哥或者更远的东海岸,这个过程当中牛会得病会死会因照顾不周而瘦成皮包骨\ldots{}\ldots{}于是,他弄了一些冷冻剂放在封闭车厢里,就此发明了冷藏车。这可不是一件小事,接下来会发生什么呢?运输效率空前提高,原来一个车里塞进去几头牛,现在则是已经切割好的整条牛肉挂在车里(一百多年后,宜家发明了家具的平板运输,它们的祖师爷在芝加哥),原来他要为那些牛内脏之类的提供运费,现在这些就可以在上车前就拿掉,运输过程当中牛不再有死伤\ldots{}\ldots{}整个食品加工业的生态就全变了。

如果生在斯威夫特的时代,你不幸又是个屠夫,如果你只看到那孙子卖些死牛肉,而老子现杀现卖绝对新鲜,那你就死定了。威廉姆·戴维德还是在讲那句话,多想一步总没错,它不像看起来那么简单。

他把这个话题引到正反馈和负反馈这个话题上来,举的另一个例子是1987年的美国股灾。美国政府先是宣布贸易逆差为157亿美元,那个时候可是欣欣向荣的里根政府时期,这远远高于市场预期,于是美元贬值,市场担忧通胀,再接下来30年期国债收益率首次超过10\%,然后,美国国会这时火上浇油,说正在认真考虑取消对公司营业额的税收优惠政策。

这不但打击了企业的盈利前景,而且又降低了企业兼并的吸引力,股票需求量随之降低------我们都知道1980年代末的垃圾债券大王迈克尔·米尔肯,我们总是忽略他的成功一样依赖前面说的这些大背景------现在所有这些貌似都要停止了。于是整个股市就此急转直下。

单纯的157亿美元贸易逆差只是一个开始,不简单的都在后面。而更不简单的则在于这个时候出现了电脑,出现了一种叫``投资组合保险''的反馈机制,一种风险管理工具。一旦股价下跌1\%,电脑软件会自动将客户的资金抽离股市。戴维德作为英特尔的前副总裁以及风险投资人,提醒了我们过度互联正反馈的可怕之处。你可以说程序控制对于市场恐慌来说,它表现的是一种淡定------机器才不会像那些受肾上腺激素分泌支配的浅薄的人类那样做决定呢,但它实际上更多的作用是加剧恐慌。

有了电脑、软件、程序、互联网,那么你看到的是恐慌的升级和迅速传播。而这些都不是大家所能控制的。就像僵尸出现时,你不知道哪个指路人最终会成为老大哥,也像刘易斯·托马斯所担忧的,你虽然只想杀个千八百万人就够了,但你没想到的是毁灭地球。

圣贤说:在这样的事件中,人类幸存的问题几乎就无关紧要了。当然,有一些人会渡过劫波,甚至能活下去,但是,生存的条件将比一二百万年前刚刚出现的人类苛酷百倍。文明,以及文化的记忆,将会一去不返。既然有了我们物种这样的大脑,还有记忆的天赋,那么,留给散落各方、四下呆看的幸存者的,只会是一种负罪感,痛悔自己对于这么可爱的一个生物做出这样的破坏。对于这个惨淡的开端,那将是一份惨淡的遗产。

\begin{center}\rule{3in}{0.4pt}\end{center}

\textbf{《最年轻的科学》}

刘易斯·托马斯

湖南科学技术出版社 2011年10月

定价:48元

\textbf{《过度互联》}

副标题:互联网的奇迹与威胁

威廉姆·戴维德

中信出版社 2012年8月

定价:36元

\textbf{《僵尸生存指南》}

副标题:如何在活死人横行的疯狂世界求生

巫毒僵尸通常都在以下地区遇到:撒哈拉沙漠以南地区,加勒比海,中美洲和南美洲,美国南部地区。
马克斯·布鲁克斯

江苏人民出版社 2011年12月

定价:36元

\section{工匠们创造的历史}

有没有过那种时候,有一个叨逼叨叨逼叨的人在你身边一直说啊说,你又说不过他,最后只好埋头去做给他看------当然,大多数时候那些没完没了的话你都可以视若无物,大多数人你更不要去理他,所以我可不主张去跟他们争口气。但一定有个别时候,你觉得值得这么做一下才可以让他闭嘴------我虽然一生所为离创造性的技术相去甚远,但我还是觉得技术这东西是身为人类值得迷恋的一种东西。简单点说,就是哪怕你是一个宏大事件的爱好者,成天研究的都是道,但你也不能把技术看成是一些可有可无的精巧的小玩意,它们有很多让人兴奋的元素,你不能简单把它们一律归为形而下者那一坨里去。

这里有一个有趣的现象,如果用弗洛伊德这样的理论去套一下它的话,就是喜欢大词谈些主义的,总是跟精神分析派们爱提起的那些``口腔欲望期''有关。我总觉得接下来你长大成人,就应该真刀真枪地上了,就不能再耽于口唇之欲了。所以胡适与鲁迅当年的问题与主义之争,我的观感就是这有毛可争的,这是人的不同成长阶段啊,互相说服或批判都是使错了劲。

话说在12到13世纪那一段时间,本笃会修道院的师父们觉得每天看太阳来算时间侍奉上帝,总是不够严肃,碰到个阴天就会感觉怠慢了上帝,于是觉得应该有一种计时工具来让修道院的日常事务有章可循。

在这个时候,发明钟表相对于修道院来说,就是一个小玩意。但过了一百来年,钟表被工人和商人利用起来,再隔了几百年之后,一个叫刘易斯·芒德福的人说:``机械钟表使按部就班的生产、准确计时的工作和标准化的产品成为可能。''------如果没有钟表,资本主义的兴起是绝无可能的。

我们用一些大词来为这事总结的话,就是历史变革背后的技术推动;用另外一个体系里的大词来说,就是,群众,人民群众创造了历史;如果平实一点,我们可以说:有的时候,什么Geek啊,Nerd啊,工匠啊,死宅技术派啊弄出来的一些新东西,也绝不简单。

史蒂夫·沃兹尼亚克那种说得太多了,乔布斯要改变世界,你总得有个趁手的家什。阿基米德要把地球撬起来,沃兹就是乔布斯改变世界的那个支点。当然,沃兹不这么看,他觉得他是一个发现问题和解决问题的人------这对我们的启发就是,当年的问题与主义之争,另外一种解读方式是:问题有无和解决与否实际上就是主义本身。

跟13世纪相去不远,还有一个技术控的例子。这个人叫约翰尼斯·古登堡。作为一个修道院的供应商,成天印赎罪券,他一定感觉又累又烦,重复劳动。同时作为一个爱动手的工匠,他就把一个酿酒机改成了活字印刷机,他的世界果然就焕然一新了------他还印了《圣经》,还印了``大便历''。前一个我们很容易理解,后一个粗俗一点说就是``今日宜大便''这样的玩意。我还没考证过中世纪的便秘情况,选个黄道吉日如厕轻身------这就是约翰尼斯·古登堡这个技术控对人类直接的重要的贡献。

间接的,40年之后,印刷机进入6个国家的110座城市;50年之后,印刷书籍达到800多万册。过去人们无法得到的信息,涉及到法律、农业、开发、冶金等;商界迅速成为印刷品的世界,合同、契约、本票、期票和地图都普遍使用开来。``信息标准化了,且可以复制。在这样的文化里,地图绘制人在地图上剔除了天堂,因为天堂的位置太难以确定。''当然还有最着名的马丁·路德,他看不惯赎罪券赤裸裸的商业用意,于是发起了对天主教会的批评。1517年,他印刷了《九十五条论纲》,于是新教革命开始了。

一个修道院的供应商,印赎罪券的家伙,就这样把自己的生意给毁掉了。所以你要说技术控们有多少胸怀大志,我就觉得这事不那么靠谱。但我总感觉他们虽然无意于这些伟大事业,但如果看到了一个天翻地覆的结果,他们会由衷地高兴。就像60来岁的沃兹总是乐颠颠地排队抢购苹果的新产品一样,这事还真是很值得高兴一下的。

如果约翰·哈里森活到现在,他应该比沃兹还要高兴。有关这个人显然知道的人不多,我们可以把他看成是18世纪的Geek,他一辈子做的事就是做了4个航海钟。从这个意义上说他是一个表匠,但有一个插曲让我觉得他还是一个木匠。他为了让钟表走时更准确,找到一种解决方案------用木头做钟表的关键零件。木头跟所有的东西一样都有热胀冷缩的麻烦,于是他选用了不同膨胀系数的木材镶嵌在一起,让零件始终保持一个恒定的标准。我觉得这创意足够天才而且实在,除了作为一个木匠对木头的深刻理解,没有什么理由可以解释。当然,我说他是一个Geek的原因并不仅仅他是一个好表匠和好木匠,他做航海钟解决了计时准确性问题,但他的最终极目标是如何在航海过程中定位------人类很早就通过星座之类的定位解决了纬度问题,但经度问题的解决是在计时准确的基础之上。伽利略啊牛顿啊发现彗星的哈雷啊这些大人物都没有解决掉,就是因为他们不是一个好工匠,但哈里森是。所以我推荐一本叫《经度》的书,它就是讲这段故事的,如果你很懒,还可以到豆瓣上去找一个叫``远游人''的ID写的简介,我觉得一样很精彩。

我觉得,约翰·哈里森差不多是工匠领域里的一个高峰,在他之后还有一个高峰是伊莱沙·奥的斯父子,他们是完全以工匠的标准、直觉和解决问题的方式一手创建了一种叫``电梯''的应用和行业,到现在我们还在用他们父子的产品。但这个高峰之后工匠师傅们就再没有这样的荣耀了。

这事要归咎于亨利·福特,在他的生产线之前,拿一个榔头的工匠就是设计师和工程师,在他的生产线之后,那个工匠就是一个榔头或者扳手了。这也是一件很遗憾的事。

好在Geek还在。现在一般把Geek跟程序员们联系在一起,我们都知道,程序员们相对来说喜欢自嘲一点,程序猿啊,码农啊,这些都是他们自己编出来的。我觉得这倒很可能来自于他们的前辈的基因。他们工作的特点、从事这个工作的职业态度和素质要求,还有他们的性格,似乎都跟我们人类工业革命前的那些工匠有相似之处。你也可以说程序员是智力劳动工作者,比如叫工程师,但有一点差别的是,他们分明是撸起袖子就要解决问题的------记住哦,不是给出解决方案,而是直接动手解决掉。他们显然不能被视为艺术范儿的设计师,但你看他们追求的东西往往又要好看又要实用又要简洁,这是对最好的设计师的要求,也是一个体面的工匠所要具备的素质。北方有谚语说,巧妇和面要有三光,手光面光案板光------这是对一种普通劳动的赞美,你看,把它送给程序员可以,给工匠也可以。

\begin{center}\rule{3in}{0.4pt}\end{center}

\textbf{《魔球》}

副标题:一个勇敢面对自己, 逆转胜的真实故事

迈克尔·刘易斯

早安财经文化有限公司(台湾)

2011年11月

定价:新台币350元(约合人民币75元)

\textbf{《IT不再重要》}

副标题:互联网大转换的制高点---云计算

随着互联网的能力、范围和实用性的扩大,它最具有革命性的后果可能不是电脑开始像我们一样思维,而是我们将开始像电脑一样思维。

尼古拉斯·卡尔

中信出版社 2008年10月

定价:29元

\section{日光之下,并无新事}

关于把钱借给别人,最纠结的事在于你总是不好意思提醒这个家伙应该还钱了,但心里又成天惦记着这个事。关于借钱的另一个悲剧是,借钱给别人是换不来这个人的友谊的,他会不断地质疑你:为什么成天盯着我还钱?我要是有钱早就还你了,你这人真婆妈\ldots{}\ldots{}这事如果上升到国家层面,除了这两个悲催定律之外,还有一个更让人绝望的:大部分国家借了钱之后通常是越借越多,并且十有八九还不起了。

美国人在一战之后二战之前跟欧洲就是这样的关系,钱越借越多,欧洲人也没什么长进,除了买军火买装备练级,没啥好做的。中国人借钱给美国人?这事早就发生过了。美国人就这时就心下嘀咕:借钱给欧洲,并且还是德国这个手下败将,可贷款只带来他们的厌恶,我们这是图啥?

有意思的事情来了。美国外贸商界的回答是:我们对外放贷确实增加了我们的出口贸易。如果没有贷款,欧洲是无力从我们这里大量购买的。我们的工厂因此正常生产,我们的劳动力也能保住工作------这话听着就很耳熟是吧?

加雷·加勒特就说了:确实如此,但这能持续多久呢?你借钱给国外顾客来买你的商品,你的出口贸易就会一直膨胀,可你卖出商品到底收回了什么?除非你继续借钱让他们来付款或者免除他们的债务,不然你什么都得不到。这能算是好买卖吗?

第二个声音这时就会出现了,我刚看到的时候还以为是托马斯·弗里德曼说的,``但是请记住,现代世界是一个共同体。没有哪一个国家能独享繁荣,美国也不例外。

作为有剩余资源的国家,我们应该要关心战后憔悴的欧洲\ldots{}\ldots{}这一理由足够把美国信贷置于欧洲的控制之下。此外,我们也有义务这么做,我们这么做也是明智的,因为我们不再自私。''

对于这个世界是不是平的,解释权在弗里德曼,但是对于这个世界上欠钱不还这事的解释权,我觉得还是应该交给历史。在这一点上,我觉得他的前辈加雷·加勒特说得更像是那么一回事:贷方如果把自己的借钱给别人当成道义责任,那么借方的合同就不要想太多了,它不可能是一个义务。所以他的结论是:国际借贷最终会导致无度和不负责任,要么借方觉得借给他们钱的人是在掠夺他们,要么就是根本不在乎。

尽管我的标题是``日光之下,并无新事'',但我说的可不是中国人借钱给美国人这回事,虽然这两者听起来挺像的。

那么德国人当时把借来的钱做了什么呢?除了继续买装备练级之外,他们建了当时世界上最好的大桥,修了最好的公共设施,盖了好多房子,稍带着还创造了一种叫包豪斯的艺术风格------要盖多少房子才会产生一种建筑艺术啊?结果你发现德国人不是很爽,跟美国人借了中国人的钱之后的抱怨是一样的:美国人(中国人)在用他们的钱影响我们的政策,他们太有钱了,对世界是一种威胁。

美国的国债评级被调低之后,中国作为最大的债权国也跟着认为这是天底下的好事之一,这是近来最挑战我的智商的事,严重点说,它伤害了我。虽然人均下来我借给美国人的800多美元也不算啥,但也差不多可以让美国人过上半个月寒酸的生活了。

不过,``日光之下,并无新事'',我们的钱被胡乱花掉的太多了,要是为这事愁肠百转,我们恐怕早就死过一万回了。

如果我们愿意看的话,现在的时髦东西都在默默地以``日光之下,并无新事''这种原则行事:你觉得马克·扎克伯格现在很风光?在十几年前另一个叫马克·安德里森,辍学创业搞了一个网景,我到现在也觉得他对互联网事业的改变更深刻一些;你现在看乔布斯推一个新产品就了不得了,在1927年,全美国人都等着亨利·福特宣布他的新车上市;你现在秀一个iPhone4来已经没办法吸引目光了,但福特的A型车在上市半年之后,走在路上还要被围观;你看现在的硅谷每天都有人融资发财,但它产生最多百万富翁的时候是在1970年代。

那个时候所有人都在炒的是半导体的概念,半导体之前是原子能,原子能之前是汽车------比尔·杜兰特是横跨两个时代的炒股做多大师,依据``日光原则'',两次他都输掉了,这是题外话。

如果还要往前的话,就是铁路股票,运河股票------这个时候诞生了标准普尔,普尔先生提出的口号是``投资者有知情权'',于是他开始为所有可以评级的东西评级。还有个大家忘了差不多的泡沫是马车栈道,它就是那个时代的信息高速公路。

再往前,就是我们都知道的南海泡沫了。查尔斯·麦基写过一本着名的书,叫《大癫狂》,它的英文书名叫非同寻常的大众幻想与全民疯狂,说的都是一样的事------人因为贪婪和恐惧而陷入到一种不能自拔的境地当中。没错,就是贪婪和恐惧,跟巴菲特说的那两个词是一样的,你贪婪也好,恐惧也好,最后都会是差不多的一个结果。所以麦基引用了《圣经》里的话来说人类这个社会:一代过去,一代又来,地却永远长存。

已有的事,后必再有;已行的事,后必再行;日光之下,并无新事。

当然,这本书写得太早了,以致于它连铁路股的疯狂都没赶上。如果麦基活得足够长,他会看到在接下来的19世纪、20世纪里发生的更好玩的事,并且不论是哪一个故事,最终都是导向他的那个结论。所以,尽管它写得如此早,但最后还是会出现在所有20世纪的商业必读书目中。写得早没关系,所有事都已经发生过了。

关于前面提到的那些,它们的结尾纷纷是:福特在前两年险些破产,按照保罗·格雷厄姆的说法,美国车之所以不景气,唯一的原因就是产品是烂车;马克·安德里森的网景被微软打败了,卖给了AOL;标准普尔不但给了投资者知情权,而且还编了很多我们要知道的评级来骗大家投资;美国对欧洲的债务?哦,那是我们更熟悉的第二次世界大战。

这真是一个悲观的世界,我可不想现在就去设想现在风光的公司、崛起的大国和人类们的未来。需要补充一点的是,关于国债也不是所有都已经发生过的,比如希腊和西班牙,他们的国债问题就比以前复杂。当年的德国靠胡乱印钞票就解决了一轮债务危机,现在的希腊和西班牙不能,因为钞票主权已经不在它们自己手上了。

这真是悲剧中的悲剧。所以还是说点乐观的吧。1970年代,《商业周刊》发表了一篇文章《股票之死》,报道中说,你有听说过、参加过美国公司的股东大会吗?那里全是老人,死气沉沉,不代表未来。《商业周刊》显然错了,那些死气沉沉的老家伙们个个笃信``日光之下,并无新事'',他们知道萧条之后一定会有一轮股市的反弹,未来的大牛市等着他们呢。

\begin{center}\rule{3in}{0.4pt}\end{center}

\textbf{《世界是平的》}

托马斯·弗里德曼

湖南科学技术出版社 2008年9月

定价:58元

\textbf{《通用汽车的缔造者》}

劳伦斯·古斯廷

上海远东出版社 2008年8月

定价:39元

\textbf{《大癫狂------非同寻常的大众幻想与全民疯狂》}

查尔斯·麦基

北京邮电大学出版社 2009年1月

定价:39元

\section{有一些人值得为他起立鼓掌(上)}

就跟成了精的地产界的专业人士最爱提``地段''一样,事业有成的投资界人士挂在嘴边也有一句话,``投资就是投资团队'',``我看好你的人''。如果这些话只是在风险资本家那里说说,还算是有价值的,后来泛滥到买个股票也要说投资团队,我觉得就要提高警惕了。依我看,上市公司已经这么成熟了,很多时候都是财务总监和华尔街那帮家伙们控制着,左右你的收入回报的与那些个被赞美的团队没有太大关系。并且,团队这回事,即便你是一个田野调查爱好者,你也未必能看明白。比如我们身边的无锡尚德,说起来还真是不错,有一个科学家CEO,有激情,有理想,有政策,有领导重视,新能源,新技术,并且有一段时间还垄断了全球的上游资源,从哪个角度看都是一个有智商有情商的伟大公司胚子。但你看现在\ldots{}\ldots{}当然现在已经沦落为仙股了,你要去投资它,我也不是很反对,你的风险就是它有可能退市和破产。

并且,还有一点要强调,我们如何判断一个团队的好坏,通常是借助于他们既有的成绩,而被赞美或被证实不灵光的团队也都是事后才知道的。比如有一个叫迈克尔·艾斯纳的家伙,他是迪士尼的前主席和CEO,很多人不喜欢他,跟乔布斯的皮克斯过不去,又赶跑了才华横溢的杰弗瑞·卡僧伯格,在那之后的好长一段时间,迪士尼就根本没有出品过什么说得过去的动画片\ldots{}\ldots{}但你要就此判定他是个败家子,那也不对,在他任上迪士尼的资产和利润都涨了十几倍,如果想赚钱的话,你当然会得到一个好回报。但迪士尼的麻烦是在他离任之后才出现的\ldots{}\ldots{}地产界大佬们所说的决定房产价值的三个要素:地段,地段,地段。所以,如果真有什么价值的投资建议的话,即便是针对那些声名赫赫的团队,你更应该说的是:时机,时机,时机\ldots{}\ldots{}

就跟我们以前说过的,如果你投资1997年的苹果,而不是2个月前的苹果,这就是时机的价值。肯恩·格林伍德有一本小说叫《倒带人生》,就是不断时光倒流回转------他靠着炒苹果的股票而赚了大钱,但你要不是打算死上一回,并算准还能回到从前,你就不要做这样的梦了。想一想吧,在1997年那个时候你心里想的会是什么呢?你即使爱乔布斯,但你也顶多是心疼地看着这个前嬉皮士做困兽犹斗状,人们热爱看悲剧,潜意识又难免会有一种代入感------庸常的、昏噩的生活是我们的主旋律,我们都一样。

在1990年代初,有一个叫巴里·迪勒的人突然迷上了苹果的笔记本电脑,这个50岁的家伙就辞掉了工作,像凯鲁亚克一样在美国大陆来了一次穿越,他相信这个笔记本电脑背后的东西将要改变世界。

我在看到这个桥段的时候,有两个感觉。第一个感觉是这家伙眼光很毒辣。要知道,在1990年代,互联网革命的前夜,世界上最强大的``宽带''网络是有线电视网,那些既能做内容又掌握网络的娱乐业大亨们打算成为新时代的娱乐王。第二个感觉就是英雄迟暮,人总是把自己推到一个不能胜任的位置上才罢休------巴里·迪勒可不是简单的厌倦了公司政治和格子间生活的小白领,他是福布斯的老板,他只要对默多克一个人负责就行了。

众所周知的是,电视业策动的这次互联互通革命失败了,败给了网景的游览器和万维网HTTP协议,好莱坞一如既往地回到从前扮演新技术敌人这个角色去了。``本该顺应时局的电影执行官因阻挠每一次技术进步而声名狼籍。他们反对声效、抵制彩色影片,还抨击无线电。现在他们认为电视是恶魔用来毁灭电影业的发明。于是,客厅的电视机里播放着制片厂的宣战声明。他们禁止旗下的明星出现在电视上并拒绝向暴发户似的竞争者提供电影。''理查德·泰德洛说如果好莱坞把自己当成娱乐业而不是电影业,就不会愚蠢地扮演新技术反对者的角色。当互联网出现的时候,这帮把持着版权的家伙,还在对视频网站、搜索公司继续表示着敌意。

当然,好莱坞也有例外,巴里·迪勒就是一个。

这个人没上过大学,他把自己的青春奉献给了威廉·莫里斯公司的文件收发室事业。``当年19岁的我觉得那是一块风水宝地,还能学习业务。我希望从每一件正在发生的事中获益。我会抱着成堆的文件阅读里面的每一个细节。我是说,别人进大学念书,我却在威廉·莫里斯经纪公司学习。我读完了档案室里的所有文件。这用了我3年时间,但我办到了。有人标榜自己在牛津大学读书。我却为自己能在威廉·莫里斯经纪公司读书而骄傲。这听上去有些自命不凡,却是一个准确的类比,因为在我看来威廉·莫里斯经纪公司是世界上最棒的图书馆。其他人常说大家都希望尽快离开收发室。我却拼命留在那里。''

简短一点说,这个收发室小子,在13年后,也就是他32岁的时候,已经是派拉蒙的主席和总裁了。那个时候,他已经在电视业发明了电视电影这种产品(这是好莱坞对电视恨之入骨的开始),还发明了电视连续剧。在派拉蒙之后,他主导乔治·卢卡斯拍了《夺宝奇兵》系列。斯皮尔伯格发布了他的处女作,一部叫Duel的电视电影。他旗下两个大将,一个是我们前面说的败家子迈克尔·艾斯纳,另一个是杰佛瑞·卡僧伯格。后来他搞了一个叫``梦工厂''的东西。

这,对于投资者来说应该是最伟大的组合了吧?但别说普通投资者,就是派拉蒙的母公司------西方与海湾工业公司董事会------也看不出这个团队的价值。在来了一个新CEO之后,这群天才转眼间被解雇的被解雇,辞职的辞职。派拉蒙失去的不仅是华尔街曾批判的十几亿美元的娱乐市场,简直是失去了未来。

然后,艾斯纳和卡僧伯格去了迪士尼,巴里·迪勒投奔默多克,做了福克斯集团的老板。他从电视出来,去了电影界,再从电影界回到了电视业。然后,他事实上打破了美国三大电视网的垄断,为默多克建立了基本上算做第四电视网的福克斯电视网。

神采飞扬洋溢着创业激情的巴里·迪勒在1993年辞职的时候,放弃的就是这个事业。然后一头扎进``有线电视光缆入口的娱乐互联计划'',成为事后被证明失败的那个计划的推动者。这你就明白,为什么我会把巴里·迪勒视为悲剧角色了。

我有点孤陋寡闻,再看到他的消息是因为另一个悲剧。有一位很有名气的女作家,早年间曾写过派拉蒙和好莱坞,在《纽约客》上开创了企业家传记的文风。她叫蒂娜·布朗,后来成为《新闻周刊》的总编辑。然后就是我们现在听说的那个故事,它们将在2012年年底出最后一期印刷版,然后转向网络和电子版。我的一个根深蒂固的观点是,从平面媒体转向新媒体,基本上是宣布死亡,因为商业模式变了,市场、受众和受众习惯完全不同------在2年前,《华盛顿邮报》宣布以一美元出售它,就已经宣告一次它的死亡了\ldots{}\ldots{}慢着,它被卖给了谁?

然后就是突然发现了还活着的巴里·迪勒,现在他70岁。

如果你在1993年的时候跟着他,你会得到什么?我在把蒂娜·布朗和巴里·迪勒联系在一起的时候,首先发现的还是身体里那个``热爱悲剧''的猥琐小市民的自己:你看,悲剧了吧!你看,不行了吧!悲剧和悲剧对对碰了吧?

然后,发现的是另外一个世界。现在的巴里·迪勒跟我们最大的关系是他掌握着一个叫Expedia的公司。你可能不了解它,它是世界上最大的旅游在线服务公司,我们偶尔会用到的艺龙旅行网的母公司;还掌握着一个叫TripADvisor的公司,它在几年前买了酷讯网和到到网;他的另一个公司IAC旗下还有match.com、ask.com等若干个网站。

有的人你值得为他投资,但即使你错过了也没有关系,你还可以随时准备起立,为他鼓掌。

当然,他还没有买下他最热爱的派拉蒙,他还没有回到娱乐业,他一定还觉得没有达到他一生的标杆------大卫·格芬的高度\ldots{}\ldots{}当然,这是另外一个故事了。

\begin{center}\rule{3in}{0.4pt}\end{center}

\textbf{《倒带人生》}

肯恩·格林伍德

译林出版社,2010年11月

定价:25元

\textbf{《巴里·迪勒》}

副标题:美国娱乐业巨亨沉浮录

乔治·梅尔

上海财经大学出版社,2010年5月

定价:36元

\section{有一些人值得为他起立鼓掌(下)}

大卫·格芬最初也是威廉·莫里斯经纪公司一个送文件的小伙子。他不如巴里·迪勒的地方就是,巴里是个高中毕业的好莱坞本地青年,大卫则有一个假文凭------这是一件偶尔会被人提起的事。假文凭在任何时候都是一个污点,即使是大卫·格芬也不能把这个当成特立独行。除了这个文凭问题,大卫·格芬就是处处都比巴里·迪勒强。至少巴里是这么看的。他们从威廉·莫里斯时代就算是至交,但在巴里那里,他一生都是在追赶这个家伙。《洛杉矶杂志》的罗德·鲁里曾经评价过这个事,``受访者都说,迪勒的真正动力是超越格芬------从事业到性生活,乃至握手的力度。''需要说明一下的是,大卫·格芬是个着名的同性恋者。至少从这一点上看,巴里·迪勒到现在也没有机会超越了。

我们回顾巴里·迪勒当年,领着迈克尔·埃斯纳和杰弗里·卡僧伯格创造了派拉蒙的辉煌时代。他们联手将派拉蒙的利润由1973年的3900万美元提升至1983年1.4亿美元,增长了3倍。他们凭借着《夺宝奇兵》等电影将派拉蒙由6大电影制片厂的最后一名推至榜首,在18个月里制作了20部赚钱的电影。``好莱坞的许多人都避免与迪勒-埃斯纳团队合作。但是2人联手的8年中,派拉蒙不仅在电影领域称霸,还凭借《干杯酒吧》和《家族的诞生》等联播视频引领电视网的电视节目。''

杰弗里·卡僧伯格对他的评价就更要高,``我们并不总是清楚他的计划,不过巴里·迪勒在整个职业旅程中一直担负开拓者和梦想家的角色,他总是高瞻远瞩。他建起另一家新电视网的举动不但打破常规,还出人意料。巴里一旦开始行动,就打出一记全垒打。''这说的是他们被派拉蒙扫地出门之后在福克斯的故事,如果你稍微了解一下卡僧伯格的刻薄和自视甚高,如果你再想一下当年他对``派拉蒙三杰''中另一杰埃斯纳的态度,你简直都不能相信这话会出自他的嘴里。

一位与巴里·迪勒并驾齐驱的好莱坞巨亨羡慕地说他,``不容忽视。瞧瞧他这一生干的事。这家伙要主宰世界。他要在传媒界称王。他要比大卫·格芬更有钱、更有权。这个混球既顽固又有才华。''如果我没猜错的话,有资格说这话的人应该就是迈克尔·埃斯纳。

巴里·迪勒做了这么多事,但有一个让人恼火的共同点:他是一个打工的。而他一生追赶的大卫·格芬则只要向自己汇报就可以了。

要是说起大卫·格芬,那也绝对应该是一个为之起立的硬角色。1970年,他成立自己的庇护所唱片公司,推出了杰克逊·布朗、杰妮·米歇尔、汤姆·维茨以及老鹰乐队。但这只是开始,1974年,他发现了鲍勃·迪伦。弗雷德里克·马特尔评价说:``如果没有格芬,这些乐队对于大众来讲,就显得过于硬摇滚或者过于变幻莫测,而格芬的战略就是使它们变得沉静。他的天才之处在于尽可能按照商业化的要求制作音乐,同时并未抹杀酷的元素。相反,他将这种时尚潮流的元素融入音乐之中。''《纽约客》的一位评论家后来写道:最酷的事情是越时尚的音乐越成功。

1975年,他把唱片公司卖给华纳,就退休了。一个人可以随便退休,我觉得这对于巴里·迪勒来说算是一个大刺激,因为他一生总是有点苦哈哈,虽然成就无数,但奋斗也无休无止,看着就感觉累。

格芬在退休的时候32岁,很显然,他总得再找点什么事做。5年之后,他成立了格芬唱片,把约翰·列侬和小野洋子的回归唱片《双重幻想》当成主打,这一次他的运气不是太好,唱片卖得不怎么样\ldots{}\ldots{}但运气这东西还真说不准,列侬在这一年突然被杀死了。在之后的几年里,雪儿、音速青年、贝克、史密斯飞船、彼得·盖布瑞尔、尼尔·扬,当然还有科特·柯本。大卫·格芬让穿着破烂牛仔裤的科特·柯本成了新生代的代言人,一个本来想着只卖20万张的Nevermind卖了1000万。

然后,他开始拍电影,汤姆·克鲁斯的《夜访吸血鬼》,以及马丁·斯科塞斯的《三更半夜》;然后再投资百老汇的音乐剧,《猫》和《寻梦女郎》获得了成功,他把摇滚带入了音乐剧的舞台。在做好这件事之后,再一次把音乐公司卖给美国音乐娱乐集团,就是现在的环球音乐公司。

再然后,就是1994年,大卫·格芬和史蒂夫·斯皮尔伯格、杰弗里·卡僧伯格一起建立了梦工厂电影公司,他同时还创办了一个梦工厂音乐公司。一直到2008年,大卫·格芬第三次退休,卖掉梦工厂的股份,然后住在马里布海滩上从杰克·华纳手里买来的房子里。

``在美国当代文化中,大卫很可能是同时在三大关键娱乐行业(流行音乐、百老汇音乐剧以及好莱坞电影)里步步为营、大获全胜的少数几个人物之一,这是史无前例的。''卡僧伯格这么评价。

大卫·格芬在娱乐业抄了一条离自动提款机最近的路,赚了好多好多钱,并且还塑造了美国的娱乐业,或者美国的主流文化。如果再加上,他是给自己在打工\ldots{}\ldots{}你就知道为什么巴里·迪勒总是有那么一点``既生瑜,何生亮''的耿耿于怀了。

如果我们总结经验,有一点是可以记下来的:不要为了打工皇帝而投资他的老板。除非有一种可能性,这公司好到了``谁来干都坏不到哪里去''的地步。如果要举起例子来,当然还是巴菲特的可口可乐,就是他说的那个护城河。如果盘点巴菲特那些``必需品''公司,当然不止这一个,我感觉巴菲特差不多用了一生的时间在考察IBM到底是不是谁来做都能稳赚不赔------恭喜他,最近他终于有了结论:可以投资。但如果早个40年投资IBM呢?这事儿没有什么好说的,只要有足够长的时间,差不多都会赚个大的,只是你做不到而已。就连巴菲特也做不到。1966年,他成为迪士尼的投资者,持有500万美元股份,在第二年以600万美元卖掉了。在30年后,他做主把汤姆·墨菲的大都会卖给迪士尼的时候,在给股东的信中说,如果不卖的话,那现在怎么说也有几亿美金了。

这是题外话,巴菲特把大都会、NBC卖给了迪士尼,成就了迈克尔·埃斯纳。这是一个普通的投资故事。但我们的巴里·迪勒在那个时候可不是那么好。

50岁的巴里·迪勒决定要超越大卫·格芬和派拉蒙。在他心中,超越格芬,第一步就是自己当老板,第二步就是把派拉蒙再拿回来------这一次可不是去找工作,而是去当它的老板。

通常来说,这种创业动机很难有什么好结果。但好在,巴里·迪勒可不是寻常货色。他给自己安排的路径和入口------我们不要忘了,是那台苹果的笔记本电脑。简单点说,他的复杂的路线图是这样:先搞定一个有线网,然后拿下一个电视购物频道,然后通过这个盈利频道,继而买下派拉蒙,然后再度杀回娱乐业和电视网,然后\ldots{}\ldots{}超越大卫·格芬。

时机。这时机有多重要,他对互联互通的娱乐网的信仰在几年后就借助网景和雅虎成为全民宗教;他对购物这个入口的把握------你现在看看市盈率最高的亚马逊,再看看沃尔玛现在的窘境,看看淘宝、天猫的一万亿,你就知道他的眼光非同凡响。但你知道时机才是决定一切的最重要因素\ldots{}\ldots{}

于是,他开始``销声匿迹''了。但我不大相信他会把大卫·格芬和派拉蒙都扔在了一边。即使他现在财富可能不比大卫·格芬差太多,但他也只是一个买便宜机票、找便宜酒店的平台的老板。你知道他即使在离开这个领域很多年后,在提到派拉蒙的时候还是会说:``派拉蒙在我的人生中意义重大,它对我意味着太多太多。我想起它时,心中只有浪漫的感受。我永远想不起那里的争斗。不过知道争斗确实有过。''

唯一的障碍就是他现在已经70岁了。纵是天才,也敌不过年龄、衰老,这叫自然规律。自然规律最终还是会打败这些金光闪闪的东西和金光闪闪的人物,一想到这些,还是会让人有些伤感。

\begin{center}\rule{3in}{0.4pt}\end{center}

\textbf{《倒带人生》}

肯恩·格林伍德

译林出版社,2010年11月

定价:25元

\textbf{《巴里·迪勒:美国娱乐业巨亨沉浮录》}

乔治·梅尔

上海财经大学出版社,2010年5月

定价:36元

巴里·迪勒的帐篷杆理论吸纳了欧文·塞拉伯格的一项基本宗旨:盈利的公司会选择制作不赚大钱的电影,只要电影可以巩固制片厂的声望。

\section{``加里驾驶小飞机兜风那天''}

英雄不问出处。这句话在我看来有两种解释,一种是不管什么背景出身,都能把自己的那一摊事业做得很好,以至于我们根本不用去看他们的来历到底如何;另外一种就是在人家事成之后,你冷不丁地看到一个伟人杵在那里,怎么看都是英雄自己的能耐,你就忽略了其实他成功背后还有诸多原因。想起这个问题是有一天华老师都跑过来感慨了:你知道伐?巴菲特他老爹是参议员,比尔·盖茨他妈妈是IBM的董事,Balabala\ldots{}\ldots{}你明白我的意思吧?

这就是典型的``英雄不问出处''的第二种情况,你一直在高山仰止地看着这两位轮流登上首富宝座,充满着羡慕和尊敬,直到有一天你开始查他们的来历,原来是官二代啊!原来是富二代啊!这难免让英雄形象打点折扣。当然,更多时候,还是心下宽慰好多,嗯,这辈子做不成首富,也不能全怪自己不努力,爹地不行啊。

但如果平心而论,参议员和大公司董事的儿子还是有很多的,也不是每个人都挤入了首富前两名,我们这里听到看到的大部分例子还基本上是酒后驾车之类的,爹地有的时候反倒是坏事的原因。所以,如果我们不那么苛求的话,我们还是可以拿来做励志的典型。

在励志的版本中,故事可以这样说:巴菲特在华盛顿度过了几年不开心的生活之后,决定还是抛弃掉那些浮云,回到老家奥马哈过平淡无奇的生活,同时做点自己爱做的事,那就是投资。接下来的故事就是他利用他的敏锐的眼光、坚定的投资信念、无与伦比的定力成就了他的首富事业,在他甘于寂寞的30年里,实现了每年增长30\%的这个挑战人类极限的辉煌目标。

比尔·盖茨呢,我们则可以引用马克西姆·格拉德威尔眼中看到的那部分现实,这个富家子弟完全不在意他的同伴们的那些爱好,一门心思扎在学校的计算机房里,苦练编程技术,经过一万小时的磨练------任何人在自己的领域里经过一万小时的训练,都可以成就自己的一番事业------他在计算机的民主化时代到来之前,已经做足了准备,并最终成为民主化的推手之一。他要让每个人桌子上都摆上一台电脑,并且这种最伟大的愿景真的实现了\ldots{}\ldots{}

你看这些成功的励志故事里,基本上不需要爹地出场就可以解决的。当然了,如果换个角度看,可能就未必如此了。

比如说,在30年前你刚开始准备用自己的定力、眼光、信念开始自己的投资生涯的时候,如果你是1万美元起家,那么30年后,你即使同样有巴菲特那样的30\%复利,你也就是有了2600多万美元,离首富远着呢。这就要重新打量一下,在巴菲特还是一个小角色的时候,他为什么会有那么高的起步资金。在复利这个领域里,人们总是更关心年回报率的多少,往往忽略掉开始的那个数字。这里可以额外提个醒,对于在职场里混事的大多数人来说,起薪很重要,如果你开始的起薪比同样岗位同等资历的人少30\%,那么大家一起进步,最后到你将要退休的时候,你的收入可能要比人家少个几百万。这也是为什么大家都更愿意到大城市来工作的原因,从长远看,因为大城市的起薪水平可能远高于二三线城市,最后的总收益也会高出很多,足以抵销掉房子、消费水平的高昂。

巴菲特的人脉资源对他起步阶段肯定会有很多的帮助,如果你把这个当成``官二代''的红利,我觉得也没有什么不对的。相比之下,比尔·盖茨的故事就会复杂一些。

这就要说到题目里的那个人------加里·基尔代尔了。知道这个人的人恐怕不够多,简单点说,他发明了CP/M系统,内含的BIOS就是DOS系统的前身。比尔·盖茨在揽下IBM的操作系统的研发任务之后,在不知道怎么完成任务的情况下,直接抄袭了过来,这就是着名的MS-DOS。对加里·基尔代尔的标准的评价是:他编写出第一个微机操作系统程序语言和第一个软盘驱动程序,后来电脑里分为ABC盘也始于他。关键是他做这一切的时候,微机还没有诞生,比苹果电脑要早好几个月,比第一台IBM兼容机要早数年。他还是``家酿俱乐部''的发起者和创始人。看过《乔布斯传》的人差不多还会记得,史蒂夫·沃兹尼亚克的早期启蒙都来自于这个极客小组织。关于``驾小飞机兜风那天''则出现在比尔·盖茨的传奇故事当中。在成功的了不起的盖茨的版本里,当IBM来跟加里谈判他的操作系统的时候,这位贪玩的大学老师当天开着自己的小飞机兜风去了,于是错过了人类历史上最重要的一次谈判。在加里·基尔代尔的版本里,就是跟IBM谈得虽然算不上很好,但也达成了购买他的系统的协议\ldots{}\ldots{}

接下来的故事,就是IBM推出了彻底改变PC业面貌的兼容机,与它们的PC搭配出售的操作系统,是和240美元一套的CP/M系统一模一样抄袭来的但只卖40美元的MS-DOS系统。战争还没开始就结束了。

最恨比尔·盖茨的人不是史蒂夫·乔布斯,而应该是加里·基尔代尔。在关于比尔·盖茨的商业传奇里,他更执着,更相信操作系统是PC业的核心,更具备商业头脑,这些评价都很准确,并且显然做过飞行员的大学老师且热衷于当老师的加里不是他的对手,我们还可以相信比尔·盖茨有更强的公关手段可以让IBM绑定了自己的小公司。噢,我想起他的妈妈了,我斗胆猜测一下,这里多多少少还是会有一点关系的。

之后故事就更简单了。微软打败了要打败IBM的苹果和乔布斯,加里·基尔代尔虽然还接着成为图形操作界面的发明人(GUI),但与比尔比起来就只能算湮没无闻了。了解到他的人可能更多的只是惋惜和同情,实际上他也是这么同情自己的,结果就是自艾自怜。后来他离了婚,酗了酒,在他52岁的时候,在自己家中很有可能是喝了酒之后摔倒,三天之后就英年早逝了。

但你知道了这个事又会怎么样呢?你不用微软的产品,但在鼎盛时期还是有90\%以上的人会用,尤其是你还要用网银的时候。英雄不问出处的结果就是这样,尤其是连``加里驾驶小飞机兜风那天''都快要成为商业经典时刻的时候,一个``知识英雄''就足以让历史定论发生一种偏差,我觉得这是商业社会里最庸俗的一面。当然,人类把``成王败寇''之类的话挂在嘴边,这可能说明人类社会总体上比起商业社会来说,也没好到哪里去。

好的一面是,每个人桌子上都摆上了一个电脑,虽然Windows操作系统也不是那么好用,但它足够让我们用了,因为桌子上的这个电脑,整个世界也跟着变化了。我们虽然觉得``成王败寇''是一种很让人不齿的想法,但如果这个东西算得上是一种正向的革命的话,由比尔·盖茨来指挥还是要好过由加里·基尔代尔来指挥。因为在比尔·盖茨那里,他确实更相信这东西会让人类的生活变得不一样起来,并且坚定地实现了它。

\begin{center}\rule{3in}{0.4pt}\end{center}

\textbf{《美国创新史》}

《独立宣言》只是1776年美国历史上的三大事件之一。同年3月8日,瓦特揭开了第一台商业冷凝蒸汽机模型的面纱,几天后,亚当·斯密出版了《国富论》。

哈罗德·埃文斯 /盖尔·巴克兰 / 戴维·列菲

中信出版社,2011年4月

定价:498元

\textbf{《沃伦·巴菲特传:一个美国资本家的成长》}

罗杰·洛文斯坦

海南出版社,2007年

定价: 39.8元

\textbf{《异类:不一样的成功启示录》}

马尔科姆·格拉德威尔

中信出版社,2009年6月

定价: 26元

\section{你是智慧多一点,还是钱多一点}

很多时候,我还是很庆幸自己不是一个投资者------这句话说得可能不是那么严谨,严格说来每个人都是投资者,好在我是特指那些以投资为主要收入来源并且主要是在各种资本市场上的投资者------当我不是一个投资者的时候,作为一个商业爱好者,我只需要关心这个公司到底是怎么回事就行了,我还能发现这里有很多有趣的东西。但如果我是一个投资者,我就要成天警惕地看着四周别人如何看待这个公司,不但担心那些公司的经营者弄了假账来骗你的投资,还要想着分析师们是否拿了上市公司的红利或者是一个跟自己一样没有什么经验的蠢货,当然资深一点的还要分析市场热情和理性程度,说白了就是猜市场上每个其他投资者都在想什么------要知道,博傻不是发生在市场大起大落时候的击鼓传花,而是随时都得琢磨着哪个眼拙的人把好东西很便宜地卖给你,或者哪个笨蛋把你手里的烫山芋给接过去。

我看过一本书,叫《别对我撒谎》。我在看到一半的时候,有一个迷思:一个人如果成天琢磨着别人是不是骗自己,这事比较累;如果还要想办法去证明对方骗了自己,要斗智斗勇------这本书说一个人每天要撒150次谎------我觉得至少有三分之二就是100次是为了要使计策证明其他人是不是骗自己------这就不仅仅是累了,基本上相当于悲剧了。你看,在投资领域里,如果看大家在做什么,基本上讲的就是这样的悲剧故事。

所以,真正的投资者,理论上都不能算是冲着好公司来投资。如果大家都看好一个公司,那么二级市场就没有什么意义,都留在手里如何进行下一步的交易呢?所以史上最着名的投资公司,长期资本管理公司的天才们说:``我们大部分交易都是为别人提供流动性。''一点没错,天才就是能看到本质。我们还可以为这句话做个注解,同样来自于长期资本管理公司。这群天才数学家们失败的原因在于,当他们充分发现概率的价值和利用无与伦比的数学模型找到了足够多的市场定价错误之后,事情并不会向着他们所希望的那个方向发展。换句话说,天才们可能发现了好公司,并且认定它上涨的概率超过20\%,但市场并不是以好公司为标准来运行的。

``事后罗杰·洛温斯坦为他们总结的是:金融市场出现极端值的概率,要比发生`百年一遇'的暴风雨大得多,长期资本管理公司后来把它们的失败归咎于此。随机事件最关键的条件在于,在掷硬币时,所有的投掷都是独立的,硬币不会记得前3次掷出的是人头还是字,对于下一次而言,其概率依然是50对50。但市场是有记忆力的。有时候,某种趋势之所以得以继续的原因,就在于交易员们希望(或害怕)这种趋势持续下去。有时候,投资者们会像奴隶一样跟从这种趋势,只是因为他们认为,其他人也会像自己一样行事,而绝没有其他原因。这种交易动力与评估股价的逻辑毫无关系,也与高效率市场上的理性投资者不相合拍。但这就是人性。当市场出现三次投掷错误时,第四次投掷一定不可能完全是随机的了。''

于是我们看到的是,一些交易员或者因为蒙受损失而抛售,另一些可能就会因为害怕而落井下石。哪里有人性存在,就会有各种``黑色星期一''存在。

看到了吧,他们最后的成败得失都跟一个公司的价值没有什么关系的,你还看到了,他们居然还用了``人性''这个词。投资者们在这个领域里要随时发现人性的存在,并且把它当做一个重要的变量,我说的悲剧差不多就是这样,它跟每天严密监察撒谎者动向的那帮家伙们一样,这不是一个让人高兴的事。

当然,这容易推导出一个可能是更悲剧的结论,一个纯真的人很难在这个市场里占到分毫便宜,我可没说反之会怎么样,所以我不会说那些赚了大钱的人就不纯真。在我看来,沃伦·巴菲特相对来说就算得上纯真了,除了他买纳贝斯克那一次。而比这个还要悲剧的则是如果纯真正直的人进入市场,他很有可能会成为一个笑话。举个我上一次曾经讲过的例子,如果你在1980年代进入市场,你会看到有加里·基尔代尔的公司,也有比尔·盖茨的公司,你觉得比尔·盖茨在关于操作系统的这件事上基本上表现得像个恶棍,于是你的纯真让你相信恶有恶报,你的正直让你决定力挺加里,然后\ldots{}\ldots{}当然就没有然后了。

这个世界里为你准备的又纯真又正直又理性又赚钱的机会,悲观地说基本上没有。同样在1980年代,可能只有苹果公司有那么一点可能性,当然,还要有一个前提,你不是史蒂夫·沃兹尼亚克的粉丝,最好都不知道有这个人存在,否则你可能还是会觉得史蒂夫·乔布斯是一个欺世盗名的家伙。

好在,这个世界根本就没打算让纯真正直者赚大钱。因为你打量下来,没有哪个公司会让人觉得安心,洛克菲勒那一代搞垄断巧取豪夺就不用说了,知识英雄们,尤其是哈佛大学那两个着名的辍学者都有横刀夺爱据为己有的嫌疑,原来看着有理想有操守智商高还为人类解决知识传承的谷歌,自从野心蓬勃起来,也有一点把持不住自己\ldots{}\ldots{}只要一投资,就会对价值观有冲击,并且在这条路上只能越走越远。

前面说到的长期资本管理公司,本来是利用市场错误定价来获得收益,那靠的是做数学模型的能力,他们更会用电脑的能力以及他们强大的推广自己的能力,但到最后,他们比的就是看好明年某只股票能涨20\%的眼光,这里面的技术含量也不能说没有,但我们知道它更合适的名字叫赌博,更直白一点,就是押大押小。虽然任何时候都不能小瞧赌博,但你要知道长期资本管理公司可是有几个货真价实的诺贝尔经济学奖的获得者的。他们做这事有一点对不起诺贝尔当年辛辛苦苦赚到的那些卖军火的钱。

这里还有一个洋洋自得的小故事,其中一个得了奖的经济学家叫马尔隆·斯科尔斯,有一次美林证券一位负责外汇交易的人问他:你是智慧多一点呢?还是钱更多一点?他很真诚地说:``当然是智慧了!不过,钱也越来越多了!''

还是前面说到的巴菲特买纳贝斯克,之所以它会成为一个纯真正直的人的人生污点,就是因为在那场空前的并购案中,巴菲特也参与进来,并且成为一家有烟草业务的公司的股东。对于一个反烟者来说,这差不多就是一个丑闻了。当然,在我们这里,这还算不上,但你要知道的是曾经有几个家里有农场的医生,在那几亩薄地里种了些烟草------这在美国就已经是天大的丑闻了------医生在做图财害命的事。纯真正直还不抽烟的巴菲特如何做这事的,当然是因为利益,一旦这种事发生,接下来再买高盛,也就可以理解了,虽然他一向看不起华尔街那些唯利是图的投资银行。

关于洋洋自得的经济学家,最后钱没有越来越多,智慧也跟着被人质疑,当然还有获诺贝尔经济学奖的经济学家的声誉也打了折扣。不过,最后他们还是跟着长期资本管理公司一起被美联储解救了一回。纯真正直的人们一直对这事百思不解:这帮坏人,明明是一个私营公司,明明只是做了一个对冲基金,又不像银行那样涉及国计民生,政府为什么要主导出资救助?这世界究竟还有没有一点底线?你看,当他们为这些苦恼的时候,看起来既不像智慧多多,更不像财富多多,纯真正直总是要很受伤。

\begin{center}\rule{3in}{0.4pt}\end{center}

\textbf{《别对我撒谎:23篇震撼世界的新闻调查报道》}

即使有一天,网络作为媒介取代了传统媒体,记者的正义感、勇气、思想、采访技巧、文字表达能力仍是这个领域最重要的资源。

约翰·皮尔格

华东师范大学出版社 2010年3月

定价: 68.00元

\textbf{《拯救华尔街:长期资本管理公司的崛起与陨落》}

如果说,约翰·麦利威瑟在所罗门兄弟公司得到了什么秘诀的话,这个秘诀就是:千万不要放弃你的损失,直到它转变成为你的盈利。

罗杰·洛温斯坦

广东经济出版社 2009年3月

定价: 39.80元

\section{当范德比尔特最后一次经过码头(上)}

你知道什么才能叫高富帅吗?自从这个粗鄙的词流行起来之后,在我的脑海中就始终有一个人跟这个词牢牢地捆绑在了一起。按照哈佛商学院教授理查德·泰德罗的说法,他的一生就是``一个成功接着另一个成功''。

从上学开始,他不仅成绩优异,而且还是戴着光环的成绩优异------他不是简单地从高中毕了业,而是毕业典礼上的致词者;他也不是从大学毕了业就了事,而是获得了美国优秀大学生的称号(我还真不知道美国也有这个评选);他不是随便挑了一所学校做研究,而是一定要在麻省理工这样排名第一的工学院里搞到一个自然科学的博士学位。然后,他毕了业,去了当时还不错的飞歌公司,这当然不能算是一个好的选择,于是------罗伯特·诺伊斯决定跟随当年最有号召力的从贝尔实验室辞了职创业的物理学家威廉·肖克利,他刚刚作为晶体管的发明人而让世界瞩目。然后有一天,他从费城飞到硅谷------那个时候还不叫硅谷,硅谷是因为他才被叫起来的------在到达的当天上午,先找中介买了一个房子,然后下午去找肖克利面试\ldots{}\ldots{}泰德罗也忍不住说:注意一下这个顺序!好吧,这个顺序可以秒杀所有自信心爆棚的人(这里插播另外一个故事,在另一个关于杰夫·贝佐斯的传说当中,当这个人决定远离东海岸的金融事业而到西部创业的时候,他找了一家搬家公司,告诉司机师傅们,你们往西开,搬到哪里我还没想好,这也是一个有趣的故事,只是后来贝佐斯也觉得这事跟实际出入太大)。

总之,罗伯特·诺伊斯成功地获得了肖克利实验室的工作。

接下来的故事就是大家都知道的,因为不满新科诺贝尔物理学奖获得者肖克利的无厘头管理方式,他和另外七个人一起叛逃成立了另外一家公司,这就是仙童半导体,这后来被称为硅谷摇篮;隔了几年,他又自己成立一家公司,叫做英特尔,有了英特尔,硅谷最终才得以被如此命名,所以罗伯特·诺伊斯是唯一那个可以被叫做``硅谷之父''的人。

不仅仅是泰德罗,所有人都会遗憾的一件事是,他62岁的时候就英年早逝了,作为一枚高富帅,我觉得他的人生后来就有一点空虚。如果一直那么高歌猛进下去,我觉得他也不会这么早就死掉,如果他再多活10年,他一定会和德州仪器的杰克·基尔比共同获得诺贝尔物理学奖,因为他们分头在1958年发明了集成电路。

我们还没说他能歌善舞、有专业的赛级滑雪水平以及还是一个熟练的飞行员和潜水员。高富帅嘛,样样不能落下,他要活在现代中国一定还可以参加中国好声音。

这人生,看起来多少让人兴致勃勃。我们从小到大接受过的各种苦逼教育中,一帆风顺一定导致经不起挫折、小时了了大时未必、吃得苦中苦方为人上人之类的陈词滥调充斥了我们的生活,但我们还是得承认这世界上有一种高富帅,他们很让我们嫉妒,他们存在的意义就是让我们的生活黯然失色\ldots{}\ldots{}好吧,这样的价值观是不对的,好在我们可以投资他们,某种意义上我还可以算做他们的老板,这样就皆大欢喜了。

在各种创业并且大获成功的企业家中,这种中产阶层背景的企业创始人是我觉得最靠谱的。首先他们不是非创业不可,所以他们多少都具备了一点创业者必备的那种冲动;而他们受到的那种中产阶级的严格教育又总是会让他们练就出某种技术优势或者独特的竞争能力;如果他们再有一点经营管理的天赋------只要稍高于芸芸众生一点就足够了,在我看来这个最没有门槛了,那么他们的公司总是会有一点亮色的。

虽然他们最靠谱,罗伯特·诺伊斯也是我最喜欢的几个企业家之一,但他们这种背景不是我最喜欢的。在我暗自给他们的分类里,另外两种背景,我们可以简单定义为草根和富二代。关于富二代,我说的是那些继承了家族生意的那些富二代,他们最重要的意义在于让你忘了``嫉妒''这两个字,然后换成另外两个难以名状的叫``家族''的字,这世界对他们来说,就是``天下是老子的''\ldots{}\ldots{}当这些人主导一个公司的时候,你首先要关心的就是这个家伙到底是不是个败家子,以及他们是不是准备好了去管理,以及你什么时候准备逃出来------不管你是投资的,还是打工的。

但这个通常也很难判断,比如小托马斯·沃森,小时候被老爹的威严吓出了抑郁症;上中学的时候,在回家路上走着走着,一想到要继承IBM这个庞然大物,就号啕大哭起来,作为一个不争气的儿子的典型而让所有人担惊受怕;直到二战之后,他才焕发出另外一种人生------你就想像一下迈克·柯里昂,阿尔·帕西诺塑造的那个年轻一代教父,他如何扞卫了家族荣誉,如何重振了黑帮事业,那么发生在IBM的故事与之相比就相差无几,甚至连小沃森如何对待他的亲兄弟的感觉都如出一辙。但在老教父没有遇到黑帮火拼之前,迈克看起来就像个文艺青年。

还有一款富二代,是我们这些穷鬼比较讨厌的,就是会摆出那种一副不可一世的世家子的样子,也没有什么家族产业好继承,但就是有钱,也不愿意做小事,所以就总以看起来像是眼高手低的样子筹划一些大手笔的事的那些人\ldots{}\ldots{}当然,也不是所有故事都是以败家来收场的。

有一个叫胡安·特里普的家伙,他爸爸是华尔街的交易员,妈妈是做房产中介的,积累了好多财富。

他跑到耶鲁上了学,同学都是亨利·卢斯、布里特·哈登这样的人物,一起玩的发小儿则是范德比尔特、洛克菲勒家的那些后代们。在他们上学的时候,参加第一次世界大战是每个耶鲁的爱国青年最热衷的一件事,他就决定加入海军,但视力不及格,于是他爹就只好找了当海军部副部长的富兰克林·罗斯福来打通关系\ldots{}\ldots{}

一战之后特里普从学校毕了业,自动把自己调到富家子胡乱做公司的频道里,买了七架旧飞机,做起了航空公司的生意,虽然那个时候航空公司也不是什么了不得的大生意。但搞定这些事的牌照和资金还是要下一点工夫的,好在他倒不缺投资,那几个发小儿都凑了钱给他。然后破产了,然后再做一个,竞争输掉了,然后再做,最后就成了泛美航空,当年全球最大的航空公司。

这公子哥一直到65岁的时候还是同样的德性,他跟波音公司的老总比尔·艾伦想弄一种新的宽机身喷气机,大小是707的两倍,能运载400名乘客------这在当时来说完全是一种疯狂的想法。

``一个想法便自然而然地在他们之间被交流起来。''后来他们之间的那段经典对话是这样的。特里普告诉艾伦,``如果你敢造出来,我就买下它。''而艾伦回敬说:``如果你敢买,我就敢造。''然后就有了波音747。

好吧,这些喜欢吹牛,热爱各种新奇玩意,喜欢飙个车开个飞机之类,从小就跟这个世界一点也不见外的家伙,很多时候都是很讨厌的。但你要知道他们如果更用心一点,他从小受到的训练和视野对于他们掌管一个企业没有太大的坏处,很有可能还是一个好事。

小沃森让IBM进入到计算机时代,就是富二代的机会和中产阶层受教育的敏锐性结合在一起的一个结果。而且,作为一个投资者,如果你有耐心,富二代看着不那么让人放心,你还可以等富三代------当然这就是另外一个企业中兴的话题了。

不过,在这三种企业创始人中,他们还都不是最丰富多彩最让人感觉卓尔不凡的,真正的厉害角色当然还是那些白手起家的真正的混蛋们。

\begin{center}\rule{3in}{0.4pt}\end{center}

\textbf{《美国创新史》}

哈罗德·埃文斯 盖尔·巴克兰 戴维·列菲

中信出版社 2011年4月

定价: 498元

\textbf{《影响历史的商业七巨头》}

理查德·S·泰德罗

机械工业出版社 2003年2月

定价:33元

\textbf{《硅谷巨人》}

汤姆·珀金斯

中国青年出版社 2008年8月

定价: 36元

\section{当范德比尔特最后一次经过码头(下)}

在上一篇中我们说过了两种企业创始人。一种是以罗伯特·诺伊斯为代表的中产阶级高富帅,另外一种是富二代到富N代,他们的代表是胡安·特里普和小托马斯·沃森以及所有非亨利·福特一世的所有福特家的继承人们。

第三种就是我们马上要说到的那些很草根的企业创始人。混蛋们终于可以出场了。这里要从政治上正确的角度解释一下,草根与混蛋之间不是划等号的关系,混蛋也未必全然是骂人话,你也可以理解成是特立独行、不拘一格、不按牌理出牌之类的不同凡响。也正是因为这样的一些元素,所以草根总是更容易被拿来作为最好的励志的素材。如果说让我在三种企业创始人中选一个作为最值得推荐的群体,那么显然我会倾向于这一群。想一想我们从小受到的教育,王侯将相宁有种乎,人家能做到为什么你不能之类的,他们也具有天然的教科书气质。

如果要更好地看到这一坨人的价值,看到丰富多彩的好玩的寓教于乐的人生,我觉得还是应该把眼光放远一点,我们可以把19世纪那些草莽气息浓厚的家伙们翻出来,看看草根是如何不同凡响的。

约翰·洛克菲勒,穷小子,他爸爸是个卖假药的,当然,他对于洛克菲勒来说也没有什么太恶劣的影响,因为他也不怎么回家。洛克菲勒16岁的时候找到一份做簿记员的工作,就是记账的,然后开始做农产品贸易,然后在发现石油之后,他果断地认为这是一个大事,成为其中一员,然后他又敏锐地发现采原油是个周期性的行业,不利于稳健经营,炼油才是最好的生意,于是他就开始了他的标准石油公司\ldots{}\ldots{}在几十年后,美国为他立了一个反托拉斯的法,他的公司被拆成34家,但这并不妨碍在美国历代所有富豪的大排行榜中,他总是轻松排到第一位,因为他的个人财富大约值4000亿美元。

科尼利尔斯·范德比尔特,穷小子,种地的,有一个无所事事的老爹,所以他想到要做一些什么事的时候,首先想到的还是他的母亲。他的职业也起步在16岁,但他的起点要更高一些,他认为美国贸易将会越来越重要,内河航运是个大事,于是他从他妈妈那里借了100美元买了条船。几十年以后,他作为航运大王和铁路大王而成为美国首富,他死后8年,他的儿子让他的家族财富达到了2亿美元,按占美国GDP的比例看,相当于2005年的2180多亿美元,而当年首富比尔·盖茨不过只有510亿美元。

安德鲁·卡内基,穷小子,老爹是个loser,一事无成,被他和他的妈妈鄙视得离家出走,他事业起步在13岁,一周可以赚1.2美元,是一个酷爱记账并且很早掌握了复式记账法的聪明孩子,他本来是跟着范德比尔特的铁路事业起家的,但转而认为造铁路钢轨才是更好的赚钱方式,在1868年的时候,他又果断发现何止是钢轨,美国人会进入消费时代,钢铁注定要改变人类文明的物质基础,能以最低价格提供最高质量的钢的人,将成为(一个为金钱而发疯的世界上)最富有的人,最伟大的商人。几十年之后,当J.P.摩根祝贺他成为美国首富的时候,他拥有3亿美元。

我还是要做一个说明,钱从来不是励志的全部,虽然衡量这些草根的成功人生最好的方式就是钱------尤其是相比于贵族、富二代来说,更是这样。你一定已经注意到了,除了钱,还有一个失败的老爹,严厉的老妈,那些豪情万丈的企业家们通常来说更喜欢他们的妈妈,这是另外一个有趣的话题,我还提到了他们都喜欢记账,这也是另外的话题,它们都不是我在这里要说的。我最想说的是,他们其实跟我们现在谈论起的某个不世出的天才企业家一样,他们给自己的事业找了个基础性的定位才是根本。

当然,你也不能因为我举了19世纪的三个例子,就会有一种误解,那些划时代意义的市场空白早就被这帮家伙瓜分掉了,20世纪山姆·沃尔顿还曾经改变零售业,21世纪史蒂夫·乔布斯还重新定义了手机,这些都可以让他们成为最伟大的企业家。关于乔布斯还可以多说一句的是,虽然他差不多过着中产的生活,还从最昂贵的私立学校之一的里德学院退学,但作为一个汽修工的义子,基本上还是可以把他视为草根出身。只是20世纪的草根们不会再像一个世纪之前的那些前辈们经历那么苦的童年生活。

而除了上面说的那几个让他们敏锐发现大机会的桥段之外,这些19世纪的巨头们的成功与今天的这些创业成功者也没有什么两样。

比如技术永远是最重要的,在航运公司时代,最高时速20英里,但消耗的木料只是同等规模蒸汽船的一半,提高燃料利用率是取胜关键。范德比尔特除了在竞争中打打杀杀,他还能把船造得更快,这是技术的胜利;安德鲁·卡内基也一样,他嘲笑那些同行们,``我们太傻了,我们的对手还是比我们更傻,我们已经雇用化学家指导我们的生产多年了,而我们的对手却说他们雇不起化学家。他们不能没有化学家来指导他们。''洛克菲勒对于炼油技术的掌握也同样是致胜关键。

当然这还不是全部,洛克菲勒更懂得集约化生产的价值,在大企业还没有一种公认的成熟的管理经验的时候,他已经开始实践他的托拉斯生产方式了------垄断是什么,对于企业主来说,最大的价值还是体现在效率上,在他身后的一百多年里,所有的管理围绕的核心就是效率问题。安德鲁·卡内基就懂得``价格的低廉和生产的规模是成正比的。生产10吨钢的平均成本要比生产100吨高出几倍\ldots{}\ldots{}因此,生产规模越大,成本就越低。''当然,这些都比不上范德比尔特,作为年长四十来岁的前辈,他对价格战的把握和控制,对如何建立市场壁垒的敏感,让他很轻松地就在航运和铁路两个领域里建立起自己的垄断地位------最值得一说的是,他的另一部分名声还建立在他是一个打破专营权垄断的斗士。

在这里,我还可以加上另外一个声名不如他们显赫但在恶棍领域里排名更靠前的杰伊·古尔德,人们都记得他是一个做资本运作的投机分子,但是在传记作家小爱德华·勒内汗笔下,``在两年的时间里,杰伊完成的事情相当于在19世纪创办了一个苹果公司,并利用这个砝码,促使其与IBM联合。''而这也让他在美国工业舞台两个最重要的领域都成了主角:交通和通信。

如果作为一个励志的故事,这些还不足以表达榜样的无穷力量。沃伦·巴菲特说,在商界获得成功需要三种力量:脑力、精力和性格。他补充说,没有最后一条,前两条会置你于死地。性格的一部分是对自己的了解,知道你是谁、你对他人有什么样的影响力、他人怎样看你。好吧,这个说起来就太复杂了,执着、精明、敏锐这些都可以列进来,我们前面说过的爱妈妈、会记账也可以列进来------为什么它们也会很关键?这就更复杂了。那可能又是另外一篇了。

最后要说的一个花絮是,不管是这些无法无天的草莽英雄,还是高富帅们,或者是富二代(简直是一定的),这些家伙都有那么一点色胆包天,好学生罗伯特·诺伊斯的绰号就叫做``Rapid'',那可不是什么好话,但所有这些人都比不上范德比尔特来得传奇。在他去世的时候,有人这样描述:纽约湾一小段航行中,拖船和汽笛鸣笛致敬,帆船降半旗表示哀悼。

另一个刻薄的人的说法:当载着范德比尔特棺材的船只路过码头,渐渐远去时,他看到许多码头上的妓女们屈膝致敬,目送着她们最忠实的顾客永远离开了曼哈顿。

在1980年代以前的中国内地未授权翻译的图书的中文版前言中通常都会有一段话,``应该指出,作者的观点受制于错误的理论指导,读者应该以马克思主义为武器,辩证看待其中的糟粕''------虽然我很喜欢那个刻薄的促狭鬼的描述,但不代表我认可范德比尔特这个混蛋的人生观。

\begin{center}\rule{3in}{0.4pt}\end{center}

\textbf{《美国创新史》}

哈罗德·埃文斯 盖尔·巴克兰 戴维·列菲

中信出版社 2011年4月

定价: 498元

\textbf{《影响历史的商业七巨头》}

理查德·S·泰德罗

机械工业出版社 2003年2月

定价:33元

\textbf{《硅谷巨人》}

汤姆·珀金斯

中国青年出版社 2008年8月

定价: 36元

\section{然后,他就开始沮丧起来}

前阵子听说一个令人沮丧的故事。几个媒体老总结伴驱车来到北京怀柔三岔村,他们的目的是看一看彼得·海斯勒生活和战斗过的地方,这个中文名叫何伟的人以村民魏子淇为主人公写了一本叫《寻路中国》的书。

这本书确实很好看,这没什么可让人沮丧的,沮丧的点在于,这些前辈人情练达,如果要了解中国和中国人,您哪怕就是成天在微博上挂着,也一样可以找到你的魏子淇,更何况发现中国的本质------那来自于另一位不在场的精英感慨,海斯勒发现了中国的本质,``这就是中国啊''------本来就是老几位的本分,不至于按图索骥跑到三岔村来探访神迹,按钱钟书的说法,属于吃了好吃的鸡蛋之后不但要见一下母鸡,而且连公鸡踩蛋处也要凭吊一下。

有关这个事的另一层思考是,为什么我们没有去写魏子淇?我猜想很多铁肩担道义妙笔着文章的前辈同行们,在提起如椽大笔的时候\ldots{}\ldots{}会想我有纷繁的世界、伟大的事迹、复杂的人格要展现,那么魏子淇的意义在哪里呢?然后\ldots{}\ldots{}这多少有点像我们总说到的投资,彼得·海斯勒更像个价值投资者,做长线,仔细安排投资组合,慢慢积累,享受复利的好处------直到有一天,发现这就是中国,这就是一次高回报的投资。

而我们运气太好了,身处荡气回肠的吊诡市场里,情商决定了我们不缺各种真假消息,智商决定了我们不分真假概念\ldots{}\ldots{}快进快出,短线多刺激!结果么,就如所有的短线投资一样,最终还是输给长线。但这也不妨碍我们对价值投资的顶礼膜拜不是?我们比谁都在意跟沃伦·巴菲特去吃个午餐,我们也更想到奥马哈去朝个圣,就像来到三岔村探访神迹。

好吧,虽然硬扯到了投资上,但我本意也不在此。我是想说,如果把中国看成一个公司或一只股票的话,``财经记者''彼得·海斯勒做得不错,而我们的大部分人基本上做得是不及格的。这就要说到正题上,对于个人投资者来说,大部分信息来自于媒体的报道,中国市场之所以显得很扑朔迷离,媒体对公司云山雾罩的报道可能也要负不少责任------我们很难提供更准确的对公司的报道,就像彼得·海斯勒可能下工夫写好了一个叫中国的公司,而我们则只在门口东看看西看看,顺道感慨一下彼得·海斯勒的行云流水,但却总是高山仰止------因为我们似乎没想着那个稳定的长久的收益,我们不过是画个公司的皮毛而已。

为什么会如此,我觉得也很简单。比如,我看过彼得·海斯勒写中国,我还看过另一个叫西达尔塔·德布的印度人写印度的企业家:

``在我第一次和阿林丹姆碰面时就有人告诉过我,说他是我们这个时代的弄潮儿。他集华丽、勇气、财富于一身,如果说他的个性使这些特点为世人所看到,倒不如说这些品质早就像带电粒子一般存在于当代印度的空气中。随着我越来越了解他,我觉得还有另外一个方面让他成为这个时代的代表人物------他的财富其实是建立在印度中产阶级的渴望和愤慨之上的。没有渴望者的仰望、模仿、钦佩和他们所贡献出来的财富,像阿林丹姆这样的大亨就不会存在。他利用市场经济及其政治伙伴和印度右翼分子带给学员们的渴望、莽撞和不安全感做生意。阿林丹姆非常了解这个社会给予这些渴望者怎样的一种坚定信念,他也非常了解这些渴望者所饱含的那种与真实感受到的挑衅所不相称的愤怒。这是我们这个年代的胜利:让渴望者们既感受到力量,又感觉到被排挤,并且让全世界的人见证新兴小资产阶级急于表达他们的受挫感,对自己未能成为精英而愤愤不平;然后当面对那些真正的贫困时,他们却异常冷漠,甚至麻木不仁。''

我在这段西达尔塔·德布描写印度的文字中,看到:印度企业家如何赚钱;印度企业家对社会是什么样的影响;他的用户(可能也是粉丝)是如何消费他的产品;中产阶级如何蠢蠢欲动以及他们贫瘠的心灵(而这也通常是我们自身最鲜明但却试图抹去的标签)\ldots{}\ldots{}如果你有兴趣看德布的《美丽与诅咒》,你会进一步发现更丰富的世界。某种意义上说,我觉得他可能比冷艳高贵的彼得·海斯勒写得还要好一点,海斯勒更多是对田园不再的纠结,那个当然也很中国,但它应该不是那些跑到中国来做生意的美国人所迫切需要了解的中国,但西达尔塔·德布给我们看的要准确得多------如果你同样把印度看成一个可以投资的公司的话,这个``财经记者''可能还要略胜一筹。

然后这两个家伙一起秒杀我们所有的``财经''记者。有一句流传甚久的话叫,``有一千个观众,就有一千个哈姆雷特'',但在我们这里则是``有一千个财经记者,也只能写出一个马云''。因为大家都是等着好运气临头得到一次十分钟到几个小时不等的召见,然后找出一些以前没听过的``金句''如获至宝,区别只是时间的多少,拣到金句的多少。然后,很有可能会从这位企业家身上找到商业的真谛,并且爱上这个朝气蓬勃的商业世界。

乔·诺塞拉在《德州月刊》采访石油商人布恩·皮肯斯的时候,发现了商业的价值远大于类似于休斯顿选举市长之类的新闻。事隔多年他回忆,``在过去的25年间,我变得越来越坚信商业的重要性------这与我刚踏上这条职业道路时的认识可谓天壤之别。商业同政治、医药、宗教、法律以及其他任何一种驱动美国社会前进的巨大动力一样重要。它能带来巨大的危害,也能带来巨大的益处。商业的原动力就是追求利润最大化这一事实并没有让我像很多人那样从骨子里就怀疑它。追求利润有时会促使商人干糟糕的事,但是,它同样可以促使商人干一些极好的、天翻地覆的事情来。''

我觉得他说得没错,所以他没有因为爱上商业而爱上企业家,虽然他那时可以自由出入皮肯斯的任何会议和私人空间,而我们的记者一旦得到这样的机会,多半就以另外一种方式爱商业去了------对内是``起居注'',对外是发言人\ldots{}\ldots{}而另外一小半可能就是太熟悉了,以至于丧失敏感到无睹,既不海斯勒,也很难德布------我一直觉得对于中国和印度这两个发展程度差不了多少的国家,印度会因为有德布这样敏感的记者而更有优势------所谓敏感,某种意义上是媒体和记者的敏感,他们可能会让我们看到更真实的东西:不管是中国的马云还是印度的阿林丹姆,企业家都很有表演天赋,所以你得知道表演背后是什么东西。如果你很有爱,那么即使你有好的操守和向善之心,我们也只能评价他是个善良的好人,可在这一行里并不是很需要这种``善良''。

所以回过头来看,我们会发现彼得·海斯勒根本不能算是一个记者,虽然他来自《纽约客》。我记忆深刻的一段话来自于他的另一本书《江城》。他在涪陵街头与一个擦鞋匠吵架。他博得了大家的同情,沮丧的擦鞋匠说涪陵不需要这样的外国人,海斯勒则声称中国也不应该有这样的中国人。然后他就开始沮丧起来,觉得这样与一个人争执的无谓,``也许涪陵人真的不需要这样的外国人呢。然而,在一定程度上,也是他们助长了这样的外国人。不管怎么说,我们都是同病之人。''

没有一个记者会关心这个事,横冲直撞的记者们怎么会有这种沮丧,哪怕他是一个绅士,文笔优美、为人宽厚。所以最终的结论可能就是,彼得·海斯勒是一个爱中国的作家,西达尔塔·德布是一个爱印度的记者,乔·诺塞拉是一个爱商业的记者。我们呢,可能就只好做一个爱企业家的发言人\ldots{}\ldots{}然后,我也就开始沮丧起来。

\begin{center}\rule{3in}{0.4pt}\end{center}

\textbf{《寻路中国:从乡村到工厂的自驾之旅》}

彼得·海斯勒

上海译文出版社 2011年1月

定价: 33元

\textbf{《江城》}

彼得·海斯勒

上海译文出版社 2012年1月

定价:36元

\textbf{《美丽与诅咒:全球化时代的印度》}

西达尔塔·德布

中信出版社 2012年5月

定价: 38元

\section{潘多拉的盒子是什么时候打开的(上)}

话说在资本市场的田园牧歌时代,人们买一只股票的目的可能只是希望在每年的某个固定时间里,能够得到一次分红。你能想象得到当时的对话可能是``爷爷当年入股的Town上最好的磨坊分红不如去年了'',也有可能是``马丁退出的那100股我们要不要接手?一年至少能带回50美元红利。''

后来,就大萧条了,再后来\ldots{}\ldots{}也不知道是什么时候,你会发现市价比和每股盈余成为投资者最关注的指标,或许你可以理解成,市盈率这个更代表未来收益预期的数字更能表现供求关系的变化,你可以理解成,价格比价值更容易成为你投资的原因。

当时间到了2007年的时候,你可能不知道你投资的公司到底是做什么的,甚至你都不知道自己是在投资什么,还有一种可能是连卖东西的人也不知道自己在卖什么。

我一点也不觉得这事有点夸张。2007年次贷危机刚出来那会儿,对于隔岸观火的我们来说最痛苦的一件事莫过于,弄不懂什么叫次贷。

在几年之后,这世界上有无数本书写次贷和次贷是怎么回事。我见过的最好的版本来自于查尔斯·塞费的《数字是靠不住的》。他用我们稍微熟悉一点的保险来解释次贷:假如火灾的风险是每万户中可能会发生40次火灾,每次火灾的赔偿额是2万美元,那么整体上会有80万美元的赔偿额发生;如果你有1万个客户,每个客户收100美元保费,那么你的收入是100万,当你付出80万美元之后,你有20万美元的毛利。

但是如果你觉得赔偿起来太麻烦,可以把1万个保单捆绑销售给一个更大规模的再保险公司,你就不用为赔付这些事操心,当然你也要让一部分利给再保险公司,比如你们各自分了10万美元。这差不多是个皆大欢喜的结局。

问题就在这里了,这20万元利润的前提在于风险------风险永远是我们在离开田园牧歌的资本市场之后最先要考虑的问题------的准确把控上,但如果1万用户是发生60起火灾呢?那就要赔付120万美元。对于你的保险公司来说,因为已经转让出去,所以毫发无损,而对于再保险公司来说,就要亏损了。

这个时候风险就出现了,40起火灾是建立在你对客户的细致选择的基础上,一个独居的喜欢躺在床上抽烟的老头儿,你是不会把他当成你的客户的。但是如果你知道未来你会把保单整体打包卖给大公司,那你心里可就放松多了,因为这些风险反正又不会成为你的,所以你就把这些不那么优质的客户一起卖给了再保险大公司。

然后,你就会知道那些原本要花200美元买一份保险的高风险客户,现在只要花100美元就可以有保障了\ldots{}\ldots{}

如果你把火灾保险替换成房屋贷款,就知道为什么美国人会突然``疯''掉了。唯一的不同在于,把房子卖给那些根本就不可能还上贷款的穷人,只需要有一个信念就行:房价永远在上涨。

这个房价永远上涨的传说在我们这里也是存在的,与美国稍有不同的是,在一个70\%的人有自有产权房屋并且已经实现了城市化的国家里,希望房价上涨的人永远是大多数。一个市场里如果所有人都盼望一个同样的好消息,那效果与好消息真的发生没有什么不一样,所以它一定所向披靡,无可阻挡。

如果说在2007年的时候,有一个比``次贷''更能显示出得我们的无知的一个词存在的话,那一定是这个叫``信用违约掉期''的家伙。直到2012年,我每一次看到它都跟一个失忆症患者面对一个热情的老朋友一样,啊,是你,你到底是谁来着?并且像每个失忆症患者一样相信下次再见到它的时候就会认识了\ldots{}\ldots{}

关于这个万恶之源,还是贝萨尼·麦克莱恩和乔·诺塞拉最终治愈了我的失忆症。他们的解释可以这么理解:所有的问题都出在风险和风险的定价上;你不愿意承担风险的时候,很有可能有别人愿意承担风险,那你就卖给他好了。

你看,就是这个东西。它这里牵扯到的最核心的一个东西就是,风险到底应该值多少钱。理论上买和卖的双方都有数学模型,如果有那么一个完美地综合了这个市场里所有因素的模型,那么其结论应该是一样的\ldots{}\ldots{}当然,这些确实只是发生在理论上。真操作起来的时候,其实满不是那么回事,但我们可以确认的是,数学模型这种东西出现之后,我们就更大胆了。先知先觉的人比如约翰·保尔森开始操作他的对冲基金,并且拉上几个像高盛这样没有操守的公司来一起做局------在我看来,这基本上跟行骗没有什么两样;更多的人,他们跟我们前面说的那些大多数一样:只要房价上涨,还不上贷款就不是一个问题;或者只要不超过比如20\%的违约率,就平安无事;或者只要AAA一直存在,就不会有任何风险\ldots{}\ldots{}并且,以上所有这些,都有天才数学家们的模型支持。

如果这些还不够,那么人类还有其他共识存在:``房价从大萧条以来一直处于上升中,没有经历过全国性的下降'',``没有哪个金融机构想要毁了自己'',``次贷对房地产市场的影响是有限的,不会带来太大的连带影响'',更重要的是``出现这种结果之前是一个漫长的过程,所以人们很容易就相信要摆脱它也需要一个漫长的过程。''

数学家本来是跟华尔街没有任何关系的,拜电脑时代的来临,所有数据都可以得到处理,而对处理的数据如何解读,A数据和B数据是什么关系,人类就开始进入到一个想法大过理性的时代。也就是我们这个标题所说的,潘多拉的盒子在这个时候打开了。

而这几乎一律来自于我们最优秀的那部分头脑,和最美好的希望。比如他们最初出现的时候,那应该是在20多年前了,刘易斯·拉涅利带领所罗门兄弟投资公司进入到一个伟大的未知市场里,而那个时候他们手中玩的扑克牌,与20几年之后的是一样的:美国人的住房按揭贷款,或者说美国人的自有产权房屋的社会和个人理想,或者干脆就是``美国梦''本身。而刘易斯·拉涅利所仰仗的,也是那些初出茅庐的数学家们。

然后是美林证券,号称是为美国中产阶级带来证券福音的投资银行,那是个革命者的形象。然后是组合保险------又一个跟电脑和数学模型有关的。并且是又一个跟``躲避风险''的故事相关的发明\ldots{}\ldots{}然后,人类或者说部分人类就进入到了一种独特风险的时代。他们就像某个电影里我们看到的搞笑桥段一样------你有了一件可以防御子弹的神秘紧身衣------然后,你闪亮登场了,然后,大胆地开始了实验,然后,咣地一声,死了。

人在武装起来的时候,相信自己所向披靡。天才发明家们看着这个悲剧,他们想的不会是回到田园牧歌的时代,而是想我要做一件更好的紧身衣。

于是,我们看到的后来30年的历史,就是一个防弹紧身衣不断推陈出新的时代。他们不会想到的是,让这个悲剧不断发生的原因并不是人类永不妥协永不服输的进取精神,而是贪婪。

那应该是1980年代,一个让我们乐观开心,雄心勃勃的1980年代。

\begin{center}\rule{3in}{0.4pt}\end{center}

\textbf{《金融往事:恐慌、危机和迟来的复苏》}

迈克尔·刘易斯

中信出版社 2011年5月

定价:49元

\textbf{《众魔在人间:华尔街的风云传奇》}

贝萨尼·麦克莱恩 / 乔·诺塞拉

中信出版社 2011年12月

定价:68元

\textbf{《数字是靠不住的》}

查尔斯·塞费

中信出版社 2011年9月

定价:42元

\section{潘多拉的盒子是什么时候打开的(下)}

1980年代真是让人迷恋啊。

1980年代,美国选出了一个会打星球大战的演员出身的总统罗纳德·里根,不仅打赢了冷战,还让美国人重新树立起了信心。华尔街吸引了一群聪明人去做事;当年被肯尼迪总统登月计划鼓舞的年轻的理工科生们也顺利毕业,他们开始进入到实业领域,并且创造了更多石破天惊的好想法;更早一批受益于``曼哈顿计划''的信息科学家们开始了真正的信息革命,个人电脑已经遂了比尔·盖茨的愿正在摆上每个人的书桌。于是蓬勃的1980年代就应运而生了。

与这个让人兴奋的历史场景不那么匹配的是,事后的灾难也已经开始孕育,潘多拉的盒子正在打开。从某种意义上说,这盒子从来没有被关上过,只是出于杠杆效应,它既有可能挑动地球,也有可能让贪婪以倍数增长。

查理·梅里尔在1914年创建了后来成为美国最大经纪行的美林证券公司,他除了是一个天才的投资者,敏锐地发现一些为中产阶级服务的公司并从中赚取大量财富以外,更重要的还是在于他还把中产阶级也拉到了他的经纪行中,``让华尔街搬进小市镇'',让``勤俭节约的人的适量积蓄''进入到股市当中------如果我们换一种更好听的说法,他是一个投资民主化的推动者,在美林证券,你很有可能就跟那些白鞋蓝血一样享受到资本带来的人生况味。

而美林和其他的投资民主化的推动者所依赖的工具,当然就是普及的个人电脑,每个人都可以有更好的路径对数据进行分析;与此同时,迈克尔·布隆伯格开始了创业,他是帮助美林实现梦想的人,他推出的彭博社可不再是传统意义上的通讯社或者信息服务提供者,价值不菲的资本市场终端机成为让那些即使与华尔街相隔千里的人也可以第一手地享受这些信息的工具。

在迈克尔·刘易斯看来,计算机带来的信息民主化让更多的人进入到资本市场当中。这一点跟乔·诺塞拉所说的是一样的,``依我看,这次牛市最吸引人的地方就是,它第一次让众多的普通美国人把钱投入其中。股市,这个长期以来属于富人的专有领地,正在被民主化,''他从他自己的行为中看到了这一点,``1970年代末,我同许多美国人一样,把自己的钱从一个有管制的银行储蓄账户,转移到了一个全新、未受管制的货币市场基金上------在那个通货膨胀率很高的年代,这可是我们可做的唯一明智之举。''

当然,也不是所有人都这么看。比他们早一点的约翰·布鲁克斯在1980年代到来之前充满了对未知世界的恐惧。在他眼里,1970年代或许还是一个草莽英雄们可以自由驰骋的年代,但在1980年之后,则是机构在左右一切。让乔·诺塞拉引以为自豪的规避通货膨胀风险的新的共同基金的账户,其操作并非是由所有者诺塞拉本人完成的,而是由那些你从来没见过但却可能具有广泛声誉的某个基金经理。在约翰·布鲁克斯笔下,英雄的个人时代在1980年代到来之前就已经结束了。

从后来实际发生的情况来看,约翰·布鲁克斯可能应该说是完败,个人英雄们并没有绝迹,有天才的迈克尔·米尔肯发明的垃圾债券,一样可以让MCI这样的小公司得以挑战AT\&T的权威;也一样有默多克这样的家伙横扫整个英语国家的媒体,建立起新闻集团这个新帝国,这从另外一个角度证明了约翰·布鲁克斯过于悲观了,实际情况是每个人都有了投资机会,并且可赚了不少钱,直到1987年10月14日。

但约翰·布鲁克斯说对了一件事:随着旧世界的崩溃,每个人都有了投资机会,尤其值得一提的是每个人都有了悲剧的机会。

1987年那次危机,让我们所特别喜欢的美好的东西的集合体,经受了一次完败。

你觉得投资民主化是人类进步的表现之一?没错,你在共同基金中的那些钱,被号称层层保护的那些钱会被你所聘的专业机构投资者给吞噬。

你觉得``组合保险''通过复杂而且精巧的模型可以让你规避掉所有违背``鸡蛋不要放在一个篮子里''的原则所产生的风险?在市场正常推进的时候,它的确如此,但在市场崩溃时,它不会考虑你的这一揽子投资的实际价值。

而另一个美好的新事物------计算机,带来了信息传递的速度和自动反馈机制,配合你的模型预测,共同落井下石,一起把市场从``非理性''转变为``疯狂且非理性''。

威廉姆·戴维德记录了这个世界是如何在美好原则的指导下崩溃的:这是一个高度互联的市场,一系列事故或意外是引起金融危机爆发的原因。``组合保险''本身是一个反馈机制和风险管理工具,一旦价格下跌,电脑软件会自动将客户的资金抽离股市。当时很多机构投资者都采用这种交易策略(没有电脑的帮助很难做到这一点),于是市场恶化之时,这个传染以及以更快的速度传染的机制被率先启动了,``在全电脑的交易环境中,金融传染仅用短短几天的时间,就扩散到了整个世界市场。''聪明的电脑也没有能力推导出,原因究竟是美国的逆差高于预期促进了市场繁荣的积极力量,还是美国国会宣布政府正在认真考虑取消对公司营业额的税收优惠政策,最终的结果就是这些因素加剧了企业对经济的担忧,降低了企业兼并的收益和动力,股票需求又大幅降低\ldots{}\ldots{}

威廉姆·戴维德将之称为``过度互联'',要知道那还是在1987年,其后信息更加泛滥,人类也更适应了电脑的思维和反馈机制,然后世界就变得更加草率了。

潘多拉的盒子一经打开,就不那么可控了。我对一个悲伤的故事始终不能释怀:一个叫丹·罗伯森的人在15年后,也就是2002年7月22日发现,投资在互联网股票上的钱在不到2年的时间里从145.7万美元变成了46.8万美元,也就是说他损失了将近100万美元。这个绝望的可怜人写下了一段话:我觉得自己就像是雨中的洛杉矶高速公路上的一条狗。一辆经过的汽车碰伤了这条狗,它在雨中一瘸一拐地蹒跚,而汽车依旧穿梭而过。于是,这条狗停下脚步,看看迎面疾驰而来的汽车,脸上露出一丝苦笑,似乎在想,``我不在乎你撞了我,也不在乎你会不会管我,我仅知道的,就是我不能再跑了,我再也不能跑了。''我对自己说:``这就是我自己------丹·罗伯森。''

这听起来真是太让人感伤了。而这一切,是从1987年开始的。迈克尔·刘易斯、乔·诺塞拉或者约翰·布鲁克斯以及所有人可能说得都对,大资本大机构建立的市场,信息民主化带来的对冲基金,以及不动产和资本市场的衍生品,最终让资本市场走到了今天。但在1987年之前的时候,没有哪个机构投资者会相信最后市场会变得如此无趣。更重要的是,在1987年之前,雄心勃勃的人哪里知道会有丹·罗伯森这样的结果。

\begin{center}\rule{3in}{0.4pt}\end{center}

\textbf{《众魔在人间:华尔街的风云传奇》}

贝萨尼·麦克莱恩 / 乔·诺塞拉

中信出版社 2011年12月

定价:68元

\textbf{《沸腾的岁月》}

{[}美{]} 约翰·布鲁克斯

中信出版社 2006年10月

定价:35元

\textbf{《过度互联:互联网的奇迹与威胁》}

威廉姆·戴维德

中信出版社 2012年8月

定价:36元

\section{底特律到底发生了什么(上)}

我一直觉得有一个人类永恒的问题被我们低估了,就是乐观和悲观的问题。我们在大多数面临选择和判断的时候,实际上就是在选择做一个乐观者还是悲观者。

底特律从1950年代到现在,人口从180多万下降到7
7万,全世界都用悲悯的目光看着这个城市,用病、崩溃、衰落来形容它,甚至要为它做个尸检\ldots{}\ldots{}但爱德华·格莱泽不一样,在全世界都悲观的时候他选择做一个乐观者,他的观点是,你们要多想想留下的那77万人,他们为什么不走?

如果你看过那个着名的叫Detropia的记录片,你脑子里会浮现出几个慷慨陈辞的家伙,他们无助但也是语气无比坚定地说:``上帝让我们生在这里,我们不走!''在美国如果你提上帝,这基本上就相当于结案陈词,大家只有鼓掌的份儿,你没办法再跟他讲,如果上帝,咳咳,他如果是个价值投资者,那他可不建议你留在这里。爱德华·格莱泽虽然指出了正确的思考方向,但在他那本《城市的胜利》的书中,我也没看出来乐观者究竟应该给底特律规划什么样的未来。

底特律到底发生了什么?先是各种人都移民离开了,然后几年前是通用汽车和克莱斯勒宣告破产,在最衰的2009年,失业率达到了25\%,比当时已经糟透了的美国整体还要高9个百分点,最能让我们眼前一黑的是,那里的一个独栋,据说只要1000美元就能买下来。我猜想,有幸看到那部着名的也有可能是史上最烂的记录片的人,看到美国拆迁队拆房子的时候,一定比看到拆我们的胡同四合院还要心疼。

底特律的市长叫大卫·宾恩,估计他应该是有一个偶像鲁迪·朱利安尼,这人在1980年代的时候作为纽约地方检察官,干掉了名声赫赫的垃圾债券大王迈克尔·米尔肯,抓住了一群做内幕交易的华尔街高智商交易员;在1990年代的时候他是纽约市长,在这个时候他以降低纽约的犯罪率而知名,他相信``破窗''理论,就是如果一个街区里有一个窗子破了,那么余下的好窗户会更容易破掉,于是整个街区可能就完蛋了。于是大卫·宾恩决定拆掉破房子。

市长让城市变得好看一点,肯定是一个正确的事,但那些便宜到要死的房子就值得抄底,那可不一定。我们总是把投资定义成买个房子或者买些有价证券,但实际上人生里最大的一笔投资还是你在哪里生活和工作,投胎比投资这事要重要得多,虽然在最开始的时候你没有什么选择权,但当投资不靠谱的时候,我们有一件事总是可以做的,那就是止损。

对于底特律的公民来说,搬家肯定是一个好选择。对于以单一经济为主导的城市或者说是资源性城市,当行业不景气或者没落或者资源枯竭的时候,你搬走是最好的选择。即使对于一个衰落中的城市来说,挽救并且复兴也并非是唯一的解决办法,放弃一个城市也未必就是最坏的选择。

判断一个城市是不是值得投资,或者是否应该做出止损的决定,首先当然还是得看资源和行业前景问题。底特律当初发展起来所依赖的并非是汽车,而是五大湖的水系和迅速建立起来的运河和铁路枢纽地位,在1907年的时候,它的货物吞吐量达到6700万吨,是伦敦和纽约的三倍。接下来的交通优势又让它成为汽车业的中心,所以它又见证了大工业美国的兴起。但问题是交通优势随着制造业转移和交流交通的更便利化,实际上是逐渐削弱的。当制造业的最大成本变成人工成本之后,南卡罗来纳的竞争力就超过了密歇根,当然,还有中国。好,从这个标准看,底特律复兴------如果是靠汽车复兴的话,有点难度。

第二个标准是看人。一个城市是不是值得投资看的是未来发展,而不管怎么发展,人总是最重要的,关于人的好坏可不是拍胸脯证明自己是上帝的子民,而是这些人的受教育水平。格莱泽提供了一个办法来衡量教育水准:看60年前的大学教育水平,如果某个城市在1940年时只有不到5\%的成年人持有大学文凭,那么这个地区在2000年时有大学文凭的成年人不会超过19\%。如果这么看的话,大卫·宾恩如果想学朱利安尼,这个硬件就比不上。格莱泽认为,如果我们要为一个城市的振兴做点事花点钱的话,这钱给人比盖房子要重要得多。

第三个就是这里的人员构成。一个城市的创造力中,人民的年龄很重要,太老了肯定不好,所以年轻人要够多,但年轻人也未必都好。托马斯·索维尔在《美国种族简史》中专门讲了第二代移民的问题:爱尔兰的第二代移民让纽约进入到最黑暗的坦慕尼协会时代;意大利的第二代移民则卖起了私酒以及搞起了黑手党;黑人的第二代移民则让纽约进入到``破窗''时期。底特律的第二代黑人移民在1979年的时候搞过大暴动,这是它走下坡路的开始和标志性事件,并且看起来也远没有真正结束。

这就是不应该投资底特律,以及底特律本地人也应该跑路的原因所在。当然,这在中国是不大能想象的一件事。在中国一是城市户口还有那么一点稀缺性,二是城市公共服务和资源的配置还有足够的价值,三是城市都没了,市长怎么安排呢?还有那么多局级处级领导呢。

这其中的第二条我觉得可以多说两句。因为即使我们一个城市的资源枯竭了,或者行业衰落了也没有关系,市长可以继续卖地,制造新的繁荣。这有一点像是\ldots{}\ldots{}庞氏骗局。

政府把地卖给开发商,开发商再卖给投资者,投资者再炒下去,这个庞氏链条看起来十分清晰,但因为我们根本不觉得自己是最傻的那个,所以我们才不会相信这事真的会发生。比如大同有一个``造城市长''------这种外号即使是安在豪斯曼伯爵身上,我都觉得是一种唐突,但在我们这里,似乎还有赞美的成分。这位市长要调到太原当市长,大同公民们就急了,觉得这么好的市长可不要走,他走了之后盖了一半的楼和执行了一半的规划怎么办。其实这就跟庞氏骗局很像了,在庞氏骗局接近尾声的时候,受害者是最坚定的骗局支持者:如果有警察想逮走骗子,所有受害者都会跟你急------他要走了,我们就彻底输了。所以把他留在这里(继续行骗)才是受害者的最好选择。

这也不新鲜,就像你被一只有欺诈行为的股票套牢的时候,你是受害者,但你衷心希望这个公司能东山再起。

你住在底特律------被几家大公司所左右的底特律,然后你又在其中一个大汽车公司里工作,虽然有工会保护着你,如果你的401K又是跟公司股票挂钩的话,那基本上就相当于买了一只股票,这只股票决定着你的工作、你的退休和你的投资。你把所有鸡蛋都放在这个篮子里,所有,还包括你的家庭。

\begin{center}\rule{3in}{0.4pt}\end{center}

\textbf{《美国种族简史》}

{[}美{]} 托马斯·索威尔

中信出版社 2011年11月

定价:35元

\textbf{《城市的胜利》}

{[}美{]} 爱德华·格莱泽

上海社会科学院出版社 2012年12月

定价:49.80元

\section{那些数学大师们}

小时候对伽利略的印象一般,一是我们所有的书里都说他最后屈服于教廷压力而作了伪证,当时我满脑子想的都是不成功便成仁,他怎么可以这样;二是他很无聊,居然想起来跑到比萨斜塔上去扔铁球。

尼尔·波斯曼为他多少正了一点名。他说那个实验不是伽利略做的,是亚里士多德的粉丝、他的对手乔齐奥·科雷索做的,本来是想证明他的偶像的说法有多正确,结果被伽利略逮了个正着。我看到这个消息很高兴,大师么就应该是这样,与那些凡夫俗子较劲,没有品味,再说,较得过来吗?

我看尼尔·波斯曼的这本《技术垄断》本来是因为最近看有关技术的书太多了,担心被技术控们洗了脑,想寻一点``独眼龙''们(波特曼这种反技术的人这么自况,我也不知道为什么)的观点来对冲一下。但如你所见,最后我记住的还是这些。

有一个问题一直萦绕在我脑海中:那些大师们,比如伽利略和牛顿,他们谁更厉害------看了《说唐》,难免要比下李元霸们的武功,看这些大师当然也是一样。按照波斯曼的说法,显然伽利略更厉害一些,因为伽利略那一辈人基本上搞掉了中世纪,这是皇家造币局长依萨克·牛顿爵士无论如何比不上的。在波斯曼看来,三个新发明终结了中世纪,机械钟表、印刷机和望远镜。钟表本来是给教士们提供服务的,用它每天按点来跟上帝交流,这样上帝不会被吵到;印刷机这东西发明不到五十年,人类知识就爆炸了,虽然大部分印的都是《圣经》,但也一样让人类进入到知识的新境界,就连史上最大既得利益者马丁·路德自己都被惊到了,他哪里知道会有这么大的影响力;第三个望远镜就被挂在伽利略的名下(虽然它真正的发明者至今还无定论),在他手里,望远镜被用来发现了一个新世界,宇宙是这样的!同时也毁灭了一个旧世界:上帝在哪儿呢?所以虽然牛顿有三大定律,但相比于眯起一只眼睛的伽利略来说,后者杀伤力更大。

虽然我跟尼尔·波斯曼观点相近,但我得承认我不大在意天文学上那点事,木星有没有卫星,它都在那里。我对伽利略刮目相看并耿耿于怀的原因在于:他是怎么想起来有光速这回事的?这是无中生有的功夫。

当然,在大师界,他们比起数学家来也显得出身不够高贵。按保罗·格雷厄姆的说法,全世界的人都嫉妒数学家,科学界的每个人暗地里都相信数学家比自己聪明,所以他们都一律要把自己的学问披上数学的外衣,这样仿佛自己就顶级聪明起来了。

关于数学家的聪明,你可以在网上搜索一个Flash,正十七边形的画法,我每次看到它,都会觉得很悲观,看到了吗?它比想光为什么有速度这事还要匪夷所思是吧?如果我们多看一眼,还会知道这是数学家高斯在18岁时搞出来的\ldots{}\ldots{}数学家让人崩溃主要是他们不管不顾地在十七八岁时就搞得不可超越的样子,作为一个死文科生,这简直就是一个未知异次元空间。

还有一个欧拉,乍一看这个人不是很厉害:一辈子做各种数学题,还写了很厚很厚的书,听起来像个公务员。但他做数学题可不像我们,简单点说吧,如果我们的大脑是486电脑的话,那么他的大脑至少有20个core,如果他的大脑叫大脑的话,我们的大脑就应该叫鹅卵石。比如他随手写下一个公式,eπi+1=0,惊天地泣鬼神------允许我这个死文科生看图说话地解释一下:e是自然对数的底(不要问我什么是自然对数),π是圆周率------这是人类最重要的两个无理数之一;i是虚数单位,1是自然数单位,还有一个印度人对人类最伟大的贡献:0。它们完美地形成一个等式。所以你就知道他的运算能力有多厉害了。有一天,他从椅子上滑下来,那个时候他的大脑中那20几个core还在飞快地运算,算的结果很不乐观,他说:``我要死了。''果然就真的死了。

每次看到数学家的故事,我都会很怀疑进化论,人类的智力水平似乎并不是线性向前的。我们的鹅卵石究竟是怎么退化出来的,这还是一个谜。

好在数学这东西跟我们的生活没啥关系。有一个叫迪克舍曼的中学数学老师说说:``数学相当于思维上的举重训练。对于大多数人而言,它是通向某一终点的方法,它本身并不是终点。''再问他这东西有什么用,我们什么时候才会用到,他就会很激动地说:``永远不,你们永远用不到这个。''这种解答会让我们稍微放一点心。

当然,这种说法也并不对。我们这个时代数学家并没有全部退化消亡,貌似他们还有很多大展拳脚的地方。比如,有投资银行或者什么对冲基金的研究部。终于说到我们的本行了------数学家还在做投资。

比如,我们都知道有一个特别有名的公司叫高盛,有一本书叫《高盛帝国》,这是我最近这几年看的书中最无趣的一本,没有什么别的原因,主要是我对这个公司的价值观十分不解:纠集一群最聪明的数学家,然后做研究,发明各种模型,然后发现市场中的漏洞,然后发现赚钱的地方狠捞一票,他们管这个部门叫套利部门,那些穿着西装的数学家,单个拿出来我觉得发现人性和市场中的种种漏洞发个小财或者大财都是一件很牛的事,但如果一个公司把这个当成愿景一样的东西,我宁可还是做一个死文科生吧。

另外一个经典机构是长期投资基金公司,他们创办之时,同样召集了诺贝尔的经济学家、退了休的财政部领导(美国的)、新锐的数学博士之类,他们号称是智力密度最高的一个机构,他们做的事、玩的模型我也一样不甚了了,但总之他们发现了市场的一些秘密,然后狠赚了几年钱。

不管是高盛,还是长期投资。在我头脑当中,我对这些硕果仅存的数学家们的印象就是阿基米德一样:他们躺在浴缸里苦思冥想,突然跳了出来,我找到啦我找到啦我找到啦!阿基米德找到的是浮力定律,而这帮家伙们找到的是市场上的那些定价错误的傻瓜,终于让他们发现了市场中的机会。

当然最后他们也会一丝不挂的。高盛就不用多说了。长期投资这群知识分子忍不住要洋洋得意地讲他们的模型,直到市场上都觉得这东西不错------当所有人都用一个模型来玩对冲基金的时候,接下来会发生什么?没有人跟他们对冲了,市场取向只有一个,只有卖家没有买家,全世界最聪明的头脑也没有办法。

按照以前我们提到过的《刺耳的繁荣》中的说法,数学除了用来毁灭一个中学生的自信心之外没有什么用处。但不管是中学数学老师迪克舍曼,还是20core的欧拉,还是牛顿莱布尼茨,或者伽利略,他们都不知道:数学家们在今天,不但摧毁了我们的自信心,还险些用他们的模型摧毁了我们的生活。

没准真的就摧毁了,这事还没完呢。

\begin{center}\rule{3in}{0.4pt}\end{center}

\textbf{《数学那些事儿:思想、发现、人物和历史》}

威廉·邓纳姆

人民邮电出版社 2011年3月

定价:39元

\textbf{《高盛帝国》(上下)}

查尔斯·埃利斯

中信出版社 2010年1月

定价:80元

三个新发明终结了中世纪,机械钟表、印刷机和望远镜。

\textbf{《技术垄断》}

尼尔·波斯曼

北京大学出版社 2007年10月

定价:22元

\section{底特律到底发生了什么(下)}

上一次我们在讨论底特律时说到,当一个城市中几大公司占据主导地位时,这个城市的居民基本上相当于把工作、退休和投资的所有钱都投注在一个股票之上。所有
的事都围绕着这个股票的衰荣来决定你的生活是什么样子的,底特律就把人逼到这个绝路之上,我觉得这是投资界最悲催的一件事。如果还要挑比这个更悲催的,那
就是中国的单一经济的城市,因为它不但把前面几个占全了,而且还有他爹他妈,他的孩子,以及他能借到钱或者资助他的所有家族成员。

上次说到这里我们的话题就停止了,因为我的同事正在底特律进行着采访,现在她带回来了最新消息:底特律正在复苏,有一些事情做得不错。一个叫丹·吉尔伯特的人试图让人们回到内城来工作和生活,给这个城市增加一些除了汽车之外的东西。

不过,说老实话,我还是觉得没有什么太好的东西可学。就像我们前面所讲过的,衡量一个城市的价值不是建筑,不是花园,而是伟大的人民。伟大的人民之所以伟大,那还是要有活力才行。我觉得活力这东西既不是朱利安尼和丹·吉尔伯特,也不是大卫·宾恩所能解决的。

杰弗里·韦斯特有一个观点我觉得很好,他的疑问是:为什么公司总是有生老病死,而城市却可以生生不息?亚当·拉辛斯基在引用他的疑问时说:``一家公司是从
创业开始的。它一开始会毫无头绪,手忙脚乱,然后会经历困难时期。创业之初,公司正在寻找新法则,没有欠债的担忧。规模在50人之下时,公司会有很多自发
的行动。当发展到50人至100人时,如果公司还存在的话,就会开始出现S曲线了。在这一阶段,公司需要官僚机构、人力资源、服务人员等,公司会变得越来
越官僚主义。''

这段话是说公司本身的局限性,公司的活力不如城市,因为公司缺乏``革新''的阶段。一座城市能够容忍各种疯狂的人生存下去,但没有一家公司能容忍这一点。
``和城市不一样的是,公司会变得无法忍受新想法,对反对意见更是敷衍了事。当一家公司开始精简裁员时,它就不再风光了。''问题有两个,一个是,当城市被公
司绑架的时候,城市的活力是否如``通常''那样让人有信心;第二个是,对于城市来说,``容忍各种疯狂的人生存下去'',说的都是什么样的人?

先说第二个,我们通常把那些异端或者疯狂的人理解为有破坏力的人,但往往忽略了其实在很多情况下,有一部分精英实际上也在扮演着异端或者疯狂的角色,尤其
是在一个很Low的城市里的时候,这些人,你可以把他们称为知识分子,也可以看成是异见领袖。但如果这个城市里只有另外一个维度上疯疯癫癫的家伙(这个我
觉得在底特律是不缺的),而对未来有着更多思考、有创造力的人却都拒绝``投资''这个城市,那么它还有竞争力吗?同样按照爱德华·格莱泽的观点,人类的黑暗
时期大多与城市的衰落有关,比如中世纪是众所周知的人类最悲剧的时代,其中一个相辅相承的因素就是那也是欧洲城市陷于低迷的时期------整个欧洲超过5万人的
城市只有四个,一个是罗马帝国的遗存君士坦丁堡,另外三个塞维利亚、帕勒莫和科尔多瓦,都属于伊斯兰世界。

而底特律衰落的同时也意味着知识和文明的受挫,意味着城市创造力的不足,那些``令人疯狂的''的想法可能也丧失了。

这很可能与三个汽车业巨头有关,公司过于强势,它的效率和它的影响力会让这个城市按照它的节奏来运转。

这也是单一经济类型的城市所具有的另外一种破坏力。比如像通用汽车这样的公司,它们被称为工业中的工业,又被看成是大工业的典范,但规范的坏处也是显而易
见的。比如在通用汽车,它们的财务报表的单位是5000万美元,也就是说如果你的账上没有这么多钱放在这里,可能你就会被显示处于亏损状态。所以你至少要
有一笔钱放在这里,而通用汽车下面若干个机构可能都要如法炮制,你就知道为什么它们的``成本''会那么高,因为有那么多钱都在柜上,但你就是没办法用。在
2000年初的时候,福特生产一款车可能要150万辆才能达到盈亏点,而雪铁龙可能只需要60万辆,名义上的说法是因为开始的研发成本太高,但实际上的原
因多半是跟通用汽车一样,只有把那些其实在账上的但所有人都看不到的钱都算进去之后,才能开始算盈利。

这样的规模让公司的盈利能力变得很差,又因为即使是决策者也不真正掌握资金的使用情况,所以就存在大量的瞎指挥行为\ldots{}\ldots{}然后再度造成成本上升,进入到一个恶性循环。

如果这中间还有一个工会,那就更麻烦了。关于汽车工业的那个UAW工会,我觉得是人类历史上的几朵奇葩之一。举个例子说,比如在通用汽车的弗林特工厂,在
UAW长期艰苦卓绝的英勇斗争之后,工人们现在每天只需要工作4小时至5小时,其余的时间都用来休息,但工资可得按照8个小时来支付。我一直觉得工会在美
国是以对抗为目的的一种社会关系,当这种对抗过于强大的时候,对于整个社会弥合裂痕是无益的,而且,当大公司过于庞大的时候,这种本来是企业内部的对抗性
的社会关系会影响到整个社会。它们跟政党之间的对抗的不同在于,我觉得它更充满了意识形态色彩的零和博弈的逻辑。当然,也有例外,工会的死对头,也就是公
司的管理层其实跟工会之间也并非没有默契。因为按照自由主义派经济学的观点,公司管理层的收入并非是从工人的工资里分出来的,它们之间不具备零和博弈的特
征,所以对于公司管理层来说,给工人涨工资,那么同比例给自己涨工资也没什么不对的,所以在经历了漫长的对抗,尤其是职业经理人和公司所有者之间越来越是
不同的两坨人的时候,大家就心照不宣地分钱了。当然,钱是不会平白无故生出来的,汽车公司自己又不会量化宽松,那就一定是分股东的收益了。你,作为投资
者,发现不但工会是你的对头,管理层也是你的对头,你有没有想一头撞死的冲动?

那个可怜的大卫·宾恩就是给这样的城市当市长,他可调动的资源和发挥的空间,其实跟他的中国同行一样,就是没有太多事可做。并且,市长是民选的,而公司是集权的,无论从效率还是管理力度上,你一定知道这其中的差别的。如果再有一个工会\ldots{}\ldots{}这世界就完全崩溃了。

到目前为止,底特律内城已经开始恢复,不过最重要的消息还是汽车工厂重新开始招工------对的,它还是依赖于这几个生意好转了那么一点的巨头。就像在面对周期
性的股票的时候,你要提起十二万分的精神一样,你现在面对的是一个周期性的城市了,你的工资,你的401K,你的投资现在终于遇到了不是最坏的时候了\ldots{}\ldots{}

\begin{center}\rule{3in}{0.4pt}\end{center}

\textbf{《城市的胜利》}

【美】爱德华·格莱泽 上海社会科学院出版社 2012年12月

定价: 49.80元

\section{即使聪明如理查德·费曼\ldots{}\ldots{}}

前段时间一个骨灰级的猜年龄游戏被微信用户们翻了出来,简单点说就是:说出一个0至9之间的你认为的幸运数字,把这个数字乘以2,再加上5,再乘以50,把得到的数目加上1763,然后减去你出生的那一年,最后得出的三位数,百位数是你的幸运数字,后两位是你的年龄\ldots{}\ldots{}按惯例,这种``神秘数字''游戏,一定会有人跟帖做苦苦思索状,并且不得其解,惊呼太尼玛神奇了。

在吃过100个豆之后,如果还不知道豆腥,不是你的嗅觉出了问题,就是智力有问题。我第一次用公式------居然是公式啊------把这道题解开的时候,对自己的智力表达了无限景仰,觉得人生豁然开朗,恨不得跟每个人说,你知道吗?那些数字,那些让我们崩溃和一头雾水的数字,只是干扰我们的神经的\ldots{}\ldots{}然后,你知道,在我们的生活当中总是充斥着大量的连公式都不肯列的死文科生们,接下来就可以蹂躏他们了。

这么重要的知识,如果只是用来跟别人讲笑话显然是一种浪费。如果你理智一点的话,你就会发现有人不是用它来证明自己的优越感,而是来决定你的前途。比如\ldots{}\ldots{}

据称大公司,尤其是那些需要动用智商的大公司和岗位经常会出一些怪题目,其中有一道比较简单的题是:你有好多好多硬币,撂在一起有帝国大厦那么高,问这些硬币能否堆放在一个房间里?这题当然不难,如果你知道帝国大厦有381米高,通常一个硬币在1.5毫米厚左右,直径在25毫米,你算出硬币的体积和硬币堆的总体积,然后你再接着算一个房间的体积------对了,是什么样的房间?这样当然也能得出结论,我随机问了五六个人,他们一般在算这一步的时候先给出一个结论,估计差不多。因为这算起来实在有点麻烦。

大公司的面试Boss们肯定不是在考你的心算能力,他们想了解的应该是你算法的正确性,实际上是你的逻辑推理能力:即使你是一个中国人,从来没去过纽约,大约也知道帝国大厦有100来层,那么你只要知道在一个房间里是否能平铺100个硬币就够了,把它们撂在一个房间里的高度,就是你的所有硬币数量。100个硬币能占多大地方?所以你就知道了,在这个题目里,帝国大厦多高不重要,硬币多大不重要,房间多大也不重要。

我们把事情搞复杂的一个主要原因就是被过多的前提条件所干扰,不仅仅是在面试的时候你思前想后把事情复杂化------通常来说就是搞砸了,而且它差不多无所不在。

理查德·费曼,就是参加过曼哈顿计划的那个很有名气的物理学家,他是一个热爱生活和各种恶作剧的聪明人。有一段时间他爱上了拉斯维加斯,顺道也爱上了它的赌场,对于一个物理学家来说,发现这里的骗人把戏是件非常简单的事,但他发现,居然有一个叫希腊历克的家伙在靠赌博赚钱!于是他决定去会一会这个人。

``我很想知道你怎么可能靠赌博维生,因为像骰子之类的赌赢概率才0.493。''

``你说得对,让我解释给你听,我不赌骰子什么的。我只赌那些对我有利的。''

``呵!它们什么时候对你有利过!''

``其实这也很容易,我就在赌桌旁闲逛,如果有人说,9点,一定是9点,那人兴奋极了,他认定就是9点,而且正想下注。于是我说,我跟你赌4元对你3元,这不是9点,长期来说我会赢。我不直接下注在骰子上,但我跟其他赌客赌------他们都有偏见,会迷信一些幸运数字。''你看,即使聪明如理查德·费曼,他也会纠结于赌场的输赢概率,从而忽略赌场中真正的胜负手其实与赌场没有什么关系。希腊历克接下来的生活比他想像的还要好,``现在我已声名在外,就更好办了,因为很多人会来跟我赌。就算知道机会不怎么大,他们也只是为了如果真的赢了希腊历克就可以四处告诉别人这个动机。我是真的靠赌博维生,这种生活也好极了。''费曼对这件事的总结就是,``希腊历克确实是个很有学问的角色,他人很好。我谢谢他教了我这么多,现在我全都明白了。知道吗?我总喜欢弄明白这个世界到底是怎么一回事!''不过,我估计即使聪明如理查德·费曼,大约也没有完全弄明白这个世界到底都是怎么回事。

对于希腊历克的把戏,如果我们给它取个新名字,好比叫做``21点的衍生品交易'',这世界新时代的起点就变了。华尔街的天才们后来学会了这一招,不管是刘易斯·拉涅利的住房抵押贷款证券化,还是后来的``信用违约掉期'',都有希腊历克的的影子和遗风存在。你要是直接判断一个债券或者某个证券的价值,可能有点难度(估计胜算不会超过0.493),但如果你来判断它们的违约情况,或者把别人的债务打包成新的证券,远离最初的那些经营上的风险,那么你还是有更多的赚钱机会。你考量的不是一个你完全不了解的公司的经营状况,而是你的其他同行们的判断力。

我们知道的华尔街的数学天才们的大部分商业模型,还有利用市场定价错误找到的赚钱途径,其实都可以归类到希腊历克的衍生品范畴中。这些在此后几十年里呼风唤雨的金融衍生品是华尔街的财富源泉,也是我们悲惨命运的源头。

如果说到本质,那其实也怪不得华尔街的天才,因为它根本就是凯恩斯说过的那个有关选美的问题。小镇要选出公认的最漂亮的美女------你判断谁漂亮不重要,你应该判断的是别人会如何判断。你看,希腊历克的幽灵``唰''地一下又出现了。它不光是出现在衍生品当中,还会出现在所有跟你赚钱相关的资本市场的地方。

所以,在一个人人都提大数据的时代,教会我们不那么为这些事纠结的人,很有可能是最终帮我们懂得``这个世界到底是怎么一回事''的人。如果我们一定要再加一个强有力的论证的话,那么你一定记得另外一个古老的故事:两个人在森林里,突然遇上了一只准备扑向他们的老虎,这个时候一个人选择撒腿就跑,另一个人选择系鞋带,前面那个人问,你系上鞋带就会比老虎跑得快吗?那位兄台说,我只要跑得比你快就好了。

嗯,希腊历克在拉斯维加斯不用在乎老虎机,你在森林里也不用在意老虎。这就是我们生活的真谛啊。

还有一道算术题,说假设你有一个4万公里左右与地球赤道等长的绳子,现在它紧紧缠在赤道的地面上。然后,你为这根绳子加了1米。你觉得这根绳子会因此与地面之间增加多少距离?

它的答案是,任何一个圆的周长加上1米,跟你想像的庞然大物是没有关系的,它都遵循周长=2πr的这个公式,增加1米的周长,不管原来的周长是4万公里还是4米,1米=2πr,r约等于15.9厘米------在增加1米之后,你看到紧绷的绳子整整被提起了将近16厘米,在这个故事当中,地球那个庞大的有6000多公里的半径,就是那个老虎。

\begin{center}\rule{3in}{0.4pt}\end{center}

\textbf{《π:世界最神秘的数字》}

阿尔弗雷德·S. 波沙曼提尔

吉林出版集团有限责任公司 2011年4月

定价:28.00元

\textbf{《别闹了,费曼先生》}

理查德·费曼

生活·读书·新知三联书店 2005年1月

定价:15元

\textbf{《谁是谷歌想要的人才》}

{[}美{]} 威廉·庞德斯通

浙江人民出版社 2013年2月

定价:49.9元

\section{听说你打算定制一个伟大习惯?}

眼见得《习惯的力量》红火了起来,一本书火爆而成为显学,我觉得它的标志就是有人给书做PPT,有人拿着这PPT鼓励后进并且大声疾呼------我觉得他们都是有好习惯的人。我在看这本书的时候,也收获很多,比如懂得的一个道理就是,我们的人生之所以随波逐流,是因为一堆坏习惯,而坏习惯的养成多半原因是我们对自己的奖励机制出了问题,呀,真的是应该好好反省一下。

若干年前,一位人力资源管理专家给我讲过一个浅显的道理,她说如果有一个你的得力下属突然跑过来跟你说,有个竞争对手公司出了更高的价钱请他跳槽,你应该怎么办?通常的做法是给他加点工资,稳定下来。专家说,你看,你这就是鼓励背叛。我觉得这太有道理了,以至于只记住了前半截:不应该鼓励背叛。但如果不鼓励背叛,那到底应该做点什么------这关键的后半截完全没有印象了,以至于我在茫茫暗夜中又摸索了许久,并且至今也没有摸出来,唯一的进步就是,明白不管要不要鼓励背叛,得先弄清楚手中的筹码。我想说的是,不管是奖励自己的臭毛病,还是奖励别人的背叛,你一定要知道自己手里到底有怎样一副牌。

我看小白兔们已经把改掉坏习惯、树立新习惯列了到议事日程中来,我就很想泼些冷水,好心地说,是担心它成为一个新的错误奖励回馈体系。《习惯的力量》中举了霍普金斯大师的一个卖牙膏的好案例,到底是迷人的如大明星般的微笑还是去除牙菌斑更有助于一个好习惯的养成。我知道另一个例子是这样的:当年知识青年带着牙刷牙膏来到广阔天地准备大有作为,本来他们可以成为文明生活的使者让文明在新天地中扎根发芽,但结果如你所料,他们很快就接受了贫下中农的再教育,放弃了文明生活,没有几个人坚持每天刷出白亮的迷人的城市感觉牙齿了。

我至今不知道让人们放弃文明生活的习惯是如何形成的,显然也没有什么太好的激励机制值得让他们自绝于文明世界。能想到的,比如是一大早起来就要下农田干活所以没有时间,或者是早晨起来太冷了,或者是刷牙水要到很远的地方才搞得到,当然还有可能是无期徒刑一样的绝望,对前程未知哪里还顾得上牙!在种种可能性当中,习惯脆弱得如风中之烛。

所以你打算定制一个伟大的习惯以提升自己的品格,和你真正有这个习惯之间的距离,我觉得可能需要引用一个c2来做系数。我用的是爱因斯坦的c,就是光速。好吧,你看出来了,我对《习惯的力量》的振振有辞有点不那么以为然,它作为励志书的一种,总是只告诉我们事情的一方面,只是一方面,绝对不说另一方面。

巴菲特第五十一定律说------不用琢磨为什么会有第五十一,以及Top50都是什么,我就是随口一说,人一旦成了圣人,每句话都可以写进《论语》和《新约》------这定律是说:``在投资中,对大多数人而言,最重要的并不是他们知道多少,而是他们能真正意识到,到底有多少是自己不知道的。实际上,除了少犯大错误之外,投资者根本就不需要做什么。''贾森·茨威格对此的解释是:我们之所以总想让别人以为我们知道自己并不知道的东西,最根本的原因,就是承认自己无知会损害我们的自尊。实际上,一知半解是最危险的事情,了解一星半点,会让我们感到胸有成竹,而承认自己无知带来的感觉,却会让我们无比恐惧。

回到我们的各种坏习惯的问题,我觉得《习惯的力量》最大的贡献应该是它没强调的那一部分:我们的无知,并且我们挑战无知感觉的能力并没有我们想象得那么强大。

贾森·兹威格是《Money》的编辑,又是本杰明·格雷厄姆的衣钵传人,所以他总是能找到更好的故事让我们自惭形秽。比如针对我们前面说的``人总是不知道自己是谁定律'',他讲了这样一个故事:

一位心理医生,收治了一个赌博成瘾的患者,这名患者曾经在一个周末赢过10万美元,但忽略了他总共输了190万美元的现实。关于输的那一部分,他没有任何感情色彩地讲了一遍,这部分信息事不关己,他真正记住并且津津乐道的东西,就是10万美元。这会让他进一步相信自己的运气甚至能力。由于收益预期让我们记住了昙花一现的成功,于是我们常常把记忆中模糊不清的20\%收益概率当成100\%的收益现实。

而我们当然只是记住了连续三个涨停的美好故事。2007年的时候,我的一位前同事,开会时都忍不住要私下嘀咕:我现在可真了不得了,炒什么什么涨,眼力非凡啊。后来的故事我就不清楚了,但我知道的是,即使是圣人巴菲特,他的连续二十年或三十年的年复合增长率,也大都实现于1980年前后开始的史上最大牛市中,牛市结束后,你知道,从奥马哈传出来的圣人之言越来越多,圣人巴菲特已经是一个伟大的投资教育家了。

而另一个针对``我们也没打算了解这世界到底是怎么回事定律''中,贾森·茨威格讲了一个复杂的故事:在给投资者们讲课的时候,他有时会把手伸进密封袋,掏出一条响尾蛇,然后扔向观众。``理论上说,一个理性的人应该坐在座位上------正确的反应应该是用短暂的时间判断一个投资方面的专家在演讲过程当中向观众扔出活蛇的概率有多大,是否值得高声尖叫、大惊小怪。''实际上,好像没有什么人是真正准备理性的,他们做的事情就是不管男女一律尖叫着并试图躲开,贾森·茨威格引用一位叫安托万·贝沙拉的人的话说:``我们的大脑根本就不需要真实体会才会对危险做出反应。''

贾森·茨威格的这本书叫《当大脑遇到金钱》,事关大脑的事,容易让人想起理性,在我们的语言体系里,但凡要``动动''它的时候,都是往一个好的方向发展的,但实际上我们根本没有这么多乐观的理由,我们的习惯来自于大脑的本能反应。我觉得我是一个正常人,如果在一个会场上有人扔了一条响尾蛇过来,我才不管它是真是假呢,先躲开再说啊,尖叫是不体面------我也不认为尖叫是我的习惯,但我想我也一定会尖叫的。

最后,一个显而易见的事实就是,好习惯当然是好事,坏习惯肯定应该要改掉,但通常来说我们拥有的都是不好不坏的习惯,无伤大雅的小癖好,你弄清楚了它的奖励机制,然后你想着要为它多做点什么事。也就是想想而已。回到人力资源管理专家的世界里,我所知道的是一个错误的激励机制确实鼓励了背叛,但多少也会给自己一个额外的奖赏:这孙子为什么起了背叛之心?你知道,你管不了别人的时候,就要管自己。这话说给自己听的时候是励志,但将心比心,你怎么知道这个有二心的孙子不会有所反思?虽然牌抓在自己手里是王道,但你知道对手手里的牌也是王道啊。

\begin{center}\rule{3in}{0.4pt}\end{center}

\textbf{《当大脑遇到金钱》}

贾森·茨威格

广东经济出版社 2009年11月

定价:35元

\textbf{《习惯的力量》}

查尔斯·杜希格

中信出版社 2013年3月

定价:42元

``我们的生活在某种程度上有其固定的形态,但却是习惯的集合体。这些习惯系统化地构成了我们的喜怒哀乐,让我们走向自己的命运。不管最终命运如何,我们都无法抗拒。''

\section{``这条贪婪的隧道尽头\ldots{}\ldots{}''}

疯狂工作,尽情享受!赚钱,买你所爱的!这些话现在我们听着也是越来越耳熟了,很多时候我们都很愿意畅想这样的场面,比如``忙完这几天,就\ldots{}\ldots{}''然后再一直延伸到很远的地方,比如类似于``35岁就退休''这样的豪迈。这种乐观主义的精神如果放在美国,那就是美国梦,如果放在中国,那差不多就是中国梦。

但是有一个美国人突然跑出来表达了不同的意见,他说:``我们越是努力地工作赚钱来购买这些东西,就越是缺少时间和精力来享受已经购买的东西。美国文化向外界传递了一个越来越混乱的信息:疯狂工作的同时又要尽情地享受人生,这两件事情是不可能同时成立的。''

嗯,在美国人的美国梦中,家是一座奇妙的白色小房子,羞怯地躲在树林或枫树林中,沐浴在紫丁花的芳香下,至少有一面墙上爬满了藤蔓。``带阶梯式山墙的屋顶'',``很小的木质屋顶窗'',``淡蓝色或浅绿色的百叶窗'',``通电的油灯复制品''``有四根帷柱的床''\ldots{}\ldots{}在更早一点的时候,它在威廉·莱维特的长岛小镇时代就已经深深植入美国人心中,现在,这个美国人要说,咄,你们别做梦了!他说:``有关努力工作的论点总是建立在一个谎言之上,让人们相信有一天他们终将会得到满足,就算不是事业上的成功,也会是他们用努力工作换得购买他们想要的一切东西的能力------可是那一天似乎永远不会到来,在这条贪婪隧道的尽头根本就没有光明。''

这话听着可真让人心碎,并且说这话的人可不是什么愤世的颓唐青年,而是罗伯特·赖克,克林顿时期的美国劳工部部长。你知道,一个劳工部长说出这样的话来,我们没有理由不重视一下。

赖克部长的道理其实也很简单:每个人的消费都会有乘数效应,这个世界总的来说是依赖消费来完成财富增长的,但如果分配体制出现问题,财富就会集中到少数富裕人群中,他们即使再有消费热情也消费不了太多的东西,所以他们的钱就转而进入到投资领域,以投资获得收益为主要财富流转方向。而中产阶级主观上既希望有同样富裕的生活,客观上往往也被鼓励进行更多的消费,但他们又穷,于是就会被鼓励通过贷款来实现消费,于是借钱成为美德------双方的钱最终都流转到一个金融领域当中,富人希望赚更多的钱,穷人希望借更多的钱,最后双方共同作用的结果导致金融危机出现,然后,陷于萧条。

部长很体贴,还怕大家不懂,就举了个更简单的例子,``就像玩扑克牌一样,筹码集中在少数人的手中,其他玩家如果想继续玩游戏的话,他们就得借钱。''

你是否感觉到有一个让人毛骨悚然的声音正响在耳边:``陪我玩儿嘛。''我在看到这里的时候基本上就在倒吸几口冷气之后感觉出来了世界的残酷。那个一直鼓励着我们支撑着我们勇往直前消费的美国老太太哪里去了?

有的时候你不得不承认,什么事都得有个前提。比如在美国老太太和中国老太太的完美对比故事中有一个重要的前提是:经济持续繁荣。现在我们知道有两种繁荣,一种是特别繁荣,房价不断地涨,你可以随便买房子,不用去想还不上贷款的问题,反正你还不上的时候可以把它卖掉,这个我们很早就知道了,它叫庞氏骗局,它的结果是次贷危机;另外一种繁荣是温和一点的,你量力而行像个体面的中产阶级那样把房子买下来,入住,只要你没丢掉工作,你就可以像美国老太太那样享受一个完整人生\ldots{}\ldots{}但你怎么知道你不会丢掉工作?部长是个悲伤的人,你一定注意到了前面他甚至都说到了萧条。他自己真的觉得现在离萧条相去不是太远,他引用的数据就是,在1970年代末的时候,1\%的美国最富有人群占了社会总财富的9\%,而在2007年的时候,占到23.5\%,已经达到了萧条前1928年的水平;而1950年代至1980年代美国家庭税后储蓄率是9\%至10\%,到2000年时下降到3\%,1960年代贷款额占家庭收入55\%,到2007年则上升到138\%。

然后部长悲伤地引用了大萧条时的痛苦记忆,把伤疤撕开了给我们看:曼希居民不再奢望能够活得更好,大萧条还让他们每天生活在经济倒退的恐慌当中,就像是``一个病危的人无助地躺在床上,生活中惯常的忙碌都消失了,只剩下自己和自己抗争,不断地给自己抛出一个又一个关于未来的突兀的问题。''

理论上来说,我们的工作之所以入不敷出有两个原因,一个是前面我们说过的,我们过分地热衷于消费了,另一个是因为我们的劳动力定价出现了问题,在一个自由流通的市场里,纠正错误定价并不是一件太难的事。我们总是会把自己的工作当成一种个人财产,但实际上并非如此,它只是市场中的一个元素,它只有在具有流动性的时候才表现出价值,按照熊彼特的说法,只有这个时候才会产生资本主义。他还说,在1772年前,只有4\%的人类是自由人,那个时候不大可能有对劳动力正确的定价。

所有中产阶级都出现在人类把重大技术问题解决之后,尤其在生产线、电气化解放了劳动力之后,天突然就亮了,中产阶级可以享受起生活了。

我们还要提一下亨利·福特,他认为现代社会有一个基本的经济协议,可以上升到人类层面:

一个现代高产的经济的核心是一个基本的经济协议。工人同时也是消费者,他们用工资收入来购买其他工人的产品和服务。因此,他们的工资在经济生活中是不断循环的。如果工人的工资不足,这个基本的协议破裂的话,那么即使这个经济生产出更多的商品和服务,人们也没有能力购买------想叫人陪你玩,你得拿出诚意来。

这个朴实的道理在持续几年之后,另一个天才小阿尔弗雷德·斯隆认为你没钱也没关系,我来借给你\ldots{}\ldots{}然后就是借,更多的借\ldots{}\ldots{}另一个无处不在的凯恩斯说:如果你接受无法偿还的债务,那么这就是欺诈。

我的迷思是,福特担心协议最终破裂,中产阶级最终还是要被扣上欺诈恶名,这个时间段是如此之短,只运行了二十几年就到了1920年代末的不可收拾的地步,中产阶级的日子有这么苦吗?

于是我又产生了一个与我们自己相关的迷思,要知道我们现在激情勃发,也是有了中国梦的人!虽然莱维特小镇的白栅栏爬山虎还没有普及到我们头上,但你知道我们也是可以畅想的人了。罗伯特·赖克部长说,美国经历了三个阶段:1870年至1929年是收入和财富不断积累和集中的年代,1947年至1975年繁荣得到了广泛的共享,1998年至2010年,财富又开始重新集中\ldots{}\ldots{}我想知道的是,关于中国梦的那一部分是否真的也可以先跟这时间对照起来看看。

\begin{center}\rule{3in}{0.4pt}\end{center}

\textbf{《美国的逻辑》}

【美】罗伯特·赖克

中信出版社 2011年4月

定价:32元

``某些物品只有在一部分人拥有而另一部分人缺乏的情况下才会显示出自身的价值,确实,`价值'一词的含义与我们所扮演的社会角色是息息相关的。''

\section{``我知道事情真相啦''}

这句话来自于郭德纲的段子。有人教唆一孩子,你要是想跟家里要钱,你就跟你爹妈说,``我知道事情真相啦'',于是\ldots{}\ldots{}它果然好用!拿这句话做标题并非是让大家如何骗吃骗喝,而是真相这东西很重要,但又经常让人琢磨不透。在运气好的时候你喊着知道真相,可能就足够你所向披靡了;运气差的时候,你以为你知道事情真相,但结果当你用你了解的真相去解决问题,最终可能南辕北辙。

我特别想说一个极端的例子,一个叫杰里米·温斯顿的美国人在万圣节被执行了死刑。他的律师大卫·道说:``他真的很后悔,我第一次见到他,就知道他后悔自己的作为。''这位温斯顿有一天凌晨打破一家一楼的窗户,把露西从她的床上抱走。那天下午,警方发现已经被勒死的露西,被性侵,头骨破裂,可能是被碾过,她只有5岁。这是那个后来感到后悔的变态恶魔温斯顿干的,这是真相的一部分。大卫·道还告诉你另外一些事:这个杰里米·温斯顿在8岁的时候亲眼见到父亲被枪杀,在接下来的7年时间里,他的母亲带着他先后跟11个男人在一起生活,其中6个定期殴打他和他的母亲,一个男人用枪指着他,一个男人用砖头打他,一个男人还鸡奸了他。

义正辞严者会说,别人可能也有困顿而混乱不堪的童年,但并不是每个人都成为了变态恶魔。很有可能还会举出几个特别有说服力的自强不息的励志故事来说明。迈克尔·刘易斯在《盲点》那本书里,讲了一点孟菲斯的黑人少年的故事,但他首先告诉你,有很多悲剧就是那么无可避免地要发生,你要问为什么,在整个社会所提供的资源和发展的脉络似乎都指向一个悲催的结果,生于其中,几乎无力摆脱。他在那本书里讲了一个黑人少年迈克·欧尔如何成为美式足球明星的故事,如果你稍有理智,看有那么多人欣赏和帮助他,你就知道这是一个特例。而通常情况是,在街头无所事事,吸毒,斗殴或者抢劫,监狱,更加无所事事,毁灭。

所以,在前面讲的两个故事中,真相是不可避免的罪恶发生的概率远远大于正常成长的可能性。幸好,这样的悲剧正在减少,因为从美国的统计数据来看,犯罪率正在下降。所以接下来需要知道的真相是:犯罪率下降的原因是什么?

鲁迪·朱利安尼,如果让这位纽约前市长来回答这个问题,他一定会洋洋得意地指出这是他的铁腕政策的结果,他有一个着名的``破窗理论'',就是街区中有一扇玻璃是破的,那么它一定会导致更多的破窗产生。所以问题的解决方案就是,让窗户都完好无损,减少破窗出现的机会。

嗯,我得说在纽约的中产阶级一直在记得朱利安尼市长的好,他的不顾忌种族问题的强力手段的确让纽约降低了暴力犯罪率。

当然,不同意这一观点的人也很多,道理非常简单,因为在朱利安尼主政纽约同时,全美的犯罪率都在下降。

史蒂芬·列维特,写《魔鬼经济学》的那个家伙,认为是堕胎合法化使得非婚生子女出现数目减少,客观上降低了高犯罪率人口的数量。这听着也并非没有道理,咳咳,但一定会有明眼人指出这不是出身决定论吗?

所以正统的自由派经济学家就不会这么简单地得出这样的结论,我相信托马斯·索维尔可能会指出是因为移民第二代高峰期已过,所以整个社会都更加平和才导致犯罪率的降低,是啊,全世界都老龄化了,犯罪率当然也降低了。

有一段时间我也把这种看法当成真相之一种,但这个想法最近也受到冲击了。一是老人家未必没有肾上腺,冲动起来可不在话下,最近在中国做的几起大案的主角都六十开外了,其中一个还是因为不能容忍户籍警认为他只有59岁而断然出手的。二是这群老人家少年时做过红卫兵又被抛弃到乡下去接受再教育,回得城来待业,好容易上了几天班又下岗,下了岗之后还要被统称为``4050''人员------这听起来就像一个巨大而让人绝望的包袱------的这一代人,如果郁结于心中的困惑和绝望不被有效排解和解决,那么你就很难说老龄化是社会稳定(在中国)和低犯罪率(在美国)的救命稻草。

当然,有深刻的社会学者会把这些年龄和出身的问题扔在一边,他们会探讨深刻的社会问题。比如,一种说法认为,``犯罪者将暴力犯罪当做一种报复行为。他是在惩罚受害人------受害人居然拥有这么多尊重,而自己拥有的却少得可怜。犯罪人可能甚至希望可以进行重新分配。求尊重,如果不给我就反叛。''这一派社会学家认为犯罪率降低的原因是Rap或者其他艺术形式的出现,给那些容易犯罪的人带来另外一种尊严,``那些本来尊重短缺的人突然拥有了大量的盈余'',潜在犯罪者就这样摆脱了成为社会渣滓的命运\ldots{}\ldots{}这个我觉得稍微有那么一些不靠谱了。要知道,成为迈克尔·乔丹这种梦想从来没有被黑人少年所抛弃过,但他们并没有因此在高富帅的道路上狂奔。

好吧,以上说法可能都是真相的一部分。但有一个真相可能是我们这些自由派所不愿意看到的:美国1980年至2000年的犯罪率下降很重要的原因在于监狱关押犯人的数量从180万增加到649万(掩面)。进一步的大数据表示:刑期每增加一倍,犯罪率下降10\%至40\%;监狱人口下降10\%,暴力犯罪增加4\%。

千言万语化作两个字:严打。

鲁迪·朱利安尼看起来倒可以把自己当做一个赢家了。如果这就是世界的真相,那么所有对美好世界充满幻想的人类都会万念俱灰以泪洗面吧?好吧,如果同样是看这20年,对于美国人来说,还有一个巨大的真相就是持续的繁荣。

朱利安尼当政之前是做经济犯罪检察官的,做的最轰动一件事就是把垃圾债券大王迈克尔·米尔肯送进了监狱。但在他当政之后,一手镇压布鲁克林的小混混的同时,另外一方面还是借着自由放任的里根主义让华尔街重新成为创新主流,在这一轮金融危机发生之前,华尔街光是为金融从业者发的工资就达到了786亿美元。纽约复兴和美国人又乐观向上起来的原因在于企业家精神的复兴,也在于繁荣所带给他们的回报,而这些最终让全社会受益。是啊,是在大萧条的时候人们说起了美国梦,但美国梦的样本大都存在于繁荣期之中。

这就说到了一个沉重的话题。沉重在于它只是几个我们必须要想但真是不愿意被问及的问题:未来会持续繁荣吗?中国城市里的第二代``移民''问题是否被重视过?那些存量的矛盾在增量惠及不到或者增量不再的时候是否还在郁积?每个劳动者或者市民,那些自称``屌丝''的年轻人,他们认为自己得到尊重了吗?

\begin{center}\rule{3in}{0.4pt}\end{center}

\textbf{《城市之王------纽约市市长朱利安尼》}

安德鲁·科兹曼

东方出版社 2004年10月

定价:28元

``谋杀可能是最丑陋的罪行,因此,当你发现大多数的杀人犯长相如此平凡,你总会感到不可思议。''

\textbf{《死刑台前的告别》}

[美]大卫·道

中国人民大学出版社

定价:32元

\section{我们只是想摆脱庸常生活}

在经历了两年之后,这个栏目跟我们这本杂志一起也要告一段落了。在过去的两年中,它絮絮叨叨地讲了一些有关价值投资的事,如果总结下来,可能无外乎不要把鸡蛋放在一个篮子里;在别人贪婪的时候你要恐惧,在别人恐惧的时候你要贪婪;天下没有免费的午餐之类的常识。这些常识隐藏在各种各样的书中,也隐藏在各式不同的人生中,它们以不同的面目出现,很多时候,你会沉浸于这些不同的故事当中,以至于忘了我们要说的是有关投资和财富的话题,但好在一切都关乎成长,在这个栏目中还从来不曾绝望甚至困顿过。

它大约涉及有100本书,如果为它做个总结的话,差不多绝大多数都来自美国。在这其中有一种是关于投资理念的,我得承认,对于一个从来不亲身去投资的人来说,这东西都是纸上谈兵,巴菲特总是会提及他的``护城河''理论,那些根本不用考虑谁来当CEO都会经营得很好的那些公司。彼得·林奇会说,要去看那些停车场排得爆满的公司,要投就投这样的公司。如果按这样的思路去指导我们的投资,工商银行你觉得怎么样?它们那里永远在排队。巴菲特借着这个指导了自己的可口可乐、运通、吉列还有过去的迪士尼公司,但我们知道,当年有一个CEO上来就改了可口可乐的配方,结果呢,销量大跌,我们还知道迪士尼在不同的人手中还真是有不一样的结果,吉列呢则卖给了宝洁。有位伯克希尔哈撒韦的股东抱怨说,像可口可乐这样的品牌经营商太少了。巴菲特的搭当查理·芒格尖刻地批评他说:``哪有那么容易的事,你难道做成了两三笔买卖,就指望能让全家人一辈子吃喝不愁?''而本杰明·格雷厄姆则会说,如果你只看停车场,而不去看财务报表的话,你与盲目也没有什么区别。

我觉得这些投资理念已经接近于技巧。如果还要更进一步,那就只有利勒弗尔才有资格总结那么几条。他是股票草莽时期最着名的大作手,虽然最后也同样走向破产,但他的那些总结绝对值得信赖。比如,``我常常说,在正处于上升状态的市场中买进,是最舒服的股票买入方法。请注意,关键在于不要一门心思想着买得尽可能便宜,或者卖得尽可能最高,而是一定要买在或卖在正确的时机。当我看空并卖出某个股票时,每次卖出的价格都必须比前一次卖出的价格更低。当我买进的时候,情况正好相反。我必须按照步步上涨的方式买进。''再比如,``你看看,很多人徒有精明的名声,他们之所以看多,是因为他们已经持有股票。我不允许手上已有的头寸------或者先入为主的成见------为我做任何思考。我再三强调我从不和行情纸带(自动报价机打出的电传记录纸)争论,这就是缘故。

因为市场意外地、甚或毫无道理地对你不利,你便对市场生气,那就像得肺炎的时候对肺生气一样荒唐。''

我们大多数时候都会犯这样的错误。因为你拥有它,所以你会更看重它的价值,其实大多数时候我们遇到的问题是,如果它不在你手中------你会觉得它更有价值吗?贪婪的时候恐惧和恐惧的时候贪婪,差不多是人人都懂的道理。但如果这只股票,或者更广而言之,这个业务在你的手里,你能想得清楚吗?

这就要说到第二种类型的书。利勒弗尔的这种态度背后是一种心理学上称之为``禀赋效应''的东西。愚蠢的人类经常会在这个问题上让自己困顿不堪。我说的第二种书,很惭愧地说是那些取着各种``怪诞''、``魔鬼''等让人不明觉厉的名字的或类似内容的书,这些书其实比它们的名字要好一些,虽然它们总是用大同小异的案例,说着差不多的道理。这其中被引用最多的当然是名头最大的凯恩斯的``选美理论''。这个故事是如此完美,我觉得可以再讲上一遍:所谓炒股就相当于一个小镇选美,可能有100个候选人,所有小镇居民都可以投票选出最美的5个------最后得票最多的那个将荣升为最美小姐。你要知道,对于你来说,谁是最美的并不重要,最重要的是你的选择和``公认''最美的那个结果最接近。我至今还没有看过任何一本凯恩斯的书,如果无知一点说,就这个小故事也足可以让人对他那肃然起敬,这是资本市场的真谛。

这也是我对投资始终``叶公好龙''的原因,这确实是一个高智商+高情商的工作,你要懂得别人怎么想,这是一件很牛的事,但你知道了之后怎么做呢?看出次级贷会出大问题的人有不少,觉得这个世界疯狂的人也有不少,参与其中并且知道自己正在玩火的人可能更数以万计。这个时候,一个好人会选择做那个讨人嫌的人:``就要发生大危机啦,你们再做下去会惹大麻烦的'';一个聪明人会像迈克尔·巴里那样寻找合适的对冲风险的产品希望自己赚上一笔;而约翰·保尔森则会选择跟高盛合作发行信用违约掉期------这么说吧,他就相当于在北京的大雨天里,远远地站在被雨水冲开下水井盖的路边,打赌一定会有人掉进下水井里去,然后他赌赢了,赚上一大笔。你一定觉得他是个恶人,但如果他是一个做对冲基金的,你就会当他为英雄。

所以我不仅讨厌约翰·保尔森,更讨厌把这些行为制度化的像高盛这样的公司,它们有一个共同的特点------毫无贬义的一种说法是,``利用市场错误定价''

来赚钱,好吧,市场定价错误是由那群贪婪的人造成的,聪明人可以惩罚这帮Loser,但是,我还没见过任何一个金融危机不是以华尔街之外的那沉默的大多数无辜者的困顿生活为代价的。在为这个栏目所看的书中有相当一部分是来自于这样一群``聪明人''的故事。

以至于我会觉得``聪明人''简直太多了------利勒弗尔这些早期的``股票大作手''就不提了,经历了大萧条和二战之后,有Go-goYear的沸腾的1960年代和1970年代,有撒谎者的扑克牌的1980年代,还有豪华阵容的长期资本管理公司在1990年代,接下来就是我们已经可以感受得到的互联网泡沫,如今还没有完结的次贷和欧债危机。迈克尔·刘易斯和乔·诺塞拉这些人都是写这些故事的高手,《沸腾的岁月》、《撒谎者的扑克牌》、《大空头》、《拯救华尔街》、《众魔在人间》,这其中的那些聪明人,他们为我们赚钱的欲望做了无数注解,为依赖聪明智慧成为富有之人提供了无数标杆,当然,如果你是一个警醒的人,你可能会读出异样,甚至不堪------你能想象那些有诺贝尔经济学奖获得者的顶尖智慧的人类在对冲基金的路上走到了哪一步?长期资本管理公司最终是倾其全力来赌未来3个月内大盘指数可能会上涨10\%,凯恩斯这位经济学家中的经济学家,整个20世纪的规划者,看到这些智商不比他差多少,而对人类的破坏性又是如此之大的同行们,是不是要瞬间崩溃呢?

在投资历史上有一段时间有一种理论是``股价总是趋向于正确反应公司价值''的``市场有效论''。巴菲特就始终不相信这一点,在看过这些故事之后,你很容易不再相信简单的自由市场理论。虽然股票和投资也有那么一种``供求关系''在里面------供应与需求理论在资本市场总是很容易被忽略掉,以至于威廉·伯恩斯坦要特意强调一句:不要相信``金融危机来临时,金钱潮水般从资本市场中退去''这样的鬼话,因为有撤离就有进入,它们从来都是同步的------当然也是扭曲的,不要忘了,``在别人贪婪的时候你恐惧,在别人恐惧的时候你贪婪''这句话是赚钱实践者巴菲特说的,而他这句话恰好可以证明,供求关系同时受制于心理因素和理性或者非理性的判断,最终是非理性的一个结果。

它们是如此有趣,以至于你会忘了投资这回事。好在它的另一个副作用也是如此,作为一个正直的人,忘了投资也不是啥坏事,因为它很容易让你对这个世界的聪明人产生绝望。

不过,显然不必如此。一个叫安迪·凯斯勒的人写过一本叫《我们如何来到现在》的书。这个人在华尔街做分析师,然后做对冲基金,然后又跑到硅谷投资------在我们所介绍的这个世界里,他没有成为十恶不赦的大魔头并非是因为他的智力不够,而是他知道有一种更好的活法。迈克尔·刘易斯也一样,他不但揭了自己在所罗门兄弟做投资时的老底,而且用他的聪明智慧为我们提供了上面提到的那些书。并且,《我们如何来到现在》还宽容地为我们提供了如何看待创造、创新、银行、投资银行、风险投资这些东西的看法和他们的相关性。

好在,这个世界还有创造啊。所以更多的时候,你只要耐心地看待这个世界另一部分智慧的人如何建造起一个又一个伟大的时代,洛克菲勒、J.P.摩根、亨利·福特、范德比尔特、乔治·伊士曼、比利·杜兰特、罗伯特·诺伊斯、巴里·迪勒、大卫·格芬、比尔·盖茨、迈克尔·布隆伯格、拉里·佩奇、史蒂夫·乔布斯\ldots{}\ldots{}在这一连串名字当中,没有一个人会告诉你如何做好一笔投资,哦,范德比尔特可能会有心告诉你这件事,但人们记住的是他的中央车站,是他建立起来的美国铁路系统;比利·杜兰特也可能会热衷于说他的投资经历------实际上他太爱自己的投资事业了,以至于把自己一手创办的通用汽车、雪佛兰汽车最终都输掉了。好吧,但看这些人的故事,你会感觉有另一个美国,另外一群美国人,他们从19世纪到20世纪,在近200年的时间里,完善了一个不错的工业文明,还建造了一个很有想象力的信息文明。

对于我们来说,你了解他们,可能就是了解公司的价值,了解伟大公司的价值,你可以大材小用地拿来指导投资,去发现好股票,但显然更大的价值并不在此------你会感受到创造力的价值。

在这个因为稀缺而诞生了经济学、因为稀缺而产生了经济社会的世界里,创造力是最稀缺的一个。我们或者投资创造力,或者投靠创造力------在这些书背后,是我想象中的,这个栏目的价值。

于我这个作者而言,可能是收益最大的一个人。所以最后要说的是,它几乎给我打开了一个完全陌生的一个空间,这个空间里,听上去那些人类似乎只是在为自己的财富而殚精竭虑,给人的感觉很芸芸众生,但要知道,我们对自己的生活有多不满意,就意味着我们有多渴望一种新的更美好的生活,在这些焦虑背后,每个人孜孜以求的是我们想摆脱庸常生活的愿望所在。

\section{当心,被他们骗了}

大卫·奥格威有一天注意到办公室里新来的年轻人,他厌恶地摇摇头:这群刚毕业的名校大学生们,虚荣心强,把麦迪逊大道当成自己的羽毛,做的事情呢?蜻蜓点水马马虎虎,并且不爱加班,到点回家或者出去欢乐,目无老板,随时准备着收拾东西跳槽的样子。嗯,实在不怎么样。

你听着心里是不是一惊一乍的?大卫·奥格威?奥美创始人,这不是已经是个古人了吗?他怎么知道?

他确实知道,只是他说的不是我们。

他说的是他那个时候的年轻人,如果你要有兴趣掐指算一下,那拨人就是后来美国传说中的最靠谱、最脊梁、最中坚力量的婴儿潮一代。

这说明两件事,一是大部分对80后或者90后的偏见其实并非来自于他们出生的时间,也与他们出生之后所遭遇的一些成长环境和特殊境遇无关,所有这些指责都是老一代人对新一代人的看不入眼,在老一代人年轻的时候,他们就是这样被指责着过来的。话说到这儿,看到80后们现在已经开始指责起90后的不足和非主流来了,我心中总是备感凄凉,80后这么快就老帮菜了,让我们这些70后情何以堪啊。

第二件事是你应该怀疑,就像怀疑自己一样------我们80后都有这样的美德对吧------那些说婴儿潮一代最靠谱最脊梁最中坚的人都是谁?这世界上大部分问题都是不经想的,没错,就是婴儿潮他们自己说的。自己夸自己的话,我看要打100个折扣都不止。所以就随便听听吧。

这就说到一个核心的问题了,从1945年到1965年,美国的婴儿潮一代整整生了20年,在最近这些年里,他们一直霸占着这个社会的话语权,我们是不是都被他们骗到了?在罗伯特·希勒看来,美国这么多年处在``非理性繁荣''当中,都是婴儿潮一代祸害的。

我们先不管他说什么,可以看看自己身边的事。现在房价这么高,80后们买不起房,这都是谁导致的?难道全是无良开发商和唯利是图的地方政府吗?一定还有拿着房子等高价出售或者已经出售的人,你不要说他们都是温州人,他们大多是我们不那么陌生的上一代人。在中国房地产的``非理性繁荣''当中,你会看到这样一群人,他们跟婴儿潮一代的年龄差不多或者稍小一点,60后或者70后们,他们可快乐地享受着繁荣呢。

在美国也是一样,对他们而言,更不幸的是:虽然美国房地产已经崩溃过了,但年轻人的购房需求在美国的主流媒体中几乎完全没有被反应出来,因为主流媒体的从业人员大部分都是婴儿潮时期出生的那代人。房价下降对他们来说是一件空前倒霉的事,他们的资产会因此缩水。所以格雷格·伊斯特布鲁克认为``媒体对这一看法的宣传报道,在损害年轻人利益方面起到了推波助澜的作用''。

在罗伯特·希勒看来,婴儿潮一代在1982年的这轮繁荣或者叫泡沫中起到了推波助澜的作用:1980年代,婴儿潮一代开始步入中年,他们开始为自己的养老金问题担忧,而他们的养老金大部分都在股市当中,所以他们有义务把股市推高,不断推高。

2005年,我第一次听到谢国忠对油价高企的解释时还觉得很新奇,他说有相当大一部分原因是从事原油期货的分析师们太多了,不断炒作结果导致价格上涨。经过我们的房地产繁荣泡沫、次贷危机、我们自己的2007年股市大繁荣之后,我们都理解了这一点。婴儿潮一代们啊,就这么把资本市场推向了一个谁都承受不了的一个高度,然后,我们看着在他们拿到了一些分红奖励之后,股市就崩溃了。

而另外一个相关因素还在于,婴儿潮一代刚一工作就赶上了1970年代的石油危机和股市低迷,那个时候他们还没有什么多余的钱来理财。转眼间他们就迎来了繁荣,对于从来没有经历过大萧条的这一代人来说,还有什么比恣意妄为这样的词来更好地形容他们呢?当然,反过来我们也可以回忆一下这本杂志创刊号里说的年轻人赶上熊市是一件多美好的事:虽然现在年轻人在为婴儿潮一代的老家伙们埋单、忍受他们制造的泡沫,但未来还有更年轻的人会被我们蹂躏的。

当然,这种没心没肺的话最好还是忍着不要说出来,要像婴儿潮一代那样,把该享受的都享受到,把人生的红利都赚足,然后把问题甩给未来那帮更悲催的一代人。

如果深究起来,婴儿潮一代比罗伯特·希勒说的可能更具破坏力一些。比如这群没心没肺、没有经历过大萧条的家伙们创造了消费主义的人生观,不存钱爱消费,从他们一出生的时候,战后乐观的他们的爹妈们就卯足了劲添丁进口,然后给他们准备了郊区的大房子,1950年代的繁荣也是依赖于这一代人一出生就有的强大气场。接下来就是他们拼命占有资源,破坏了环境,拥堵了交通(这一点北京的年轻一代也会对上一代的无节制汽车消费有点感触吧),最可恨的是他们因为太懒所以搞了全球化,搞得年轻一代找不到合适工作------难道要去跟墨西哥人和中国人去竞争吗?

罗伯特·希勒列了12条导致市场产生泡沫的原因:1.市场经济的疾速发展和业主社会(我们要知道``业主社会''的提法来自于婴儿潮一代的代言总统小布什);2.政治文化变迁促进商业的成功;3.新的信息技术;4.支持性货币政策与格林斯潘对策;5.生育高峰及其对市场的影响;6.媒体对财经新闻的大量报道;7.分析师愈益乐观的预测;8.固定养老金计划的推广;9.共同基金的增长;10.通货膨胀回落及货币幻觉影响;11.交易额的增加:贴现经纪人、当天交易者和24小时交易;12.赌博机会的增加。

如果我们看这些原因,会发现5、6、9、11都与交易者增加相关;政策和社会变化导致了1、3、4、8。这些的背后都与庞大的婴儿潮一代的政策影响力有关。而且一个被认为是重要影响的因素是:在世人口最多的一代人会左右市场工具的变化,在2000年前,婴儿潮既是最多人口一代,又是年富力强最有影响力决断力一代。

还是说我们自己吧。威廉·伯恩斯坦说过这么一个事,你投资的目的是得到当期和未来的收益,这就像你为了得到牛奶而买一头奶牛,你是希望奶牛的价格更高还是更低?我们的运气没有坏到家,在婴儿潮一代把资本市场和全球经济搞得一团糟退出舞台之后,我们看到了一头便宜一点的奶牛,那还是抓住这个机会比较合适。

不管怎么说,年轻是最好的事,埃罗尔·菲林说过一句话:任何一个人,如果他死时银行里还有超过1万美元存款,那他就是一个失败者。把这句话写墙上,告诉婴儿潮一代们:我们不但能活过他们,他们还应该把钱给我们留下来。

\begin{center}\rule{3in}{0.4pt}\end{center}

\textbf{《广告大师奥格威》}

大卫·奥格威

生活读书新知三联书店

2003年9月

定价:8元

\textbf{《投资者宣言》}

威廉·伯恩斯坦

机械工业出版社 2011年2月

定价:35元

\textbf{《非理性繁荣》}

罗伯特·希勒

中国人民大学出版社 2001年4月

定价:30元

\section{你从事的是娱乐业}

安迪·格鲁夫作为史上最着名的一个偏执狂,有一天他在办公室对戈登·摩尔说,如果我们被踢出董事会,他们找到一个新的首席执行官来,你说他会怎么做?戈登犹豫了一下说,他会放弃存储器的生意。格鲁夫死死地盯住戈登·摩尔:你我为什么不走出这扇门,然后走回来,我们自己动手呢?

在他们说这话的时候,远在华尔街的另一个安迪,安迪·凯斯勒作为一个菜鸟分析师,正在研究Intel。他预测在经历了一个日本制造冲击(那时候美国的敌人还是日本)、存货量大大增加引发生产过剩之后,芯片制造业应该凭借技术进步迎来反弹,他给出的建议是``买入''。

这个时候市场上传来的消息是英特尔裁员、存储器生产线停产。我们现在知道这是安迪·格鲁夫们走出那扇门的结果,但没有多少人在意这个事,于是股价应声而落了,安迪·凯斯勒栽了。

当然,这只是暂时现象,安迪·格鲁夫把英特尔变成了一个微处理器的供应商,进入它们最黄金的时代,很快英特尔的股价就涨了三倍不止。这就是传说中的价值投资。

安迪·凯斯勒告诉我们,你要懂得一个企业在想什么,还要了解技术趋势,还要懂行业的竞争态势。有了这些,你就成功了。

在安迪·凯斯勒从贝尔实验室跑到华尔街做分析师的时候,他的上司康奈尔给他讲了一番话:``分析师应该做的第一件事就是完全沉浸、埋首于自己所负责的产业里,你必须彻底了解每家公司------产品、服务、管理、问题、怪癖、八卦消息,以及和公司有关的任何东西。你得进行公司拜访,参加研讨会,参观贸易展,阅读产业报告与信息,搜集与消化所有东西。这样你才能对产业的基本面作出合理判断,得出如何投资于这一产业的具体策略,再根据所得知识,推荐买进或者避开哪些股票。''

这当然是分析师的功课。但它也告诉我们,如果你觉得你是一个价值投资者,那类似于这些东西也是非做不可的。我们知道价值投资真是一个好东西,而好东西一定是要付出成本的。

不过,按凯斯勒的说法,这一行作为娱乐业的分支,即使是价值投资,有的时候也得靠蒙。他的第一单漂亮活就是靠这个办法混来的:他不断地刷新消息,终于等到了摩托罗拉的一次误操作,把盈利数据提前公布了,等它们把数据收回来的时候,安迪·凯斯勒已经把消息在华尔街散播出去了。但作为一个有理想的人,他说如果不是提前五分钟,而是提前一年这有多好。

提前一年,那就是II的金牌分析师,II是《机构投资者》的简称,虽然凯斯勒不断调侃这个以评级着称的杂志,但作为一个不旦有理想,而且认真、正直、聪明的人,从五分钟到一年,就是娱乐业与投资业之间的区隔。

我要特别说明一下,我喜欢安迪·凯斯勒这个坏人,他很有可能还是一个混蛋,我在看他写的这本《华尔街的肉》的时候,就会经常很警觉地环顾四周,想我身边到底有没有这样的家伙,说不定哪一天我会被他写在一本什么书里成为一个反面角色。在这个领域里我喜欢的另一个坏人叫迈克尔·刘易斯,他为《华尔街的肉》这本书写了序,他也喜欢扮演一个``告密者''的角色,他的经典作品是《说谎者的扑克牌》和《大空头》。他在这两本书中告诉我们类似于所罗门这家投行是如何用衍生品发财的,次贷是如何被一群天才发明出来的,而``信用违约掉期''这种拗口的名字又是如何被另一群天才的模型控们发明出来对付前面那伙天才的,总之他告诉我们:聪明人是如何聪明的,还有就是聪明人是如何变傻的。

安迪·凯斯勒更好的一点还在于,他很认真地教我们如何做一个好的投资者,他在书中会讲,对投资如何理解,为什么不要相信各种技术分析图,为什么股票波动时,它们自己会说话------并且为什么通常来说都是谎话\ldots{}\ldots{}还有就是,为什么他会把华尔街的工作视为娱乐业。

让我喜欢他们的并非仅仅如此,还有一个因素是这两个人都特别喜欢游走于东西海岸之间,游走于技术和华尔街之间。我以前提到过安迪·凯斯勒的另外一本书《我们如何来到现在》,而迈克尔·刘易斯则有一本《将世界甩在身后》,那都是很有趣的书。

最后一个让我爱上他们的理由是,他们都能从华尔街这种鬼地方抽身而退。在投行券商界,分析师虽然不如做业务的,每天的诱惑也不少,从分析师晋身为业务也是一条捷径,从一个动辄七八位数美元年薪的诱惑中脱身出来,还愿意写书这算得上是一种美德吧。赚钱不是那么重要的事,这事他们说出来更有说服力一些。

所以你在他们的书中会看到安迪·凯斯勒自己、迈克尔·刘易斯自己,或者他们写的吉姆·克拉克、迈克尔·巴里\ldots{}\ldots{}这些聪明人都不是以赚钱为目的。在上面提到的人中,吉姆·克拉克很有趣,他一生创办了三个十亿美元市值的公司,硅谷图形公司、网景还有一个叫永健(Healthon)的公司。

这里唯一一个小人物是迈克尔·巴里。

他斯坦福毕业,住在圣何塞,但他不能算是硅谷人士,他的本行是医生,业余爱好是做投资。是他发现了次贷的秘密,并且发现用``信用违约掉期''(CDS)可以赚来大钱,于是他成立了一个叫Sin的公司(没错,他这个公司名字就是来自于三角函数里的正弦)。

但是与狂赚40亿的约翰·保尔森相比,他作为一个毫无华尔街背景的家伙可没暗地里撺掇机构去发行CDO,然后再拿来对冲。实际上他只赚了几百万美元。他更像那个不靠谱的正弦公司的名字一样,追求的是聪明人的快乐。

在介绍《我们如何来到现在》这本书的时候,我曾经说安迪·凯斯勒从贝尔实验室出来混华尔街,然后自己又去创业做对冲基金,从一千万做到十个亿,然后在互联网泡沫破裂之前结清出场。在《华尔街的肉》里,他说了如何选择泡沫的原因:在他从新技术公司赚了钱之后,他接连遇到两伙气度不凡的坐着豪华轿车带着保镖裹着头巾的中东人,他们都表示要出资五亿美元交给他来打理。他与他的合伙人说:``我们不拿这些钱,自然会有人拿,这十亿美元会找到栖身的风险投资基金和对冲基金,而且会搅进市场的混水里。我们不仅不该再接受任何钱,还应该开始把手上的资金还给投资人。''

安迪·凯斯勒说,华尔街有句名言,高峰和谷底都不会有铃响。市场的顶点和谷底都是事后才知道的。我们还有下一桩最好的事在等着我们,那些豪华轿车、保镖和撒在我们身上的钞票是我们的小铃铛。

在我们这里,这个故事通常有另外一个版本,为什么营业部门口看车的、卖盒饭的老太太炒股都赚了,因为她们发现自己的生意好做的时候,也应该是退场的时候了,那些供不应求的盒饭就是她们的小铃铛。

\begin{center}\rule{3in}{0.4pt}\end{center}

\textbf{《只有偏执狂才能生存》}

安迪·格鲁夫

中信出版社 辽宁教育出版社

2002年8月

定价:22元

\textbf{《大空头》}

迈克尔·刘易斯

中信出版社 2011年1月

定价:42元

分析师应该完全沉浸、埋首于自己所负责的产业里,你必须彻底了解每家公司------产品、服务、管理、问题、怪癖、八卦消息。

\textbf{《华尔街的肉---我从股市绞肉机中死里逃生》}

安迪·凯斯勒

上海人民出版社 2006年11月

定价:22元 \# 人人都爱非理性 \#

投资这回事,被当成又体面又高智商的赚钱手段的历史并不长,在人类文明当中,除了中国人觉得做生意这事不靠谱人品不好,其他国家的人民也不待见这个东西。我们都有一个词来说它们,``不劳而获'',这还真是一个很严厉的评价。

所以一想到现在可以有专门的杂志来教大家如何``不劳而获'',我还真是觉得这社会进步了。当然了,即使社会进步至如此,也不是所有人都能得到``不劳而获''的机会。比如我就觉得我不适合投资,这其中的原因显然不是我道德高尚,而是我总能看到各种怪模怪样的理论,这种事听得多了,你就会觉得你掉到一个赌场里------像我这种心理素质不好的,如果又担了一个``不劳而获''的名声,还被赌场吓住,那真是一件让人伤感的事。

比如凯恩斯,没错,就是那个约翰·梅纳德·凯恩斯,搞出凯恩斯主义的那位大师,他在投资理财界最大的贡献是讲了一个全城选美女的故事,从100张照片里选出6个美女来,但最后的优胜者并不是某个美女,而是选对了所有美女的那个人------而选对美女的能力在于:你要知道别人会选谁。所有理财故事的核心都在这里,你做什么不是最重要的,大众会怎么做才是重要的。比如市场上的某家公司的好坏不是重要的事,你如何判断公司的好坏也不重要,重要的是大家如何判断这家公司。你要有把握判断公众在想什么,即使这公司烂到天上去,也没有关系。

我还顺带想了下另一个大师沃伦·巴菲特的话,他说,你要在别人恐惧的时候贪婪,在别人贪婪的时候恐惧,全世界人民都记住了这句话,我也记住了,并且我相信你要是能在第二天早晨猜出全世界人民是在恐惧还是贪婪,你就赢定了。

但我看大多数人在赌场里并没有赢,因为大家总是相信自己是最聪明的,即使你看到了公司很糟糕,但你更发现,噢,还有这么多芸芸众生在抢着买进,你肯定要不由自主地去做那个最后出场的人,结果大家都知道的,有另一个词叫``博傻'',在跟别人比傻的领域里,你永远不要低估自己傻的程度和能力。你也赢定了。只要等等,再等等,你就从一个很光荣很体面的价值投资者,变成了市场中的乌合之众。

理性是一件很牛的事,但在投资界很多时候它都是一个稀缺品。在所有股市的评价当中,我听到最多的是有关信心的问题,只要说到这个信心的时候,基本上就是理性最稀缺的时候。所以最后有关投资学的研究总是升级到行为学的高度,而各种行为学研究的东西总是让你意想不到的行为,在我目力所及之处,除了这里错了或那里的想法不对之外,还没见过投资者能做对的事。丹·艾瑞里写过一本《怪诞行为学》,它的副标题叫``可预测的非理性'',说的是各种各样非理性的事。其中有一个案例说的是,如何界定一张球票的价格,在杜克大学年度最隆重的赛事中,没有买到球票的人愿意花175美元,因为他觉得这与没办法到现场看球赛付出的替代成本差不多,而手中有球票的人愿意出让的价格是2400美元------他觉得这会成为他们一生的一笔财富,要是把这财富换出去,可得换回一大笔钱才行。这两者显然没有一个靠谱的,一想到在我们这个世界里,到处都是这样的非理性行为,然后你作为一个价值投资者去跟他们博傻,就会觉得人生灰暗。

并且这还不是最让人绝望的。丹·艾瑞里还讲了另一个让人发指的故事。你在eBay网上看到一个好东西,你拍了它,然后你等着拍卖结束------也许开始的时候你不会觉得怎么样,但因为你下手早,你有充分的时间去想象拥有它的快感,第二天它还是你的,到第三天你就会得到它了。这个时候来了一个家伙要跟你抢,你会怎样?第一天你出的是一个理性的价钱,这时候还会吗?艾瑞里说,正如我们所猜测的,那些出价最高、参与时间最长的拍卖者,也是虚拟所有权感觉最强烈的人,当然,他们处在一种非常脆弱的地位:一旦他们认为自己已经是所有者,就会强迫自己一再出高价防止失去这一地位。

投资者的世界比这个要稍微复杂一点,但如果你看惯了不忍心止损的,你还是会发现拍卖者的影子,只是换了另外一个维度------因为他相信最初的那个心理卖出价,至少是买入时的那个价位。

当然,在这个由欢乐的我们组成的非理性世界里并非只有悲剧发生,凯恩斯的选美故事有无数人写过,但写得最有价值的应该还是詹姆斯·索罗维基。他是少有的愿意赞美``群体的智慧''的人之一,虽然群体中的每个个体都会搞出诸多的非理性出来,但最终还是会趋向于理性,并且比高智商的那群人做得还要好。

最成功的也是最有说服力的例子是谷歌,比如你搜索``滑铁卢战役'',它并不负责告诉你谁才是解读这场着名战役最权威的历史学家,也不是想着把最有价值的那部文献找出来,在拉里·佩奇的世界里,他的PageRank可不管这些事,他只负责解决找出哪个网页被更多的人看过了,链接在它的页面上的链接更多,那它就是最好的。通常来说,最有价值的那部文献也就是这样被找出来的。索罗维基说,这是群体的智慧。

这群体的智慧除了让谷歌赚了大钱,对我们也并非一无是处,就像单个的我们总是被博傻的洪流裹胁着成为投资界的甲方一样,我们也可以利用它来作出正确决策,比如在8个人的团体中,每个人都知道一些有利于自己的信息,如果每个人都讲出来,尽管这些信息是片面的,但最后还是会综合成相对完整准确的信息,如果你们是一组投资者,但愿群体的智慧不是``非理性×8'',而是像谷歌的PageRank一样是个好结果。

除了谷歌,群体的智慧还有更多好处。杰克·戈德斯通写过一本小书叫《为什么是欧洲》,在500年前还远远落后于中国的欧洲为什么奇迹般地成了文明的驱动者,并且相对于此前的几千年来说,让人类文明的发展速度呈现出爆发式的增长。他总结了若干条,其中有一条或者与詹姆斯·索罗维基有关。在1665年,英国皇家学会,旨在促进科学知识增长的最早的机构之一,提出一个观点:所有的新发现都应该在尽可能大的范围内尽可能自由地传播。亨利·奥尔登堡说``保密不利于科学进步'',并认为科学家们应该放弃用创始人和发现者的身份来换取别人认可的企图。抓住了科学的本质,知识在应用时不会被用坏。因此在没有丧失其价值的情况下可以进行广泛传播。

看到了吗?科学的本质。知识被越多的人用,它的潜在价值就越大。即使全世界投资者都在非理性,那也不用怕。你可以说:那又怎么样,我们在创造历史啊。

\begin{center}\rule{3in}{0.4pt}\end{center}

\textbf{《为什么是欧洲?------世界史视角下的西方崛起》}

杰克·戈德斯通

浙江大学出版社 2010年7月

定价:34元

\textbf{《怪诞行为学------可预测的非理性》}

中信出版社 2010年19月

定价:45元

\textbf{《群体的智慧------如何做出最聪明的决策》}

詹姆斯·索罗维基

中信出版社 2010年10月

定价:33元 \# 天才,还是该死的无耻之徒 \#

投资界是盛产天才的地方,何止是天才,简直充满了神启者、通灵者。投资界在这方面的鼻祖是来自田纳西州的克拉夫林姐妹,姐姐维多利亚最大的客户是接近于老年痴呆的铁路大王范德比尔特,你就想像一个比比尔·盖茨还要富有三四倍钱的人傻了,这真是一件欢乐的事,可他身边有这样一对姐妹花替他炒股票,她们该是多成功的股票经纪师啊!她们也因此成为第一个成立经纪公司的女人。

她们的厉害还不仅在于此,她们也是最早在美国做报纸的女人,最早鼓吹自由恋爱的女人,维多利亚还是最早竞选美国总统的女人,她们的报纸最早把《共产党宣言》介绍给了美国人\ldots{}\ldots{}当然,在为范德比尔特写传记的小爱德华·雷内汗看来,她们还是最早对自由恋爱进行收费的人。

对于历史人物不应该这么刻薄,我也不想探讨自由恋爱到底应不应该收费。我们这一次想探讨的是天才们和自认为天才们的人到底能不能获得成功,这其中当然也包括投资领域里的成功。

罗伯特·希勒在总结市场有效理论的时候说,既然我们没办法预测其股市第二天的变化,那么也一定没办法预测任何一个时间段的变化。所以理论上聪明人和笨蛋的投资收益应该是相等的。但他随后又否定了这个说法,确实有人在股市上赚了钱,如果分析赚了钱的人的特点,可以发现``聪明和努力的人长期能获得更好的投资业绩,这一点不容置疑,虽然与市场有效理论相悖''。这其中的理由是什么,他没有说。投资界之外的马尔科姆·格拉德威尔引用过1970年代的一个``1万小时理论'',但凡比尔·盖茨、约翰·列侬或者画鸡蛋的达·芬奇(你真信他会画那玩意画1万小时吗)都有每天3小时以上持续10年的训练,1万小时以上,你就接近于成功了。肯定有人问了,我在股市上已经赔了十几年的钱了,为什么还没变成巴菲特?丹尼尔·科伊尔专门为它写了一本书,就叫《一万小时天才理论》,这本书告诉我们,你光摩拳擦掌每天浪费3小时是没用的,正确的方法很重要。

科伊尔说,好比你看巴西人踢足球,除了从小就练,他们还有特殊的训练办法,室内足球,场地小,更贴身,传球速度更快距离更短,这样的训练他称之为``精深练习''。他认为,足球最重要的事是准确的传球,在室内足球训练中,控球传球的转换率是普通球场的6倍,这保证了巴西人踢得比别人更好。这也解释了为什么我们在股市中练了1万小时还没有发财,我们中国的那些足球明星们也练了不止1万小时了,我怀疑他们就是听信了达·芬奇画了1万小时鸡蛋的那帮蠢货。看看他们!自信一些,你的人生还没跌到谷底呢!

科伊尔解释了训练中大脑如何工作的问题:所有的反应和大脑的工作都来自于神经元的工作,所有动作都是神经纤维之间相互沟通的结果;在一开一合之间,对事物作出判断;而这些依赖于大脑中一种叫髓鞘质的东西。关于这个词,不需要懂太多,你就当它跟肌肉一样,是反复练反复练之后的结果:你的神经带宽中的髓鞘质含量更多,练出块儿来,神经真正``大条''起来,这就是更宽的带宽,信息传输量巨大。你还拨号上网呢,别人已经1个G无限量上传下载了。

我只能怀疑投资界也同样道理一样。分析处理的速度更快,掌握的数据更多,这就说到一个通常总是让我们难堪的一个词了,勤奋。

有一个叫罗伯特·加内特的美国农村青年,胸怀大志,想参加1896年的雅典奥运会的铁饼比赛,但问题在于,一他没扔过这玩意,二他没有铁饼可以练习。这个老兄找一铁匠照着画片上的样子做了一个,重30磅,每天练,没什么进展,跟传说中的那些记录相去甚远。沮丧的青年加内特还是到了雅典,到了那里才发现,我擦,原来是铁皮儿包的木头啊?他就很轻松地把记录提高了19厘米。这个励志故事告诉我们:平衡增加的工作努力会带来几何倍数的改变。

讲这个故事的人叫奥斯汀·豪,这是一个阴险的设计师,他写了一本书叫《设计师不读书》,但他不但自己写书,还讲这样的励志故事,如果只看书名就忘了勤奋这回事,所有的钱就都被他赚了,而你还在想为什么你的创意不够天才呢。按通常的理解,创意这个行业比投资业要更容易走上``通灵''这条路,但我了解的情况正好相反,他们更爱说勤奋这回事。比如苏格兰人大卫·奥格威就爱引用家乡谚语:辛勤工作不会置人于死地。他说,人们死于负担、心理冲突和疾病,但他们不会死于辛勤工作。员工愈努力工作,愈能保持健康愉快(如果你是员工,请自动屏蔽他说的后半句)。

乔治·路易斯是另一个不讲道理的狂放的创意天才,1990年10月,国际象棋大师卡斯帕罗夫与卡尔波夫在曼哈顿对决,他用两个人的侧影鼻子对鼻子设计了海报,阴影之处正好是一个国际象棋棋子的轮廓,连两个人自己都惊呆了,乔治·路易斯说:``创意一直存在,就在你周围漂浮。\ldots{}\ldots{}你不是在创造,你是在发现。''天才吧?但他背地里非常用功。他说:``这难道不是生活的全部吗?如果在一天结束的时候你没有精疲力尽,那你一定还有所保留,那么你一定没有把自己全部投入到工作中,你只不过是个懒惰而不负责任的人!

如果我们要为这些天才再总结一个素质的话,那么就可以说是热情。奥斯汀·豪跟乔治·路易斯一样,他也飙过同样的观点:''每次我拿到客户支付的酬劳,心里都有一种负罪感。这是因为,设计工作中95\%的部分,我都可以免费为他们做,谁让我那么痴迷呢!``上帝饶恕这些得了便宜还卖乖的人吧!

不过,有一点我们可以为大卫·奥格威续个狗尾:更多勤奋的人可能最终死于折磨。因为他不热爱他做的那个事。

我得再强调一次,励志不是一件讨好的事。所以最后我还是决定犬儒一次,你可以引用德国在一战期间的军官手册中的一段话来为自己辩解。在手册中说雇员分为四类:1
. 非常聪明而且勤奋(理想的雇员);2.
非常聪明但是很懒(一个该死的无耻之徒,但是倒不会有什么危害);3 .
愚蠢而且懒惰(成不了事,因为是一个废物);4. 愚蠢但是勤奋(非常危险)。

我们可以把这个做成一个象限图。如果你不勤奋,对工作也没有热情,那么你可以说你是一个善良的好人:这世界上最可怕的事情就是''傻X
敬业``,我可不想给世界带来太多的危险。但实际情况可能是,我很相信这一点:在大多数情况下,大多数人都应该划在''该死的无耻之徒``这个象限里。

而投资界的人士们,除了第一条可以保证你赚到钱,第二条你会错失好的投资机会和产品;第三条你活该赚不到钱,第四条,哈利路亚,我们一起祈祷吧。

\begin{center}\rule{3in}{0.4pt}\end{center}

\textbf{《设计师不读书》}

奥斯汀·豪

重庆大学出版社 2010年10月

定价:25元

\textbf{《蔚蓝诡计》}

乔治·路易斯,比尔·皮茨

华文出版社 2010年12月

定价:49.8元

\textbf{《铁路大亨》}

小爱德华·雷内汗

中信出版社,2009年4月

39.00元

\section{伯恩斯坦错了吗}

1995年,印度新德里的人民党政府关闭了这个城市唯一的一家肯德基,原因是卫生检疫人员在厨房里发现了一只苍蝇。爱德华·卢斯说,熟悉德里市饭店卫生标准的人都不禁要怀疑这只是一个借口。卢斯是《金融时报》驻印度的记者,他写了《不顾诸神------现代印度的奇怪崛起》,这故事告诉我们,这个世界不仅仅只有崛起的中国才会有这样的双重标准的事情发生,新兴国家们多多少少都有一点这样的爱好。

威廉·伯恩斯坦说,投资者很容易相信新兴国家的高速增长会为投资者带来高额的回报,但实际发生的往往是资本市场回报率低于全球平均水平。他为此总结了三条原因,其中第一条是国家跟公司一样,跌到一个较低估值的时候才会有入场抄底的机会,而新兴市场往往还一路高歌猛进呢,投资者根本没遇到合适的进入时间。第二条是新股发行过多,降低了整体的回报水平。第三条是政府不能完善的保护投资者利益,也没有完善法制来制约政府和大股东的随意掠夺和破坏公司的价值。肯德基在德里的那只苍蝇,可以为此作证。

你不知道一个看着还不错并且有着竞争力的企业会发生什么事,不确定的事太多,就让伯恩斯坦们感觉风险被扩大了。

但你要知道,在这个世界上,如果你过于谨慎,可能什么也得不到,我们不妨暂时忘了伯恩斯坦,看看如果放手一试会怎么样。

我们可以看到一个繁荣的例子。``佩德罗二世统治了巴西50多年,其间经历了扩张和现代化。和他的父亲一样,佩德罗二世在许多问题上都非常开明,也善于接受新技术,比如电报、电话和铁路------他将铁路视为紧密连接辽阔疆土的利器。在佩德罗的长期统治期间,相继出现咖啡种植和橡胶生产的繁荣\ldots{}\ldots{}''你看,开明、新技术、现代化、繁荣,这很有金砖四国今日的风范,但如果你知道它出现在1810年代开始的一段时间,你就知道如果你真的投资了当时的巴西,你在未来200年里得到的会是什么。

我们在评价一个洋洋得意的成功人士的时候,会提醒他:想一想吧,是你的天分还是你的平台还是一个机会才带来的这些成功!我们知道多数成功人士总是摆不平这个问题,他会把这个完全视为自己的天分和能力。很遗憾的是,对于一个崛起的国家的管理者,他们也会经常产生这种飘飘然的幻觉。印度人总认为世界应该跟印度学习好多东西,而且不仅仅是乔布斯爱的那些禅宗和静修,巴西的卢拉总统在摆脱了IMF的控制之后,很高兴有机会``作为一名借几个雷亚尔给IMF的总统而被载入史册'',中国和俄罗斯,哦,如果这个世界上有专门可用来投资的``说大话''这种金融衍生品,我相信,大家会蠃来好多块金砖回去的。

言归正传,在你踌躇满志放手一试这些新兴市场的时候,除了考虑持久性和``说大话''带来的溢价和庞氏骗局(这个倒不单是新兴市场的问题,冰岛和金猪四国这样的市场更容易做出这样不靠谱的事)这样的风险之外,你还要观察它们是靠输出能源、原材料的繁荣还是大搞基建带来的繁荣,以色列当年曾靠这个投资,获得了成功,但对于迪拜、对于中国、还有1980年代的日本以及2000年代的美国来说,事情未必如此。

除此之外,你还要考虑你是在投资那里的``人民''。这个时候,你要小心你投资的是不是抓出``肯德基的苍蝇''的那一坨人。

《一件T恤的全球经济之旅》从另一个角度说明了这个事。那本书中说,转基因在美国可以得到很好的收益,但在中国则正好相反,比如有一种有知识产权的叫苏云金芽孢杆菌的种子,解决了困扰全世界的棉铃虫的问题,美国因此大量减少使用杀虫剂。而中国棉农则转而使用新的非棉铃虫杀虫剂,最后导致的是杀虫剂使用种类从20种降低到6种之后又上升到18种。皮翠拉·瑞沃莉,这本书的作者说,``如果没有教育、公共支持和良好的培训组成的良性循环,像这种新技术带来的只是让富人更富,让穷人变得更穷''。如果是这样,伯恩斯坦当然还是会犹豫,这回报究竟在哪里。

与伯恩斯坦的新兴市场投资理论相关的书里,我最爱的是一个爱尔兰经济学家写的《与全世界做生意》,它的副标题是``一个经济学家的环球冒险'',从苏丹、埃及到印度、中国还有南美。这个柯纳·伍德曼告诉我们的其实和杀虫剂告诉我们的一样,商业文明的体系有多重要。比如他说``不管是骆驼还是信贷,整个系统全仰仗信任来启动:你得相信链条上的下家会还钱'',他还说,``经济要蓬勃发展,若干基本要素必须先行就位。可靠或是可核实的信息便是其一,负责任是其二''。常识吧?啰嗦吧?但这还真是最根本的东西。伯恩斯坦可能最怕的也是在这个常识里就产生问题,所以他会说,如果一个政府不能解决孩子们的玩具含铅问题,那你怎么能相信它会保护投资者!

是不是过于悲观了?这个世界可不是由精打细算成天想着手里几个闲钱如何保值增值的理财达人创造出来的,所以伯恩斯坦虽然说对了一个真理,但那是给中产阶级们准备的。我们,一无所有的我们,什么都不怕对吧?我们失去的只是锁链对吧?

那好,回过头来,我们还可以看看冒险家们的乐园是怎么回事。

印度总理辛格的一个下属,对爱德华·卢斯说:``我想我们应该学习20世纪早期的美国历史,了解它的政治家是何其腐败。它证明这样的国家仍然能够崛起成为一个强国。''我得承认,我在看到这句话的时候还是狠狠地点了一下头的,不管是坦慕尼协会时期的美国还是哈定总统的那套腐败透顶的烂人班子,都没能阻挡美国崛起,在中国和俄罗斯,在印度和巴西,每个都是如此。我只能说,那些让伯恩斯坦裹足不前的因素,可能不是那么重要,在冒险家和没有底线没有原则的市场里,你是可以找到投资机会的。

伯恩斯坦说过一句话,他说,要乐于享受投资过程,就像一个手艺人爱手艺,但大部分人不像木匠,倒是像牙医。我觉得这是他否定新兴市场投资价值的所有理论基础,如果你不愿意移民到这些国家,你怎么可能信赖它,投资它呢?大师说的话总是又哲学又朴实,但我还想提供詹姆斯·比尔德的一句大师级别的话:一个老想着卡路里的美食家,就像一个老看手表的妓女。对于投资这些新兴市场想要赚钱的人来说,你真的没有必要因为它有投资价值,而爱上它。就像柯纳·伍德曼所说:商人不能对他卖的东西投入感情。是的,他必须垂涎优质的商品,因为只有这样的东西他才有信心去卖;可他又得小心翼翼地避免滋生出经济学家所谓的``禀赋效应''------仅仅因为你拥有某样东西,就觉得它比相应的货币价值更值钱。

\begin{center}\rule{3in}{0.4pt}\end{center}

\textbf{《不顾诸神------现代印度的奇怪崛起》}

爱德华·卢斯

中信出版社 2007年11月

定价:39元

\textbf{《赤道之南------巴西的新兴与光芒》}

中信出版社 2011年8月

定价:39元

\textbf{《与全世界做生意------一个经济学家的环球冒险》}

柯纳·伍德曼

机械工业出版社 2010年10月

定价:32元

\textbf{《一件T恤的全球经济之旅》}

皮翠拉·瑞沃莉

中信出版社 2011年1月

定价:39元

\section{怎么就成了巴圣人了}

这人世间有一句话是我很不爱听的,就是``世风日下,人心不古'',作为一个进化论的信徒,我觉得这世界虽然看起来乱七八糟的事越来越多,但总的来说,这是信息越来越开放的结果。

但大部分的时候,既然古人说的话能够流传下来,你总能找到它有万分准确的地方。比如在说到沃伦·巴菲特的时候,我就会想,他现在怎么就成为全世界人民的偶像了呢,真是世风日下啊!要说他做的那些事,比如买个高盛的股票,买了美国银行的股票------这不就是看好机会抄个底吗?怎么就变成``美国经济的稳定力量'',``让美国人重拾信心''了呢?叫好声音太多了,以至于我眼看着巴菲特一路绝尘地奔着巴圣人去了,就只好叹一下``人心不古''了。

所以我得提另外一个人,这个人在100多年以前------那个时候世风日下与否还真不好说------这个人一生中遇到过无数次金融危机,有那么几次,华尔街上人心惶惶的时候,所有人都站在街上等这个人出现,其中还有坐火车赶来的美国总统,我没记错的话,应该是克利夫兰吧。当然,这个人也没有让大家失望,他总是气宇轩昂地出现在华尔街,然后带领一批投机或者投资家们开几个会,安排你让点利见好就收,安排他拿出些压箱底的钱,一番资本运作,然后就力挽狂澜了。那才是整个华尔街和美国都望其项背。这个人叫约翰·皮尔庞特·摩根,就是我们现在还要提起的J·P·摩根。那个时候美国经济规模小,又没有美联储这样的机构捣乱,所以以个人之力居然就让他救市成功了。不感恩的美国人觉得有这么个人虽然是好事,但如果他良心坏了呢?长远之计,还是成立个美联储吧,于是在老摩根死的那一年,美联储成立了。众所周知,这个机构除了因为不作为而要为1929年的股市崩溃负责,现在看来也要为做了太多寅吃卯粮的事而要为2007年以来的金融危机负责。

这个蓝血派头的19世纪大亨,本职工作当然不是做美联储这些活,他最爱的是债券业务,他的客户也都不简单,要打仗了,要发展经济了,两个国家要合并了,新国家要成立了,凡是需要钱的时候,找摩根去。

他的客户数得上来的是阿根廷、墨西哥,后来还要加上法国、英国这样的老牌帝国,他被叫成``全世界的债主'',这可不是随便说说的。

另外还有一句很古典的话叫``世无英雄,遂使竖子成名'',我可没说沃伦·巴菲特不是英雄,他也不是竖子,但巴菲特持续几十年的22\%复利年回报,看起来基本上是石油危机之后一轮美国大牛市的受益者。这么说吧,一个鸭子在涨潮的水面上游,它觉得自己的位置越来越高了。巴菲特说,鸭子,你错了,越来越高的是水面,不是你!我们敬仰的巴圣人也同样离不开这历史的水面。摩根呢,就是在美国还是一个新兴市场国家的时候帮美国人搞钱的人,他基本上相当于``美联储+摩根商行(投资银行)+阿瑟·洛克(美国风险投资之父,投资了仙童半导体、英特尔等高技术公司)'',他还真是那个时代美国经济发展的重要推手------不感恩的美国人也得管那个时代叫进步主义时代。

当年的美国人非但没有把摩根当成圣人,还觉得他万分对不起美国,对不起全世界,是标准的坏人,就因为他太厉害了。现在的圣人和当年的坏人比起来,我要是祭起``世风日下''这个词来,也没有什么不对。

当然,老摩根与巴圣人不是一回事。巴圣人以价值投资出名,但如果仔细去想,这也要打一个大问号。看他最近投过的,被我们所熟知的高盛、比亚迪、伯灵顿北圣达菲铁路,还有突然出现的``高科技''IBM,这些如果被看成是价值投资,虽然价值投资和投资有价值的公司不是一回事,但我还是要为此重新思考评估一下这个词。原因还是有J·P·摩根在边上比着。

话说有一天已经移居布宜诺斯艾利斯的西奥多·韦尔受邀来到南卡罗来纳州的杰基尔岛见一个大人物,这位大人物打算集合一伙金融家夺下贝尔公司,``不仅要重建贝尔公司在电话业的统治地位,更要使之成为全世界最大的有线通信垄断巨头''。韦尔知道他不是在开玩笑,因为这个大人物就是J·P·摩根。韦尔后来出任贝尔公司的新总裁,他不是一个小角色,但也说不上是什么大角色,因为那个时候电话还不是最重要的东西,总的来说是东海岸的富人们用来娱乐的一个工具,电报才是商业世界里带来最多财富的工具,但老摩根不这么看,他更看好这电话的未来。于是他重组了AT\&T,这个垄断公司,一直到1984年被分拆之前,是``公用事业''托拉斯的绝对代表。

在韦尔之前的十几年,有一个人找到了托马斯·爱迪生,觉得他的电气生意不错,值得做更多的事,于是把他的公司与汤姆森·休斯顿电气公司撮合到一起了,就是后来我们知道的通用电气公司。显然组局的人还是J·P·摩根。这一次他抓住的是那个世界里最有智慧的头脑,你看,如果老摩根要做价值投资的话,这可都是持续一百多年的公司,并且他还拿着原始股。

这还不算老摩根被人所诟病的重组美国联合钢铁公司,也没有算老摩根如何打洛克菲勒的矿山的主意。这些事通常是被看成老摩根巧取豪夺的证据,但你也不能说它不算价值投资。我不觉得吉列刀片啊、可口可乐啊这些公司差,我也不觉得巴菲特的投资不好,我只是觉得我们对圣人的标准有点低。对于摩根来说,整个世界都是他的资产负债表,而巴菲特相比之下就是一个公司财务。并且我得再强调一次,价值投资和投资有价值的公司确实不是一回事,但你总要承认,投资有价值的公司这事更有趣一些。

对了,还有一个让巴菲特成为圣人的原因是他的慈善事业,这确实是一个伟大的想法,我对捐助这事也很支持,但我还是要多嘴地拿摩根来说事。话说摩根死的时候,老洛克菲勒打听了一下他的财产,他说了一句:``哦,没想到摩根竟然是个穷人。''

我们当然不能拿我们的标准来衡量穷人这个事,我们应该按洛克菲勒的标准来看,但不管怎么说,相对于做``全世界的债主''、摩根财团控制的股份和资本占有美国经济的四分之一、他可以派出去100多个美国各大公司的董事来说,它们的主人是个穷人,这是一件更酷的事。

史蒂夫·乔布斯在年少糊涂的时候做了一件蠢事,就是从百事可乐那里挖来了约翰·斯卡利。乔布斯跟斯卡利说了一句很``现实扭曲力场''的话:``你是打算卖一辈子糖水,还是打算改变世界?''我觉得沃伦·巴菲特就是那个卖一辈子糖水的人,别忘了,他的心头大爱可是可口可乐,这是他最得意的一个投资,而约翰·皮尔庞特·摩根,当然,我可以很庄重地说,他可以被叫做是``改变世界的人''。

\begin{center}\rule{3in}{0.4pt}\end{center}

\textbf{《工商巨子》}

荣·切尔诺

海南出版社 2000年6月

定价:65元

他保养得很好,举止庄重,代表那种威胁着牧歌式老美国的财界和工业界巨头的风范。

\textbf{《摩根全传》}(上、下)

罗恩·彻诺

重庆出版集团 2010年3月

定价:75元
